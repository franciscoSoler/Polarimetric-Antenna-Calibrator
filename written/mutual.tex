\documentclass[a4paper,10pt]{article}
%\usepackage[latin1]{inputenc} % Paquetes de idioma
\usepackage[utf8]{inputenc} % Paquetes de idioma
\usepackage[spanish]{babel} % Paquetes de idioma
\usepackage{graphicx} % Paquete para ingresar gráficos
\usepackage{grffile}
\usepackage{hyperref}
\usepackage{fancybox}
\usepackage{amsmath}
\usepackage{amsfonts}
\usepackage{listings}
% Paquetes de macros de Circuitos
%\usepackage{pstricks}
\usepackage{tikz}
\usepackage[colorinlistoftodos,prependcaption,textsize=tiny]{todonotes}

% Encabezado y Pié de página
\input{EncabezadoyPie.tex}
% Carátula del Trabajo
\title{ \author{} % Lo pongo para que el warning no moleste :p
\setlength{\unitlength}{1cm} %  Especifica la unidad de trabajo
\thispagestyle{empty}

\begin{picture}(18,0)
\put(0,0){\includegraphics[width=1.5cm, height=3cm]{Logo1.png}}

\put(10.5,0){\includegraphics[width=3cm, height=3cm]{Logo2.png}}

\end{picture}
\\[1.5cm]
\begin{center}
	\textbf{{\Huge Facultad de Ingenier\'ia \\ Universidad de Buenos Aires}}\\[2cm]
	{PROYECTO DE TESIS DE INGENIERÍA INFORMÁTICA}\\[0.5cm]
	{Calibración de una antena polarimétrica utilizando los acoplamientos 
	mutuos}\\[2.5cm]
\end{center}

\begin{flushleft}
	\textbf{ESTUDIANTE:}  Soler, Jos\'e Francisco \\[0.5cm]
	\textbf{PADR\'ON:} 91227 \\[0.5cm]
	\textbf{COORDINADORES:} Ing. Marino, Pablo - Ing. Wachenchauzer, Rosita\\[0.5cm]
\end{flushleft}
\date{} % Hace que no se imprima la fecha en la cual se compilo el .tex
 }

\begin{document}
	\maketitle % Hace que el título anterior sea el principal del documento
	\newpage

	\tableofcontents % Esta línea genera un indice a partir de las secciones y subsecciones creadas en el documento
	\newpage

	\section{INTRODUCCIÓN}
		Las antenas de arreglo de fase controlada son comunmente utilizadas en 
	aplicaciones aéreas y espaciales. Para obtener un buen comportamiento de las
	mismas es necesario que estén correctamente calibradas. Esto implica, que 
	las tolerancias de fases y amplitudes se mantengan y sus valores sean bien 
	conocidas por cada elemento del arreglo.
	
		Estas antenas en tierra son, generalmente, calibradas utilizando fuentes
    externas de campo lejano o cercano. Sin embargo, en aplicaciones aéreas o 
    espaciales, la utilización de dichas fuentes son imprácticas o difícil de 
    implementar. A su vez, si se opta por caracterizar todos los componentes, el
    tiempo que implicaría es excesivo. Por estas razones, surgieron distintos 
    métodos de calibración interna.

		En este conexto, se propone un nuevo método de calibración, el cual 
	aprovecha el acoplamiento mutuo inherente entre los módulos radiantes de la 
	antena.

	\subsection{DEFINICIÓN}
		Una antena de arreglo de fase controlada es una antena compuesta por un
	conjunto de módulos radiantes dispuestos de tal forma, que, aplicando la
	teoría de construcción y desturcción de ondas, la señal emitida logra ser 
	dirigida donde se desee.
		
		Calibración interna es colocar sensores que permitan la medición directa
	u indirecta de la potencia y fase de salida/entrada de la antena 
	polarimétrica.
		
	\subsection{CARACTERÍSTICAS}
		La utilización de una buena calibracíon interna es una problemática muy
	desafiante dado que es uno de los factores limitantes en la calidad de los 
	productos obtenidos con estas antenas.

	\section{MOTIVACIÓN}
		Hay numerosas motivaciones para la investigación de un nuevo método de 
	calibración:
		Primero, a la hora de adquirir imágenes satelitales es crucial que se 
	conozca perfectamente la señal emitida y recibida por la antena. Ya sea por 
	envejecimiento de los materiales, por variaciones de temperaturas u algún 
	otro factor, se observan dispersiones de las mismas. Hay dos enfoques para 
	encarar esta problemática:

		\begin{itemize}
			\item Controlando las dispersiones máximas que pueden presentarse 
			utilizando hardware más complejo.

			\item Corrigiendo dichas dispersiones haciendo uso de calibración 
			interna.
		\end{itemize}

		Al utilizar la calibración interna se evita aumentar la complejidad y 
	peso del hardware utilizado a costa de un mayor procesamiento de software, 
	logrando así, disminuir el costo de la misión.
		
		Otro motivo es que el método de calibración convencional posee numerosas
	limitaciones y falencias; la principal es que no abarca todo el sistema de 
	transmisión/recepción, dejando así parámetros fuera de control. 


		
	\section{Objetivo de la tesis}
		La presente tesis tiene como objetivo la investigación, análisis y 
	desarrollo de un nuevo método de calibración interna de una antena 
	polarimétrica que abarque el sistema completo de transmisión/recepción. 

	\section{Metodología de la tesis}
		En la presente tesis se investigarán los métodos de calibraciones 
	actuales para poder determinar que ventajas, desventajas, limitaciones y 
	diferencias hay entre cada una de ellos. Se buscará tener una visión global
	de esta problemática para poder determinar y entender que posibles falencias
	puede tener este nuevo método.

		Posteriormente, se investigarán las limitaciones que poseen las antenas 
	polarimétricas para poder determinar que recaudos se deben tener en cuenta a
	la hora de desarrollar el método.

		Luego, tomando todo en cuenta, se determinarán las hipótesis necesarias
	para que el algoritmo funcione correctamente. Para la validación del método
	se realizará un modelo de antena.
		
		Finalmente, se probarán, analizarán y documentarán los resultados 
	obtenidos de la comparación entre el algoritmo propuesto y el algoritmo de
	la calibración convencional. A su vez, se dejará asentado que posibles 
	mejoras se podrían aplicar al algoritmo para determinar otros aspectos que
	están fuera del alcance de esta tesis.

	\section{Estado del Arte}
		La calibración de una antena polarimétrica se ha estudiado en numerosas 
	ocasiones, abordando el problema desde distintos enfoques. En el siguiente 
	gráfico se pueden observar los distintos métodos utilizados.
	\[
		\begin{cases}
			hola\\
			chau
			\begin{cases}
			todo bien\\
			seee
			\end{cases}
		\end{cases}
	\]


	\todo{todo}

		Dichos métodos se han clasificado por la utilización o no de componentes
	externos a la antena.

	\section{Cronograma}
	\subsection{Cronograma detallado}
	\todo{agregar cronograma}

	\newpage
	\section{Bibliografía}

	\newpage
	\section{Currículum Vitae}
	
	\newpage
	\section{Materias Aprobadas}	
		
	\newpage
	\section{Plan de cursada}
	
	\begin{table}[!htb]
		\centering
		\begin{tabular}{|c|c|c|c|}
			\hline
			Código & Denominación & Créditos & Fecha \\
			\hline
			75.00 & TESIS & 24-OBL & 2 - 2015 \\
			\hline
		\end{tabular}
		\caption{Plan de cursada} \label{tabPlanCursada}
	\end{table}

	TOTAL CRÉDITOS: 24	
		
		El tema de 
	
	El presente trabajo consiste en estudiar las características de una etapa amplificadora multietapa. En este caso, el circuito bajo estudio es una 
		configuración circuital conocida como \emph{Cascode}. La misma se encuentra implementada en el presente trabajo por medio de un transistor de NMOSFET
		de doble gate de código BF966. El circuito equivalente que se obtiene al utilizar este transistor en esta configuración es equivalente al de 
		dos transistores MOSFET con valores para sus parámetros principales exhibidos en el cuadro \ref{tab001}. 

		\begin{table}[!htb]
			\centering
			\begin{tabular}{|c|c|c|c|}
				\hline
				Transistor & k & $V_{TH}$ & $\frac{W}{L}$ \\
				\hline
				T1 & 7.5 $\frac{\text{mA}}{V^2}$ & -1V & 1 \\
				\hline
				T2 & 100 $\frac{\text{mA}}{V^2}$ & -1V & 1 \\
				\hline
			\end{tabular}
			\caption{Parámetros de los transistores equivalentes del MOSFET doble gate BF966} \label{tab001}
		\end{table}

		El circuito equivalente simplificado del \emph{Cascode} se exhibe en la figura \ref{circ001}. 

		\begin{figure}[!htp]
			\centering
			 \input circuito1
	     \caption{Cascode implementado a partir de una transistor de doble gate (BF966)} \label{circ001}
		 \end{figure}


		\subsection{Polarización}

			\subsubsection{Parte Teórica}
				Se procede a verificar que el circuito exhibido en la figura \ref{circ001} se polarice de forma correcta. Dado que en continua los capacitores 
				actúan como circuitos abiertos, se puede simplificar el diagrama circuital del esquemático en cuestión. El circuito a polarizar entonces es el 
				exhibido en la figura \ref{circ002}. \\

				\begin{figure}[!htp]
					\centering
			 		\input circuito2
	     		\caption{Circuito de Polarización} \label{circ002}
		 		\end{figure}

				\indent Primero se procede a calcular la corriente $I_{D}$. Para realizar esto se recorre la malla de entrada obteniéndose la ecuación \ref{eq001}.
				Luego, con la ecuación característica de un MOSFET exhibida en \ref{eq002} y la ecuación \ref{eq001} se despeja el valor la corriente $I_D$ y de la 
				tensión de gate $V_{GS_{1}}$. Los valores obtenidos son:  $V_{GS_1} = -0.698~\text{V}$ y $I_D = 0.695~\text{mA}$.

				\begin{equation}
					I_D = \frac{V_{GS_1}}{R_S} \label{eq001}
				\end{equation}

				\begin{equation}
					I_D = k_1 (V_{GS} - V_{TH})^2 \label{eq002}
				\end{equation}

				Con la corriente $I_D$ se calcula la tensión $V_{GS_2}$ del segundo transistor de la misma forma que se calculó la del primero. El valor obtenido 
				es $V_{GS_2} = -0.916 V$. Luego de calcular la corriente que entra al drain de ambos transistores y sus respectivas tensiones $V_{GS}$, queda por 
				calcular las tensiones $V_{DS}$ de los mismos para verificar que ambos están actuando en su correspondiente zona de saturación. \\
				\indent Recorriendo la malla que involucra a la tensión $V_{GS}$ del transistor T2 y la tensión $V_{DS}$ del transistor T1 se obtiene la ecuación
				exhibida en \ref{eq003}. De la misma se despeja el valor de $V_{DS_1}$. El valor obtenido es $V_{DS_1} = 0.881~\text{V}$.

				\begin{equation}
					I_D \cdot 50 \Omega  + V_{DS_1} + V_{GS_2} = 0 \label{eq003}
				\end{equation}

				Por último, se procede a calcular el valor de la tensión entre drain y source del transistor T2. Para hacer esto, se recorre la malla planteando la
				ley de tensiones de Kirchhoff que involucra las ramas en la cual se encuentra la fuente de tensión $V_{CC}$, ambos transistores y la resistencia $R_S$,
				obteniéndose la ecuación exhibida en \ref{eq004}. Despejando la única incógnita de esta ecuación, se obtiene que el valor de la misma es
				$V_{DS_2} = 5.04~\text{V}$. 
				
				\begin{equation}
					I_{D} R_{S} + V_{DS_1} + I_D \cdot 50 \Omega + V_{DS_2} + I_D R_D - V_{CC} = 0 \label{eq004}
				\end{equation}

				Dado que ambas tensiones $V_{DS}$ dieron valores mayores a $V_{GS} - V_{TH}$, se corrobora que el circuito se encuentra polarizado correctamente. 
				En la figura \ref{tab002} se muestran los valores de polarización calculados. 

				\begin{table}[!htb]
					\centering
					\begin{tabular}{|c|c|}
						\hline
						Parámetro Medido & Valor Obtenido  \\
						\hline
						$I_{D}$ & 0.695 mA \\
						\hline
						$V_{GS_1}$ & -0.695 V \\
						\hline
						$V_{GS_2}$ & -0.916 V \\
						\hline
						$V_{DS_1}$ & 0.881 V \\
						\hline
						$V_{DS_2}$ & 5.04 V \\
						\hline
					\end{tabular}
					\caption{Resumen de los valores de polarización obtenidos} \label{tab002}
				\end{table}

			\subsubsection{Simulación y Valores Medidos}
				En la figura \ref{img001} se exhibe el esquemático del circuito a simular. Se ajustaron los modelos de los transistores para coincidir con los
				exhibidos en el cuadro \ref{tab001}. El circuito armado es el equivalente del transistor BF966 presentado en el enunciado del trabajo práctico,
				el cual incluye muchas capacidades parásitas no consideradas en el circuito planteado en la figura \ref{circ001}. Dado que en el cálculo de 
				polarización las capacidades equivalen a circuitos abiertos, el circuito de polarización de continua es el mismo. \\

				\begin{figure}[!htb]
					\centering
						\includegraphics[width=11cm]{Imagenes/EsquematicoCascode.png}
						\caption{Esquemático del Circuito de Polarización} \label{img001}
				\end{figure}

				\indent En la tabla \ref{tab003} se exhiben los valores obtenidos por simulación y los obtenidos de forma empírica al medir el circuito indicado
				en el enunciado del TP. Al realizar la medición, como no se puede acceder al nodo $D_1$ o $S_2$, directamente se midió $V_{D_2S_1}$, que es igual 
				a $V_{DS_1} + V_{DS_2}$. \\
				
				\indent Dado que las mediciones se realizaron de forma directa sin armar ningún banco específico, se muestran directamente en el cuadro  
				\ref{tab003}.  
				 
				\begin{table}[!htb]
					\centering
					\begin{tabular}{|c|c|c|}
						\hline
						Parámetro & Valor Obtenido (Simulación) & Valor Obtenido (Medición)  \\
						\hline
						$I_{D}$ & 0.695 mA & 0.660 mA\\
						\hline
						$V_{GS_1}$ & -0.695 V & -0.418 V \\
						\hline
						$V_{D_2S_1}$ & 5.921 V & 6.22 V\\
						\hline
						$V_{D_2}$ & 6.73 V & 6.85 V\\						
						\hline						
					\end{tabular}
					\caption{Resumen de los valores de polarización medidos y simulados} \label{tab003}
				\end{table}

				Se puede observar que hay una diferencia entre los valores simulados y los valores medidos, especialmente en el valor de $V_{GS_1}$. Esto se puede 
				llegar a deber a la dispersión de los parámetros del transistor. % O a una mala modelización del Transistor Dual Gate?
				 

		\subsection{Parámetros en Señal}
			\subsubsection{Parte Teórica}
				En la presente sección se procede a realizar los cálculos de los parámetros de señales más comunes como lo son Av, Ro y Ri. Para hacer esto primero 
				se procede a realizar los cálculos de las transconductancias de cada transistor. La fórmula de la transconductancia en un MOSFET es la exhibida en
				la ecuación \ref{eq005}. 

				\begin{equation}
					g_m = \left[\frac{\partial I_D}{\partial V_{GS}}\right]_{Q} = 2k(V_{GS} - V_{TH}) \label{eq005}
				\end{equation}
	
				Teniendo en cuenta los parámetros de los transistores modelados exhibidos en la tabla \ref{tab001} se obtiene que los valores de transconductancia 
				buscados son los siguientes:

				\begin{displaymath}
					g_{m_1} = 4.575~\frac{\text{mA}}{{V}}
				\end{displaymath}
			
				\begin{displaymath}
					g_{m_2} = 16.8~\frac{\text{mA}}{{V}}
				\end{displaymath}

				Con las transconductancias calculadas se procede a realizar los cálculos del Av, Ro y Ri. En la figura \ref{circ003} se exhibe el circuito de señal
				que resulta de considerar a los capacitores de bajas frecuencias como cortocircuitos y los de altas frencuencias como circuitos abiertos, asumiendo 
				que se está trabajando en frecuencias medias.  

				\begin{figure}[!htp]
					\centering
			 		\input circuito3
	   			\caption{Equivalente circuital del \emph{Cascode} en Señal} \label{circ003}
		 		\end{figure}

				Con el circuito de señal se procede a calcular la ganancia de tensión Av. En la fórmula \ref{eq006} se exhibe el cálculo de este parámetro:
				\begin{equation}
					Av = \frac{vo_2}{vi_1} \simeq \frac{-i_d (R_{ca})}{v_{gs_1}} = - gm_1 (R_{ca}) \simeq -11.21 \label{eq006}
				\end{equation}

				Luego se procede a calcular Ro y Ri. Debido a que $R_{ig}$, la resistencia vista entre Gate y común del transistor T1, es infinita, el cálculo de
				la resistencia de entrada es el siguiente:
				
				\begin{displaymath}
					R_i = R_{ig} || R_{G} = R_{G} = 1~\text{M}\Omega
				\end{displaymath}
	
				Para calcular la resistencia de salida como bien se sabe se reemplaza a la carga por una fuente de tensión o corriente  pasivando las fuentes 
				independientes del circuito y se calcula la resistencia como la tensión de salida obtenida sobre la corriente de salida. Dado que no se 
				especifica en el enunciado cuanto vale el valor de $\lambda$, se asume que este valor tiende a cero y por lo tanto la resistencia de 
				señal $r_{ds}$ tiende a infinito. Esto trae como consecuencia que el generador controlado del transistor T2 no se prenda en ningún momento, por
				lo cual la resistencia vista entre Drain y común de este transistor tiende a a valer infinito. La fórmula de la resistencia de salida se exhibe 
				en la ecuación \ref{eq007}. 
			
				\begin{equation}	 
					R_o = R_{oc} || R_{D} = R_{D} = 4.7~\text{k}\Omega \label{eq007}
				\end{equation} 

			\subsubsection{Simulaciones}
				El circuito a simular para obtener los parámetros de señal es el mismo exhibido en la figura \ref{img001}. Para calcular los parámetros deseados
				sólo debe elegirse la correcta opción de simulación. En este caso, se hace un estudio en frecuencia mediante la opción de simulación 
				\emph{AC analysis} para verificar en qué rango se puede considerar que uno se encuentra en frecuencias medias y se calculan los parámetros en 
				cuestión para esta condición. \\
				\indent Primero se procede a calcular Av. El resultado de la simulación es mostrado en la figura \ref{img002}. Luego se procede a calcular la 
				resistencia de entrada $R_i$. Esta medición como la del Av se realiza sin modificar el circuito en la simulación. El resultado de la simulación
				es exhibido en la figura \ref{img003}. 

				\begin{figure}[!htb]
					\centering
						\includegraphics[width=12cm]{Imagenes/SimulacionAv.png}
						\caption{Simulación del parámetro de señal Av} \label{img002}
				\end{figure}
		
				\begin{figure}[!htb]
					\centering
						\includegraphics[width=11cm]{Imagenes/SimulacionRi.png}
						\caption{Simulación del parámetro de señal $R_i$} \label{img003}
				\end{figure}

				Por último se procede a calcular la resistencia de salida del \emph{Cascode}, $R_o$. Se reemplaza en el esquemático de Spice la fuente de tensión 
				senoidal ideal por un corto y la carga por un generador para computar el cálculo de este parámetro. En la figura \ref{img004} se exhibe el 
				resultado de la simulación. Se puede observar como la misma coincide con al valor teórico calculado.  

				\begin{figure}[!htb]
					\centering
						\includegraphics[width=11cm]{Imagenes/SimulacionRo.png}
						\caption{Simulación del parámetro de señal Ro} \label{img004}
				\end{figure}

			\subsubsection{Mediciones}
				Para realizar las mediciones de los parámetros de señal se procede a medir el circuito polarizado en la primera etapa del TP. Con la ayuda de
				un osciloscopio se mide la tensión de salida y la tensión de entrada, teniendo especial cuidado con la amplitud de la señal de entrada y
				la frecuencia de trabajo elegidos.\\
				\indent  En la sección siguiente (\ref{sec001} Excursiones) se calcula que la máxima tensión de entrada $v_i$ antes de que alguno de los transistores 
				recorte es de $\simeq 152~\text{mV}$. De esta forma, para cumplir con este requisito, se alimenta al circuito con una tensión de entrada $v_i$ 
				de 20 mV y una frecuencia de 50 kHz a la hora de calcular el parámetro Av, obteniéndose como resultado:

				\begin{displaymath}
					Av \simeq -10
				\end{displaymath}  	

				Luego se procede a calcular la resistencia de entrada. Dado que la medición de la resistencia de entrada involucra el cálculo de una corriente,
				se agrega una resistencia del orden de la resistencia a medir y se calcula $R_i$ en función de un divisor resistivo. Debe tenerse en 
				cuenta el efecto de carga introducido por la punta del osciloscopio en esta medición dado que $R_i$ es del orden de los M$\Omega$ y como se
				sabe la resistencia equivalente de una punta x1 es de $\simeq 1~\text{M}\Omega$ y la de una punta x10 es de $\simeq 10~\text{M}\Omega$. En
				la figura \ref{circ004} se exhibe el banco de medición equivalente y en la ecuación \ref{eq008} se exhibe la expresión de $R_i$ en función
				de los parámetros (conocidos) involucrados en el banco de medición. La resistencia auxiliar utilizada es de 1 M$\Omega$ y la relación de
				tensiones $\frac{\hat{V_{s}}}{\hat{V_{x}}}$ calculada es igual a $\simeq 1.46$ utilizando en la medición puntas x1. 
			 
				\begin{figure}[!htp]
					\centering
			 		\input circuito4
	   			\caption{Banco de Medición para medir $R_i$} \label{circ004} 
		 		\end{figure}

				\begin{equation}
					\label{eq008}
					\hat{V_{x}} = \hat{V_{s}} \left( \frac{R_{aux}}{R_{aux} + R_{i}||R_{punta}}\right) \Rightarrow R_{i}||R_{punta} = 
					\left(\frac{\hat{V_s}}{\hat{V_x}} -1\right) R_{aux} \simeq 466~\text{k}\Omega 
				\end{equation}

				\begin{displaymath}
					\Rightarrow R_i = \frac{R_{punta} \cdot 466~\text{k}\Omega}{R_{punta} - 466~\text{k}\Omega}	\simeq 872~\text{k}\Omega
				\end{displaymath} 
		
				Por último se procede a calcular la resistencia de salida. Para realizar esta medición se plantea un banco de medición parecido al utilizado
				para calcular $R_i$. El banco es similar dado que la resistencia en cuestión se calcula como una relación entre la tensión y la corriente
				en un malla determinada del circuito (en este caso, la malla de salida). Como toda resistencia de salida, para calcular la misma se debe 
				reemplazar la carga por un generador de prueba y se deben pasivar todas las fuentes independientes involucradas en el circuito. Sin embargo, 
				debido a que la resistencia de carga se encuentra soldada a la plaqueta, no se puede extraer la misma para realizar esta medición. De esta
				forma, el banco de medición armado para medir la resistencia de salida es el exhibido en la figura \ref{circ005}. La expresión de la 
				resistencia de salida en función de los parámetros involucrados es la mostrada en la ecuación \ref{eq009}. La resistencia auxiliar utilizada
				es de 4.7 k$\Omega$ y la relación de tensiones $\frac{\hat{V_{s}}}{\hat{V_{x}}}$ medidas es igual a $\simeq 1.53$.   
					 
				\begin{figure}[!htp]
					\centering
			 		\input circuito5
	   			\caption{Banco de Medición para medir $R_o$} \label{circ005} 
		 		\end{figure}

				\begin{equation}
					\label{eq009}
					\hat{V_{x}} = \hat{V_{s}} \left( \frac{R_{aux}}{R_{aux} + R_{o}||R_{L}}\right) \Rightarrow R_{o}||R_{L} = 
					\left(\frac{\hat{V_s}}{\hat{V_x}} -1\right) R_{aux} \simeq 2.5~\text{k}\Omega 
				\end{equation}

				\begin{displaymath}
					\Rightarrow R_o = \frac{R_L \cdot 2.5~\text{k}\Omega}{{R_{L}} - 2.5~\text{k}\Omega}	\simeq 5.34~\text{k}\Omega
				\end{displaymath} 

		\subsection{Excursiones} \label{sec001}
			\subsubsection{Parte Teórica}
				% Con el enconding utilizado (UTF-8) ya no es necesario escapear los acentos ni las ñ
				%Las excursiones son importantes para determinar la máxima tensión sin recorte que se puede obtener a la salida del circuito. Además de las excursiones,
				%debe verificarse que la señal de salida no posea un porcentaje de distorsión elevado, lo cual puede darse aunque los transistores nunca lleguen a	sus
				%máximas excursiones. \\
				\indent Para comenzar se calculan las rectas de carga dinámica de cada transistor, teniendo en cuenta el esquemático del
				\emph{Cascode} en señal presentado en la figura \ref{circ003}. Debido a que las corrientes de gate son pequeñas y despreciables, las corrientes de drain
				de los dos transistores son iguales. Con esto sabido, se recorre la malla que involucra a las tensiones $v_{ds_1}$ y $v_{gs_2}$, obteniéndose la 
				ecuación mostrada en \ref{eeq001}.
			
				\begin{equation}
					v_{ds_1} + i_d \cdot 50~\Omega - v_{gs_2} = 0 \label{eeq001}
				\end{equation}

				Dado que la corriente de drain de un transistor se puede escribir en función de su transconductancia, se reemplaza esta expresión en la
				ecuación \ref{eeq001} y despejando se llega a la expresión de la recta de carga dinámica del transistor T1. La misma se exhibe en la ecuación 
				\ref{eeq002}.

				\begin{equation}
					i_d = \frac{v_{ds_1}}{r{d_2} + 50 \Omega} = \frac{v_{ds_1}}{110 \Omega} \label{eeq002}
				\end{equation}

				Para realizar el cálculo de la excursión del segundo transistor, se recorre la malla que involucra a las ddp $v_{ds_1}$ y $v_{ds_2}$ obteniéndose como
				resultado la ecuación exhibida en \ref{eeq003}. Dado que $R_{ca}$ en este caso es del orden de los K$\Omega$, se puede despreciar el efecto de carga 
				impuesto por la resistencia de 50 $\Omega$ en esta última ecuación. Luego, reemplazando el resultado obtenido en \ref{eeq002} en \ref{eeq003} se 
				obtiene la expresión de la recta de carga del transistor T2. La misma se exhibe en la ecuación \ref{eeq004}.

				\begin{equation}
					i_d \cdot (R_{ca} + 50 \Omega) + v_{ds_1} + v_{ds_2} = 0  \label{eeq003}
				\end{equation}

				\begin{equation}
					i_d = \frac{- v_{ds_2}}{R_{ca} \cdot(1 - \frac{r{d_2}}{R_{ca}})} \quad \Rightarrow \quad i_d \simeq \frac{- v_{ds_2}}{R_{ca}}	\label{eeq004}
				\end{equation}	

				Con las pendientes de las RCDs, se procede a calcular las máximas excursiones de cada transistor. Para realizar esto hay que recordar que estas
				últimas rectas fueron calculadas en señal, de modo que cada RCD pasa por el punto de reposo en el cual se encuentra su respectivo transistor. 
				Teniendo esto en cuenta y despreciando las tensiones y corrientes de codo se exhiben a continuación las máximas excursiónes por corte y saturación 
				de cada transistor:

				\begin{equation*}
					\hat{V_{DS_1}}(corte) = I_{D_{1Q}} \cdot 110 \Omega \simeq 77~\text{mV}
				\end{equation*}

				\begin{equation*}
					\hat{V_{DS_1}}(sat.) = V_{DS_1} = 0.881~\text{V}
				\end{equation*}

				\begin{equation*}
					\hat{V_{DS_2}}(corte) = I_{D_{1Q}} \cdot 2.35 \Omega \simeq 1.645~\text{V}
				\end{equation*}
					
				\begin{equation*}
					\hat{V_{DS_1}}(corte) = V_{DS_2} = 5.01~\text{V}
				\end{equation*}

				Como se puede observar, ambos transistores llegan a sus límites debido a sus máximas excursiones por corte. Sin embargo, esto no nos dice cual es
				el transistor que está limitando la máxima señal de entrada que se puede ingresar. Para determinar cual de los transistores es el limitante en este 
				circuito se procede a obtener las tensiones de entradas $v_i$ necesarias para que las salidas de cada uno no presenten recortes. Para ello se utilizan 	
				las amplificaciones de cada etapa las cuales se exhiben a continuación:

				\begin{displaymath}
					A_{v_1}= -0.27
				\end{displaymath}
				\begin{displaymath}
					A_{v_2} =~39.48
				\end{displaymath}

				Obteniendo las $v_i$ de recorte:


				\begin{displaymath}
					v_{i_1} =\frac{v_{ds_1}}{-0.27}=285.2~\text{mV}
				\end{displaymath}

				\begin{displaymath}
					v_{i_2} =\frac{v_{ds_2}}{39.48}=41.3~\text{mV}
				\end{displaymath}

				Aquí se muestra que el recorte está limitado por el primer transistor y por lo tanto la señal a la salida del circuito presenta recorte con una 
				entrada de:
				\begin{displaymath}
					v_i = 285.2~mv
				\end{displaymath}

				Luego de realizar el análisis de las excursiones, se procede a analizar el criterio que debe tomarse para que haya baja distorsión en la señal de
				salida respecto de la de entrada. Del enunciado del presente trabajo se extrae que la condición que debe cumplirse es que:

				\begin{displaymath}
					V_{GS}-V_{GSQ} << \frac{V_{GSQ}-V_{TH}}{2}
				\end{displaymath}

				Esta condición se obtiene del desarrollo por taylor de la ecuación de transferencia del transistor MOS en el punto de trabajo Q. Realizando un 
				desarrollo de Taylor hasta la derivada primera se obtiene:
				\begin{displaymath}
					I_d(V_{GS})=I_d + 2k(V_{GS_Q}-V_{TH})(V_{GS}-V_{GS_Q})+...
				\end{displaymath}

				Ahora si se multiplica y divide el segundo termino de la sumatoria por $V_{GS} - V_{TH}$ se obtiene:

				\begin{displaymath}
					I_d(V_{GS})=I_d + \frac{2k(V_{GS_Q}-V_{TH})(V_{GS}-V_{TH})}{(V_{GS_Q}-V_{TH})}+...
				\end{displaymath}

				Reemplazando la ecuaci\'{o}n caracter\'{i}stica del MOS se llega a que:

				\begin{displaymath}
					I_d(V_{GS})=I_d + \frac{2I_d(V_{GS}-V_{GS_Q})}{(V_{GS}-V_{TH})}
				\end{displaymath}

				Esta ecuación muestra que para que la linealización sea respetada, se necesita que los términos no lineales tengan baja influencia por lo tanto 
				se necesita que:

				\begin{displaymath}
					V_{GS}-V_{GS_Q}<< \frac{(V_{GS_Q}-V_{TH})}{2}
				\end{displaymath}

				Se calcularon los valores
				\begin{displaymath}
					\frac{(V_{GS_{Q1}}-V_{TH})}{2}=152~\text{mV}
				\end{displaymath}

				\begin{displaymath}
					\frac{(V_{GS_{Q2}}-V_{TH})}{2}=42~\text{mV}
				\end{displaymath}
				Para concluir se obtiene que antes de que la tension de salida sufra un recorte por corte, esta se vera previamente distorsionada.						
			\subsubsection{Simulaciones}		
			Para realizar la simulación, se procede a calcular algunos armónicos del espectro de furier de la señal que obtenemos a la salida del circuito. Para esto se emplea el comando .four del LTSPICE el cual calcula estos armónicos y nos brinda el THD (Distorsión Armónica Total) con el cual tenemos una relación porcentual de la influencia de los armónicos sobre la frecuencia fundamental.
			El valor que se tomara para considerar a la señal de salida "optima" (baja distorsión) sera un THD en un intervalo de cero a tres por ciento. 
			\indent La senal utilizada es senoidal con frecuencia de 50 KHz. En las figuras \ref{pimg001}, \ref{pimg002} y \ref{pimg003} se muestra el aumentos del THD causada por la aparicion de armonicos. \\
			\indent Dichas amplitudes son (expresadas en mV) 10, 25 y 160 respectivamente. Como conclusión, se observa que la señal de entrada que no presenta 
			distorsión apreciable es:
				\begin{displaymath}
					\hat{V_i} =25~\text{mV}
				\end{displaymath}
					
			\begin{figure}[!htb]
					\centering
						\includegraphics[width=9.5cm]{Imagenes/FFT_10.png}
						\caption{Se\~{n}al sin distorcion} \label{pimg001}
				\end{figure}
			\begin{figure}[!htb]
					\centering
						\includegraphics[width=9.5cm]{Imagenes/FFT_25.png}
						\caption{Se\~{n}al presentando sintomas de distorcion} \label{pimg002}
				\end{figure}
			\begin{figure}[!htb]
					\centering
						\includegraphics[width=8cm]{Imagenes/FFT_160.png}
						\caption{Se\~{n}al con distorcion apreciable} \label{pimg003}
				\end{figure}

			%TODO: Faltaría realizar el análisis para ver a qué tensiones de entrada recorta la salida. 
				 
			\subsubsection{Mediciones}
			Para el análisis de la distorsión, se procede a observar en el laboratorio la señal a la salida a traves del osciloscopio. Se observa que a partir de 
			una $V_i$ de 25 mV de amplitud, la señal comienza a distorsionar. Este valor implica que a la salida se tiene 266.5 mV de tension pico, por lo que 
			el transistor comienza a distorsionar bastante antes de llegar a la máxima excursión por corte calculada por inspección. \\
 			\indent Tensi\'{o}n m\'{a}xima de salida medida:	

			\begin{displaymath}
				\hat{V_i} = 25~\text{mV}
			\end{displaymath}			

			\paragraph{¿Qué efectos causaría la eliminacion $C_g$ en los valores medidos en los punto 1 a 4?}
			La eliminación de $C_g$ a frecuencias medias causaría la aparición en alterna de Rs, provocando las siguientes modificaciones en los parámetros del circuito:
			\begin{itemize}
				\item El punto de reposo no evidencia variaciones ya que el capacitor en corriente continua actúa como un circuito abierto.
				\item La ganancia total del circuito caería ya que el primer transistor estaría realimentado para alterna. La nueva ganancia total sería la de un 
				emisor común realimentado, la cual se exhibe al ecuación \ref{eq010}.
				\item Aumenta la resistencia de entrada a de gate $R_ig$, pero como es muy grande en comparación con $R_G$, al estar relacionadas mediante un paralelo la
				resistencia $R_G$ dominaría y la resistencia interna $R_i$ se mantendría sin modificación apreciable.
				\item Al incorporarse $R_S$ en el circuito de señal la tensión de entrada quedaría compuesta por $v_i = v_{gs_1} + v_{R_S}$, lo cual permite que la 
				máxima señal de entrada $v_i$ que puede introducirse sin que haya recorte aumente.
			\end{itemize}
				
			\begin{equation}
				A_v =\frac{-R_D || R_L}{r_{d1}+1}=-1.93 \label{eq010}
			\end{equation}

		\subsection{Estudio Respuesta En Frecuencia}
			\subsubsection{Parte Teórica}
				A continuación se estudiará la respuesta en frecuencia del circuito. Para ello, cabe destacar 3 zonas marcadas, las cuales son bajas frecuencias, 
				las llamadas frecuencias medias y altas frecuencias. \\
				\indent Frecuencias medias son aquellas frecuencias en que el circuito se comporta como si fuese puramente resistivo, ya que a esas frecuencias las 
				reactancias capacitivas no son comparables a los valores de resistencias. \\
				\indent En bajas frecuencias, los capacitores que están conectados en serie a la señal comienzan a influir haciendo que el $A_v$ disminuya. Estos 
				capacitores comienzan a influir.
				\indent Para determinar cual es la frecuencia de corte inferior del circuito, es necesario saber que capacitores son los que afectan a la ganancia, 
				en este caso el $A_v$. Estos capacitores son los que están conectados en serie al recorrido de la señal, en el circuito \ref{pimg004} son 
				$C_G$ y $C_L$. Los mismos comienzan a influir porque el valor de sus reactancias comienzan a hacerse comparables con los valores de las 
				resistencias del circuito. \\
				
				\begin{figure}[!htb]
					\centering
						\includegraphics[width=12cm]{Imagenes/CascodeRespuestaBajaFrecuencia.png}
						\caption{Circuito equivalente en baja frecuencia} \label{pimg004}
				\end{figure}
				
				\indent No se realizará un análisis sobre la respuesta con el capacitor $C_L$ ya que, al considerarse $r_o$ y $r_\mu$ infinitas, dicho componente 
				no posee interacción ni con la entrada ni con el resto de los capacitores. \\
				\indent Para calcular la frecuencia de corte inferior se supondrá a priori, que los polos impuestos por los capacitores se encuentran suficientemente 
				separados entre sí para que no haya influencia entre los mismos. A su vez, se supondrá que la frecuencia de corte es causada por un polo de primer 
				orden. Para esto, se calculará la frecuencia de corte de cada capacitor por serparado, por ende, se considerará que los polos impuestos por el 
				resto de los capacitores están suficientemente a la izquierda (en una frecuencia menor), de esto se interpreta que se debe considerar a estos 
				capacitores como cortocircuitos. \\\\
				
				\indent En señal, las resistencias "vistas" por el capacitor se aprecian en el grafico \ref{pimg005}
				
				\begin{figure}[!htb]
					\centering
						\includegraphics[width=10cm]{Imagenes/RVistaCg.png}
						\caption{Circuito equivalente "visto" por $C_g$} \label{pimg005}
				\end{figure}
				
				Para conocer el valor de $r_d$ se realiza la siguiente ecuación:
				
				\begin{equation}
					rd = \frac{1}{gm_1} = \frac{1}{4.575~\frac{\text{mA}}{\text{V}}} = 219~\Omega
				\end{equation}
				
				Quedando así, la resistencia total vista por el capacitor:
				
				\begin{equation}
					R_t = rd // R1 = 219~\Omega // 1~\text{K}\Omega = 180~\Omega
				\end{equation}
				
				Como $\tau = R_t \cdot C$, la frecuencia de corte inferior queda de la siguiente manera:

				\begin{equation}
					F_c = \frac{1}{2\pi \cdot \tau} = \frac{1}{2 \pi \cdot 180~\mu \text{s}} = 884~\text{Hz} \label{eq011}
				\end{equation}

				\indent En altas frecuencias, los capacitores anteriormente mencionados continúan comportándose como un cortocircuito como lo hacían a frecuencias 
				medias, pero ahora, los capacitores del modelo del transistor comienzan a influir afectando así, nuevamente, a la ganancia $A_v$ porque están 
				conectados en paralelo al recorrido de la señal. Por ende, mientras más se aumenta la frecuencia, menor es la reactancia, hasta que se comportan 
				como un cortocircuito. En este punto extremo, la $V_o = 0$. \\
				\indent El circuito equivalente a altas frecuencias es mostrado en la figura \ref{pimg006} \\
				
				\begin{figure}[!htb]
					\centering
						\includegraphics[width=10cm]{Imagenes/CircuitoAltaFrec.png}
						\caption{Circuito equivalente a alta frecuencia} \label{pimg006}
				\end{figure}
				
				\indent Para calcular la frecuencia de corte superior se supondrá a priori que los polos impuestos por los capacitores se encuentran suficientemente 
				separados entre sí para que no haya influencia entre los mismos. A su vez, se supondrá que la frecuencia de corte es causada por un polo de 
				primer orden. Para esto, se calculará la frecuencia de corte de cada nodo por separado. Por ende, se considerará que los polos impuestos por el 
				resto de los capacitores están suficientemente a la derecha (en una frecuencia mayor), de esto se interpreta que se debe considerar a estos 
				capacitores como abiertos. \\ \\
				\indent Como puede observarse en el gráfico \ref{pimg006}, sólo se debe calcular la frecuencia de corte en los nodos de la entrada ($G_1$) y 
				salida ($D_2$), ya que el resto de los capacitores del modelo mostrados en la figura \ref{circ001} están conectados en ambos terminales a masa. \\
				\indent Primero se procederá a calcular el tau referente al nodo del $G_1$, para ello se debe calcular que resistencia equivalente "ve" 
				el capacitor equivalente (como hay un capacitor que une la entrada con la salida, es necesario reflejarlo utilizando los criterios de müller). 
				En el gráfico \ref{pimg007} se muestra el circuito equivalente. \\

				\begin{figure}[!htb]
					\centering
						\includegraphics[width=10cm]{Imagenes/CircEqNodoG1.png}
						\caption{Circuito equivalente a alta frecuencia para el nodo $G_1$} \label{pimg007}
				\end{figure}
				
				\indent Como el capacitor $C_5$ está conectado entre la entrada y la salida, su capacitancia reflejada se multipmica por un factor 
				de $(1 - A_v)$, el $A_v$ es el calculado en la sección anterior y es igual a $-11,21$, resultando en los siguientes valores equivalentes:
				
				\begin{equation}
					R_{eq} = R_5 // R_2 = R_5 = 50~\Omega
				\end{equation}
				
				\begin{equation}
					C_{eq} = C_4 + C_5^{*} \cdot (1 - A_v) = 2.2~\text{pF} + 25~\text{fF} \cdot (1 + 11.21) = 2.5~\text{pF}
				\end{equation}
				
				La fórmula de la frecuencia de corte es la misma que para bajas frecuencias, por ende se procederá a calcularla como en la ecuación 
				\ref{eq008}. El resultado a continuación:
				
				\begin{equation}
					\tau = R_{eq} \cdot C_{eq} = 125~\text{ps}
				\end{equation}

				\begin{equation}
					F_{c_{sup_1}} = \frac{1}{2\pi \cdot \tau} = \frac{1}{2\pi \cdot 125~\text{ps}} = 1.2~\text{GHz}
				\end{equation}
				
				\indent Ahora se procederá a calcular el tau con respecto al nodo de salida o nodo de $D_2$ realizando el mismo procedimiento. Esta vez, 
				el $A_v$ utilizado para la reflexión de $C_5$ es de una ganancia inversa, ya que la señal entra por el nodo $D_2$ y sale por $G_1$. Por ende este 
				valor tiende a 0 y se lo desprecia. En el gráfico \ref{pimg008} se muestra el circuito de resistencias y capacitores equivalentes. Como se 
				supone un $V_a$ tendiendo a infinito, la resistencia vista hacia el nodo $D_2$ tiende a infinito y no se la considera.

				\begin{figure}[!htb]
					\centering
						\includegraphics[width=10cm]{Imagenes/CircEqNodoD2.png}
						\caption{Circuito equivalente a alta frecuencia para el nodo $D_2$} \label{pimg008}
				\end{figure}

				\begin{equation}
					R_{eq} = R_{ca} = 2450~\Omega \label{eq012}
				\end{equation}
				
				\begin{equation}
					C_{eq} = C_6 + C_5^{*} \cdot (1 - A_v) = 0.8~\text{pF} + 25~\text{fF} = 0.825~\text{pF} \label{eq013}
				\end{equation}
				
				\indent Con los valores exhibidos en las ecuaciones \ref{eq011} y \ref{eq012} se procederá a calcular la segunda posible frecuencia 
				de corte superior:
				
				\begin{equation}
					\tau = R_{eq} \cdot C_{eq} = 2.02~\text{ns}
				\end{equation}

				\begin{equation}
					F_{c_{sup_2}} = \frac{1}{2 \cdot \pi \cdot \tau} = \frac{1}{2\pi \cdot 2.02~\text{ns}} = 79~\text{MHz}
				\end{equation}				
				
				\indent Comparando ambas frecuencias de corte superiores, $1.2~\text{GHz}$ y $79~\text{MHz}$, se puede concluir que la frecuencia de corte superior ronda 
				por los $79~\text{MHz}$. Por lo tanto el ancho de banda del circuito es de $885~\text{Hz} - 79~\text{MHz}$ \\ \\
				\indent A la hora de realizar las mediciones, se debe tener en cuenta el efecto de carga que el circuito punta-osciloscopio genera en el circuito. 
				Las mediciones se realizaron con dos puntas distintas, una punta $x10$ y otra activa, la cual es posee una factor de atenuación igual a 20. Dichas 
				puntas se cuelgan en paralelo a los circuitos calculados anteriormente para dichas mediciones. Para la medición de la frecuencia de corte inferior 
				el capacitor equivalente no afecta, ya que para esta frecuencia se lo considera como un circuito abierto. \\
				\indent Las resistencias y capacitores equivalentes de dichos componentes se muestran en el cuadro \ref{tab004}.
				
				\begin{table}[!htb]
				\centering
					\begin{tabular}{|c|c|c|}
						\hline
						Punta & $R_{int}$ & $C_{int}$ \\
						\hline
						$X_{10}$ & $10~\text{M}\Omega$ & $20~\text{pF}$\\
						\hline
						$X_{20}$ & $20~\text{M}\Omega$ & $2~\text{pF}$\\
						\hline
					\end{tabular}
				\caption{Resistencias y capacitancias equivalentes en punta $X_{10}$ y $X_{20}$} \label{tab004}
				\end{table}
				
				\indent En la tabla \ref{tab005} se muestran los capacitores y resistencias equivalentes y los valores de las frecuencias de corte superior, 
				tomando en cuenta el efecto de carga que la punta-osciloscopio genera. La fórmula para calcular la frecuencia de corte es la fórmula anteriormente 
				mencionada. \\
				
				\begin{table}[!htb]
				\centering
					\begin{tabular}{|c|c|c|}
						\hline
						Parámetros & $X_{10}$ & $X_{20}$ \\
						\hline
						$R_{eq}$ & $2450~\Omega$ & $2450~\Omega$ \\
						\hline
						$C_{eq}$ & $20~\text{pF}$ & $2.825~\text{pF}$ \\
						\hline
						$Fc_{sup}$ & $3~\text{MHz}$ & $23~\text{MHz}$\\
						\hline
					\end{tabular}
				\caption{Frecuencias de corte superior tomando en cuenta el efecto de carga de las distintas puntas} \label{tab005}
				\end{table}
				
				\indent Como puede apreciarse en el cuadro \ref{tab005} la frecuencia de corte superior es menor utilizando la punta $X_{10}$ ya que esta introduce un
				mayor efecto de carga en la medición. \\
			%\clearpage	

			\subsubsection{Simulaciones}
			% TODO: Esto está demasiado pobre. Hay que completar y hacer un análisis de la respuesta en frecuencia en función del efecto de carga introducido por
			% las puntas. 
			\indent A continuación se procederá a realizar la simulación de la respuesta en frecuencia para corroborar lo calculado analíticamente y 
			las mediciones efectuadas. 
			
			\begin{figure}[!htb]
					\centering
						\includegraphics[width=10cm]{Imagenes/tranferencia.png}
			\end{figure}
			
			Aqui se puede observar que la frecuencia de corte superior calculada analíticamente no difiere en mucho, su error corresponde al m\'{e}todo utilizado.\\
			Finalmente se observa que la frecuencia de corte superior del circuito mediante simulaci\'{o}n es $\simeq 78~\text{MHz}$.\\
			\chapter{Simulacion con Puntas}
				\\Ahora analizaremos los efectos de las puntas en las mediciones\\
				Las puntas no afectan en bajas frecuencias ya que presentan capacidades equivalentes muy pequeñas\\\\
				Punta X 10\\
				La punta x10 presenta
				\begin{displaymath}
					C_{eq} = 20 pf
				\end{displaymath}
				\begin{displaymath}
					R_{eq} = 10 Meg
				\end{displaymath}
				Ubicadas en paralelo a la salida y entre ellas generando una capacidad equivalente total en el nodo de salida de
				\begin{displaymath}
					C_{eqtot}= 0.8pf+20pf = 20.8 pf
				\end{displaymath}
				Obteniendo de esta manera
				\begin{displaymath}
					f_h = 3.24 MHz
				\end{displaymath}
				\begin{figure}[!htb]
					\centering
						\includegraphics[width=12cm]{Imagenes/Puntax10.png}
				\end{figure}
				Punta Activa\\
				La punta Activa presenta
				\begin{displaymath}
					C_{eq} = 2 pf
				\end{displaymath}
				\begin{displaymath}
					R_{eq} = 1 Meg
				\end{displaymath}
				Ubicadas en paralelo a la salida y entre ellas generando una capacidad equivalente total en el nodo de salida de 
				\begin{displaymath}
					C_{eqtot}= 0.8pf+2pf = 2.8 pf
				\end{displaymath}
				Obteniendo de esta manera
				\begin{displaymath}
					f_h = 23.83 MHz
				\end{displaymath}
				\begin{figure}[!htb]
					\centering
						\includegraphics[width=12cm]{Imagenes/Puntax20.png}
				\end{figure}
				Finalmente cabe aclarar que las resistencais equivalentes de las puntas no afectan al valor de la equivalente ya que son grandes en comparacion 
				la resistecia de salida del circuito\\
				Se concluye que el polo generado por la inclucion de las puntas afecta a la frecuencia de corte superior de la transferencia. Por consiguiente y 
				como muestran las simulaciones la punta activa al tener un tau mas pequeño que la punta X10, afecta en menor medida a la medicion de la fh.
			
			\subsubsection{Mediciones}
				\indent Para la medicion de la respuesta en frecuencia se utilizaron dos puntas, una activa $X_{20}$ y otra pasiva $X_{10}$. La punta pasiva posee una 
				capacitancia de $20~\text{pF}$ y una resistencia en paralelo de $10~\text{M}\Omega$, mientras que la punta activa posee una capacitancia $2~\text{pF}$ 
				y una resistencia de $20~\text{M}\Omega$ en paralelo. Como indica el item a realizar se midió la frecuencia de corte superior. \\
				
				\indent Para realizar la medición se conectaron 2 puntas, una a la entrada del circuito y otra a la salida, se fue variando la frecuencia hasta que la 
				amplitud de la señal de salida sea menor que la amplitud de entrada por un factor de $\sqrt{2}$. \\
				\indent La punta conectada a la entrada del circuito no influye en la medición dado que el nodo GATE-T1 no es el dominante por lo demostrado en la 
				parte teórica. 

				\begin{displaymath}
					punta X10 \Rightarrow Ceq=0.8~\text{pF}+20~\text{pF}=20.8~\text{pF}
				\end{displaymath}	

		    \begin{displaymath}
					punta Activa \Rightarrow Ceq=0.8~\text{pF}+2~\text{pF}=2.8~\text{pF}	
				\end{displaymath}

				En la tabla \ref{tab006} se muestran los valores de tensión de la salida medidos a diferentes frecuencias. La $v_i$ utilizada es de $30~\text{mV}$ 
				y, como el $A_v$ es de 10, la salida a frecuencias medias es de $300~\text{mV}$.
			
				\begin{table}[!htb]
				\centering
					\begin{tabular}{|c|c|c|}
						\hline
						Frecuencia & $V_o$ punta $X_{10}$ (mV) & $V_o$ punta $X_{20}$ (mV) \\
						\hline
						$100~\text{KHz}$ & 295 & 290 \\
						\hline
						$500~\text{KHz}$ & 280 & 285 \\
						\hline
						$1~\text{MHz}$ & 254 & 285 \\
						\hline
						$3~\text{MHz}$ & 215 & 286 \\
						\hline
						$8~\text{MHz}$ & 180 & 280 \\
						\hline
						$10~\text{MHz}$ & 82 & 272 \\
						\hline
					\end{tabular}
				\caption{Mediciones de $V_o$ a distintas frecuencias} \label{tab006}
				\end{table}			
			
				Como se puede apreciar en la tabla \ref{tab006}, se obtuvo una frecuencia de corte de $3.1~\text{MHz}$ con la punta $X_10$. En cuanto a la medición 
				con la punta activa, no se pudo medir la frecuencia de corte superior ya que el generador de funciones no entrega una frecuencia mayor que 
				$10~\text{MHz}$. Por ende, lo único que se puede asegurar es que la frecuencia de corte superior del circuito se encuentra por encima de los 
				$10~\text{MHz}$. 
		
\end{document}
