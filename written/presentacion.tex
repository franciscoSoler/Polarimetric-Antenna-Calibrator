\documentclass[a4paper,10pt]{article}
%\usepackage[latin1]{inputenc} % Paquetes de idioma
\usepackage[utf8]{inputenc} % Paquetes de idioma
\usepackage[spanish]{babel} % Paquetes de idioma
\usepackage{graphicx} % Paquete para ingresar gráficos
\usepackage{grffile}
\usepackage{hyperref}
\usepackage{fancybox}
\usepackage{amsmath}
\usepackage{amsfonts}
\usepackage{listings}
\usepackage{enumerate}
\usepackage{csquotes}
\usepackage{longtable}
% Paquetes de macros de Circuitos
%\usepackage{pstricks}
\usepackage{tikz}
\usepackage[colorinlistoftodos,prependcaption,textsize=tiny]{todonotes}

% Encabezado y Pié de página
\input{EncabezadoyPie.tex}
% Carátula del Trabajo
\title{ \author{} % Lo pongo para que el warning no moleste :p
\setlength{\unitlength}{1cm} %  Especifica la unidad de trabajo
\thispagestyle{empty}

\begin{picture}(18,0)
\put(0,0){\includegraphics[width=1.5cm, height=3cm]{Logo1.png}}

\put(10.5,0){\includegraphics[width=3cm, height=3cm]{Logo2.png}}

\end{picture}
\\[1.5cm]
\begin{center}
	\textbf{{\Huge Facultad de Ingenier\'ia \\ Universidad de Buenos Aires}}\\[2cm]
	{PROYECTO DE TESIS DE INGENIERÍA INFORMÁTICA}\\[0.5cm]
	{Calibración de una antena polarimétrica utilizando los acoplamientos 
	mutuos}\\[2.5cm]
\end{center}

\begin{flushleft}
	\textbf{ESTUDIANTE:}  Soler, Jos\'e Francisco \\[0.5cm]
	\textbf{PADR\'ON:} 91227 \\[0.5cm]
	\textbf{COORDINADORES:} Ing. Marino, Pablo - Ing. Wachenchauzer, Rosita\\[0.5cm]
\end{flushleft}
\date{} % Hace que no se imprima la fecha en la cual se compilo el .tex
 }

\begin{document}
	\maketitle % Hace que el título anterior sea el principal del documento
	
	\tableofcontents % Esta línea genera un indice a partir de las secciones y 
					 % subsecciones creadas en el documento
	\newpage

	\section{INTRODUCCIÓN}
		Las antenas de arreglo de fase controlada son comunmente utilizadas en 
	aplicaciones aéreas y espaciales. Para obtener un buen comportamiento de las
	mismas es necesario que estén correctamente calibradas. Esto implica, que 
	las tolerancias de fases y amplitudes se mantengan y sus valores sean bien 
	conocidas por cada elemento del arreglo.
	
		Las antenas de arreglo de fase en tierra son, generalmente, calibradas
	utilizando fuentes externas de campo lejano o cercano. Sin embargo, en 
	aplicaciones aéreas o espaciales, la utilización de dichas fuentes es 
	impráctica o difícil de implementar. A su vez, si se opta por caracterizar
	todos los componentes, el tiempo que implicaría es excesivo. Por estas 
	razones, surgieron distintos métodos de calibración interna.

		En este conexto, se propone un nuevo método de calibración, el cual 
	aprovecha el acoplamiento mutuo inherente entre los módulos radiantes de la 
	antena.

	\subsection{DEFINICIÓN}
		Una antena de arreglo de fase controlada es una antena compuesta por un
	conjunto de módulos radiantes dispuestos de tal forma, que, aplicando la
	teoría de construcción y desturcción de ondas, la señal emitida logra ser 
	dirigida donde se desee.
		
		Calibración interna es colocar sensores que permitan la medición directa
	u indirecta de la potencia y fase de salida/entrada de la antena 
	polarimétrica.
		
	\subsection{CARACTERÍSTICAS}
		La utilización de una buena calibracíon interna es una problemática muy
	desafiante dado que es uno de los factores limitantes en la calidad de los 
	productos obtenidos con estas antenas.

	\section{MOTIVACIÓN}
		Hay numerosas motivaciones para la investigación de un nuevo método de 
	calibración:
		Primero, a la hora de adquirir imágenes satelitales es crucial que se 
	conozca perfectamente la señal emitida y recibida por la antena. Ya sea por 
	envejecimiento de los materiales, por variaciones de temperaturas u algún 
	otro factor, se observan dispersiones de las mismas. Hay dos enfoques para 
	encarar esta problemática:

		\begin{itemize}
			\item Controlando las dispersiones máximas que pueden presentarse 
			utilizando hardware más complejo.

			\item Corrigiendo dichas dispersiones haciendo uso de calibración 
			interna.
		\end{itemize}

		Al utilizar la calibración interna se evita aumentar la complejidad y 
	peso del hardware utilizado a costa de un mayor procesamiento de software, 
	logrando así, disminuir el costo de la misión.
		
		Otro motivo es que el método de calibración convencional posee numerosas
	limitaciones y falencias; la principal es que no abarca todo el sistema de 
	transmisión/recepción, dejando así parámetros fuera de control. 


		
	\section{Objetivo de la tesis}
		La presente tesis tiene como objetivo la investigación, análisis y 
	desarrollo de un nuevo método de calibración interna de una antena 
	polarimétrica que abarque el sistema completo de transmisión/recepción. 

	\section{Metodología de la tesis}
		En la presente tesis se investigarán los métodos de calibraciones 
	actuales para poder determinar que ventajas, desventajas, limitaciones y 
	diferencias hay entre cada una de ellos. Se buscará tener una visión global
	de esta problemática para poder determinar y entender que posibles falencias
	puede tener este nuevo método.

		Posteriormente, se investigarán las limitaciones que poseen las antenas 
	polarimétricas para poder determinar que recaudos se deben tener en cuenta a
	la hora de desarrollar el método.

		Luego, tomando todo en cuenta, se determinarán las hipótesis necesarias
	para que el algoritmo funcione correctamente. Para la validación del método
	se realizará un modelo de antena.
		
		Finalmente, se probarán, analizarán y documentarán los resultados 
	obtenidos de la comparación entre el algoritmo propuesto y el algoritmo de
	la calibración convencional. A su vez, se dejará asentado que posibles 
	mejoras se podrían aplicar al algoritmo para determinar otros aspectos que
	están fuera del alcance de esta tesis.

	\section{Estado del Arte}
		La calibración de una antena polarimétrica se ha estudiado en numerosas 
	ocasiones, abordando el problema desde distintos enfoques. A continuación se
    pueden observar los distintos métodos utilizados, clasificados por la 
    utilización o no de componentes externos.

	\[
		\substack{\text{métodos}\\de\\\text{calibración}}
		\begin{cases}
			\substack{\text{utilizan}\\\text{componentes externos}}
			\begin{cases}
				
				\text{blancos puntuales - [\ref{ppr:puncTrgt1}], 
				[\ref{ppr:puncTrgt2}], [\ref{ppr:punc-ext1}], 
				[\ref{ppr:puncTrgt3}]}\\
				\text{blancos distribuidos - [\ref{ppr:dist1}]}\\
				\text{absolute Radiometric Calibration - [\ref{ppr:absRad1}],
				[\ref{ppr:rad2}], [\ref{ppr:rad3}], [\ref{ppr:abs-rad-ical1}], 
				[\ref{ppr:rad4}], [\ref{ppr:rad5}], [\ref{ppr:rad6}], 
				[\ref{ppr:rad7}]}\\
				\text{Broadcast Reference Technique - [\ref{ppr:brdcast1}]}\\
				\text{miden deformación antena - [\ref{ppr:aligment1}], 
				[\ref{ppr:aligment2}], [\ref{ppr:aligment3}], 
				[\ref{ppr:aligment4}], [\ref{ppr:aligment5}],
				[\ref{ppr:aligment6}]}\\
				\text{utilizan calibración externa - [\ref{ppr:ext1}], 
				[\ref{ppr:ext2}], [\ref{ppr:ext3}], [\ref{ppr:punc-ext1}], 
				[\ref{ppr:classic-ext1}], [\ref{ppr:mutual-ext1}]}\\
			\end{cases}\\
			\substack{\text{no utilizan}\\\text{componentes externos}}
			\begin{cases}
				\text{calibración clásica - [\ref{ppr:classic1}], 
				[\ref{ppr:classic2}], [\ref{ppr:classic3}], 
				[\ref{ppr:abs-rad-ical1}], [\ref{ppr:classic4}], 
				[\ref{ppr:classic5}], [\ref{ppr:classic6}],
				[\ref{ppr:classic-ext1}], [\ref{ppr:classic7}],
				[\ref{ppr:classic8}]}\\
				\text{aprovechando acoplamiento mutuo - [\ref{ppr:mutual1}], 
				[\ref{ppr:mutual-ext1}]}\\
				\text{método REV - [\ref{ppr:rev1}]}
			\end{cases}
		\end{cases}
	\]


La calibración interna es el método más típicamente utilizado pero adolece de 
algunos defectos como ser que:

\begin{itemize}
    \item Los circuladores no están incluidos dentro de dicho lazo de 
calibración interna. Esto obliga a que los mismos tengan que ser caracterizados
previamente y asegurar la estabilidad a lo largo de toda la vida útil de la
misión.
    \item Las antenas, o modulos radiantes, también quedan fuera del lazo de
calibración. Por ende, un módulo radiante que pudiese resultar dañado en vuelo,
no es detectable por el lazo interno de calibración.
    \item Durante los ensayos, se adolece de puntos de testeo que permitan 
corroborar que la potencia de salida por el modulo radiante es la esperada. No
hay acopladores direccionales entre el TRM y el modulo radiante entonces no es
posible si no es con un equipo de campo cercano poder ver la potencia de salida
de cada módulo radiante y la fase.
    \item Es necesario realizar una caracterización de la RFDN en temperatura lo
cual insume numerosos recursos y queda sujeto además a la suposición de que nada
cambien en el tiempo. \todo{pongo que esto es debido a que, para calibrar una 
cadena, se deben utilizan ambas (de ambas polarizaciones), una para transmisión
y otra para recepción. Por lo tanto, se debe eliminar el defasaje y atenuación 
que agrega el camino de recepción?}
    \item Existe una incertidumbre dada reductible hasta cierto punto en la 
determinación de fase y amplitud de salida dada por lo que determina la cadena y
el camino en cuestión.
\end{itemize}

Por este motivo la calibración interna requiere de un método que complemente 
todos estos aspectos, mejorandolos. Es decir un método que:

\begin{itemize}
    \item Incorpore los circuladores al lazo de calibración interno (LCI).
    \item Incorpore los módulos radiantes al LCI.
    \item Permita durante los ensayos poder estimar la potencia real y fase con
la que emite cada uno de los módulos radiante sin la necesidad de incorporar
elementos adicionales como ser acopladoes direccionales.
    \item Evite o minimice las caracterizaciones previas que demandan una
cantidad de recursos para nada despreciable y que encarece las campañas.
    \item Permita minimizar la incertidumbre de salida en fase y ganancia.
\end{itemize}

Entre la biliografia de antenas existe un método denominado de calibración por 
acoplamiento mutuo\todo{buscar en los papers este método}. Este método no puede
determinar la potencia y fase absoluta de transmisión de cada módulo radiante, 
solamente la potencia y fase relativa entre los mismos. Por ende, este método no
es completo.

Se propone utilizar el método de acoplamiento mutuo como “complemento” del 
existente denominándolo método de “autocalibracion interna extendida”. Como 
ventaja, no solo se abarca todo el sistema sino que también es fácilmente 
implementable con el hardware ya existente. 

	\section{Cronograma}
	\subsection{Cronograma detallado}
	\todo{agregar cronograma}

	\newpage
	\section{Bibliografía}
	
	\begin{enumerate}[ {[}1{]} ]
		\item \label{ppr:aligment1} F. K. LI, \enquote*{A Method for Detection 
		of Deformations in Large Phased Array Antennas for Spaceborne Synthetic
		Aperture Radars}, IEEE TRANSACTIONS ON ANTENNAS AND PROPAGATION, VOL. 
		AP-32, NO. 5 , MAY 1984.
		
		\item \label{ppr:aligment2} G. M. Shaw and R. B. Dybdal, \enquote*{A 
		Space-Fed Local Oscillator for Spaceborne Phased Arrays}, 1988 IEEE 
		MTT-S Digest.

		\item \label{ppr:brdcast1} EU-AN LEE and C. NELSON DORNY, \enquote*{A 
		Broadcast Reference Technique for Self-calibrating of Large Antenna 
		Phased Arrays}, IEEE TRANSACTIONS ON ANTENNAS AND PROPAGATION, VOL. 37,
		NO. 8, AUGUST 1989.
		
		\item \label{ppr:classic1} Anthony P. Luscombe, \enquote*{Internal 
		Calibration Of The Radarsat Synthetic Aperture Radar}, Geoscience and 
		Remote Sensing Symposium, 1990. IGARSS'90. 'Remote Sensing Science for 
		the Nineties.
		
		\item \label{ppr:absRad1} L. M. H. Ulander, R. K. Hawkins, C. E.
		Livingstone and T . I. Lukowski, \enquote*{Absolute Radiometric 
		Calibration of the CCRS SAR}, IEEE TRANSACTIONS ON GEOSCIENCE AND REMOTE
		SENSING, VOL. 29. NO. 6. NOVEMBER 1991.
		
		\item G. Gonwald *, W. Wiesbeck, \enquote*{Anomalous Effects of Phased 
		Array Antennas due to Mutual Coupling and Feeding Network}, Antennas and
		Propagation Society International Symposium, 1992. AP-S 1992 Digest.
		
		\item C. H. Tang, \enquote*{EFFECTS OF PHASED ARRAY STRUCTURE
		DEFORMATION ANDELEMENTOUTAGE}, Antennas and Propagation Society 
		International Symposium, 1992. AP-S 1992 Digest.
		
		\item \label{ppr:puncTrgt1} Anthony Freeman, \enquote*{SAR Calibration: 
		An Overview}, IEEE Transactions on Geoscience And Remote Sensing, Vol. 
		30, NO. 6, November 1992.
		
		\item \label{ppr:rad2} H. LAUR, P. MEADOWS, J.I. SANCHEZ, E. DWYER, 
		\enquote*{ERS-1 SAR RADIOMETRIC CALIBRATION}, Published in the 
		Proceedings of the CEOS SAR Calibration Workshop (ESA WPP-048) Sept. 93
		
		\item \label{ppr:classic2} M. Zink, \enquote*{CALIBRATION AND 
		PERFORMANCE ANALYSIS OF THE X-SAR SYSTEM}, Geoscience and Remote Sensing
		Syposium, 1994. IGARSS 1994.
		
		\item Mustafa Karaman, Hayrettin Koymen, Abdullah Atalar and Matthew 
		O’Donnell, \enquote*{Influence of Missing Array Elements on Phase 
		Aberration Correction for Medical Ultrasound}, IEEE TRANSACTIONS ON
		ULTRASONICS, FERROELECTRICS, AND FREQUENCY CONTROL, VOL. 41. NO. 5, 
		SEPTEMBER 1994.
		
		\item \label{ppr:classic3} Jorgen Dall, Niels Skou, Erik Lintz 
		Christensen, \enquote*{Pulse-Based Internal Calibration of Polarimetric
		SAR}, Geoscience and Remote Sensing Syposium, 1994. IGARSS 1994.

		\item \label{ppr:rad3} Brian L. Markhaml, Suraiya P. Ahmadz, James R. 
		Irons1 and Darrel L. Williams1 \enquote*{Radiometric Calibration of the
		Landsat-7 Enhanced Thematic Mapper Plus}, Geoscience and Remote Sensing
		Syposium, 1994. IGARSS 1994.
		
		\item \label{ppr:dist1} Masanobu Shimada and Anthony Freeman, \enquote*{
		A Technique for Measurement of Spaceborne SAR Antenna Patterns Using 
		Distributed Targets}, IEEE TRANSACTIONS ON GEOSCIENCE AND REMOTE 
		SENSING, VOL. 33, NO. I , JANUARY 1995.
		
		\item \label{ppr:abs-rad-ical1}Anthony Freeman, M. Alves, B. Chapman, J.
		Cruz, Y. Kim, S. Shaffer, J. Sun, E. Turner, and Kamal Sarabandi, 
		\enquote*{SIR-C Data Quality and Calibration Results}, IEEE TRANSACTIONS
		ON GEOSCIENCE AND REMOTE SENSING, VOL. 33, NO. 4, JULY 1995.
		
		\item \label{ppr:mutual1} Ashok Agrawal and Allan Jablon, \enquote*{A 
		CALIBRATION TECHNIQUE FOR ACTIVE PHASED ARRAY ANTENNAS}, Phased Array 
		Systems and Technology, 2003, IEEE International Symposium.
		
		\item \label{ppr:puncTrgt2} David Stevens, Peter Bird, Gordon Keyte, 
		\enquote*{A SAR antenna calibration method}, Geoscience and Remote 
		Sensing Syposium, 1996. IGARSS 1996.
		
		\item \label{ppr:rev1} Uoshihisa Hara, Chikako Ohno, Masafumi Iwamoto, 
		and Natsuki Kondo, \enquote*{A Study on Radiometric Calibration of Next
		Generation Spaceborne SAR}, Geoscience and Remote Sensing Syposium, 
		1997. IGARSS 1997.

		\item \label{ppr:ext1} Seth D. Silverstein, \enquote*{Algorithms for 
		Remote Calibration of Active Phased Array Antennas for Communication 
		Satellites}, Signals, Systems and Computers, 1996.
		
		\item \label{ppr:ext2} J . M . Ashe, W . Yang, T. Shen, G. Xu and S. D.
		Silverstein, \enquote*{Experimental Study of Remote Calibration 
		Algorithms for Active Phased Array Transmitters}, Signals, Systems and
		Computers, 1996.
		
		\item A. Freeman, \enquote*{THE NEED FOR SAR CALIBRATION}, Geoscience
		and Remote Sensing Syposium, 1989. IGARSS 1989. 
		
		\item \label{ppr:ext3} Daniel S. Purdy, \enquote*{In Orbit Active Array
		Calibration for NASA’s LightSAR}, Radar Conference, 1999. The Record of
		the 1999 IEEE.
		
		\item \label{ppr:punc-ext1} Hong Jun, Zang Bing-rong, Wing Hong-qi, 
		\enquote*{The progress of the aiirborne SAR calibration techniques in 
		China}, Geoscience and Remote Sensing Syposium, 1989. IGARSS 1989.

		\item \label{ppr:mutual2} Charles Shipley and Don Woods \enquote*{MUTUAL
		COUPLING-BASED CALIBRATION OF PHASED ARRAY ANTENNAS}, Phased Array 
		Systems and Technology, 2000. Proceedings. 2000 IEEE International 
		Conference.
		
		\item Envisat Cal-Val Team, \enquote*{ENVISAT Calibration and Validation
		Plani}, ESA.
		
		\item \label{ppr:rad4} Jeffrey A. Mendenhall, \enquote*{Radiometric 
		Calibration and Flight Validation}, ALI Tech\_Trans-1 JAM 10/23/01.

		
		\item Marian Werner, Martin Haeusler, \enquote*{X-SAR/SRTM Instrument 
		Phase Error Calibration}, Geoscience and Remote Sensing Syposium, 1989.
		IGARSS 1989.
		
		\item \label{ppr:rad5} Volker Kaltenborn, \enquote*{Intern Report: 
		Volker Kaltenborn}, Alaska SAR Facility (ASF).
		
		\item \label{ppr:classic4} D J Bibby \& A J Knight, \enquote*{A RF Model
		of an Active Array Antenna for a Spaceborne SAR}, Antennas and 
		Propagation, 2003. (ICAP 2003). Twelfth International Conference.
		
		\item \label{ppr:classic5} Daniel Bast, \enquote*{Parameters Affecting 
		Orthogonal SAR Transmit and Receive Module Calibration}, European Space
		and Technology Centre, European Space Agency EOP-FI, Keplerlaan-1, 2200
		AG Noordwijk (The Netherlands).
		
		\item \label{ppr:puncTrgt3} M. Shimada, T. Tadono, and M. Matsuoka,
		\enquote*{Calibration and Validation of PALSAR}, Geoscience and Remote 
		Sensing Syposium, 2002. IGARSS 2002.
		
		\item \label{ppr:rad6} Kai-Jen Calvin Tien, Roger D. de Roo, \enquote*{
		Comparsion of Different Microwave Radiometric Calibration Techniques}, 
		Geoscience and Remote Sensing Syposium, 2004. IGARSS 2004.
		
		\item \label{ppr:aligment3} P. Zulch, R. Hancock, and J. McKay, 
		\enquote*{Array Deformation Performance Impacts on a LEO L-Band GMTI 
		SBR} 0-7803-8870-4/05/\$20.00© 2005 IEEE. IEEEAC paper \#1532, Version 
		5, Updated December 20, 2004.

		\item \label{ppr:classic6} A. G. Stove, \enquote*{ISSUES FOR THE 
		AUTOCALIBRATION OF PHASED ARRAY RADARS}, $1^{st}$ EMRS DTC Technical 
		Conference – Edinburgh 2004.
		
		\item \label{ppr:aligment4} J. J. M de Wit, W. L. van Rossum, M. P. G. 
		Otten, A. G. P. Koekenberg \enquote*{Concept for Measuring and 
		Compensating Array Deformation}, Proceedings of the 4th European Radar 
		Conference.
		
		\item \label{ppr:rad7} Marco Schwerdt, Benjamin Bräutigam, Markus 
		Bachmann, Björn Döring \enquote*{TerraSAR-X Calibration - First Results}

		\item \label{ppr:classic-ext1} S. K. Srivastava, N. W. Shepherd, T. I. 
		Lukowski and R. K. Hawkins \enquote*{PLANS FOR RADARSAT IMAGE DATA
		CALIBRATION}, Adv. Space Res. Vol. 17, No. 1, pp (1)89-(1)96, 1996.

		\item \label{ppr:aligment5} Elena Zaitsev, John Hoffman, \enquote*{
		Phased Array Flatness Effects on Antenna System Performance}, 978-1-4244
		-5128-9/10/\$26.00 \copyright 2010 IEEE

		
		\item Tore Lindgren and Johan Borg \enquote*{A Measurement System for 
		the Position and Phase Errors of the Elements in an Antenna Array 
		Subject to Mutual Coupling}, Hindawi Publishing Corporation. 
		International Journal of Antennas and Propagation Volume 2012, 
		Article ID 526121.

		\item \label{ppr:classic7} Shuo Wang, Haiming Qi, and Weidong Yu 
		\enquote*{An Internal Calibration Scheme for Polarimetric Synthetic 
		Aperture Radar System}, IEEE transactions on geoscience and remote 
		sensing, vol. 49, NO. 1, January 2011.
		
		\item \label{ppr:mutual-ext1} Wei Chen, Joni Polili Lie, Boon Poh Ng, 
		Tao Wang and Meng Hwa Er \enquote*{Joint Gain/Phase and Mutual Coupling
		Array Calibration Technique with Single Calibrating Source}, Hindawi 
		Publishing Corporation International Journal of Antennas and Propagation
		Volume 2012, Article ID 625165.

		
		\item N. Fistas and A. Manikas \enquote*{A NEW GENERAL GLOBAL ARRAY 
		CALIBRATION METHOD}, ICASSP PROCEEDINGS, APRIL 94.
		
		\item \label{ppr:classic8} Eduardo Makhoul, Antoni Broquetas, 
		Francisco López-Dekker, Josep Closa, and Paula Saameno \enquote*{
		Evaluation of the Internal Calibration Methodologies for Spaceborne 
		Synthetic Aperture Radars with Active Phased Array Antennas}, IEEE 
		JOURNAL OF SELECTED TOPICS IN APPLIED EARTH OBSERVATIONS AND REMOTE 
		SENSING, VOL. 5, NO. 3, JUNE 2012.

		\item \label{ppr:aligment6} Guillaume Lesueur, Daniel Caer, Thomas 
		Merlet, Pierre Granger \enquote*{Active compensation techniques for 
		deformable phased array antenna}, Thales Air Systems Hameau de Roussigny
		, 91470 Limours, France.

		\item Neil Chamberlain, Constantine Andricos, Andrew Berkun, Kendra 
		Kumley, Vladimir Krimskiy, Richard Hodges, Suzanne Spitz \enquote*{T/R 
		Module Development for Large Aperture L-band Phased Array}, IEEEAC paper
		\# 1179 Version 5, Updated December 10, 2004.

	\end{enumerate}

	\newpage
	\section{Currículum Vitae}
	
	\newpage
	\section{Materias Aprobadas}	
	\begin{center}
		\scriptsize
		\centering
		\begin{longtable}{|p{3.5cm}|c|c|c|p{1.4cm}|c|c|}
			\hline 
			\bfseries Materia & \bfseries Fecha & \bfseries Resultado & 
			\bfseries Nota & \bfseries Forma de aprobación & \bfseries Acta & 
		 	\bfseries	Plan \\
			\hline
			(7801) IDIOMA INGLES & 29/06/2009 & Aprobado & 6 & Examen & 
			18-22-214 & 1986 \\
			\hline
			(6103) ANALISIS MATEMATICO II A & 13/08/2009 & Aprobado & 5 & Examen
			& 1-154-76 & 1986 \\
			\hline
			(7540) ALGORITMOS Y PROGRAMACION I & 18/08/2009 & Aprobado & 9 & 
			Examen & 17-101-183 & 1986 \\
			\hline
			(6201) FISICA I A & 18/08/2009 & Aprobado & 6 & Examen & 2-107-176 &
			1986 \\
			\hline
			(7541) ALGORITMOS Y PROGRAMACION II & 10/02/2010 & Aprobado & 8 & 
			Examen & 17-103-4 & 1986 \\
			\hline
			(6301) QUIMICA & 15/02/2010 & Aprobado & 6 & Examen & 3-75-25 & 1986
			\\
			\hline
			(6203) FISICA II A & 25/02/2010 & Aprobado & 8 & Examen & 2-108-63 &
			1986 \\
			\hline
			(6107) MATEMATICA DISCRETA & 02/03/2010 & Aprobado & 6 & Examen & 
			1-156-42 & 1986 \\
			\hline
			(7507) ALGORITMOS Y PROGRAMACION III & 06/07/2010 & Aprobado & 8 & 
			Examen & 17-104-12 & 1986 \\
			\hline
			(7531) TEORIA DE LENGUAJE & 07/07/2010 & Aprobado & 8 & Examen & 
			17-104-24 & 1986 \\
			\hline
			(6602) LABORATORIO & 12/07/2010 & Aprobado & 8 & Examen & 6-139-14 &
			1986 \\
			\hline
			(6108) ALGEBRA II A & 14/07/2010 & Aprobado & 8 & Examen & 1-153-219
			& 1986 \\
			\hline
			(6215) FISICA III D & 22/12/2010 & Aprobado & 8 & Examen & 2-108-221
			& 1986 \\
			\hline
			(6109) PROBABILIDAD Y ESTADISTICA B & 10/02/2011 & Aprobado & 8 & 
			Examen & 1-155-250 & 1986 \\
			\hline
			(6670) ESTRUCTURA DEL COMPUTADOR & 16/02/2011 & Aprobado & 7 & 
			Examen & 6-140-35 & 1986 \\
			\hline
			(6110) ANÁLISIS MATEMÁTICO III A & 24/02/2011 & Aprobado & 5 & 
			Examen & 1-157-49 & 1986 \\
			\hline
			(7512) ANALISIS NUMERICO I & 25/02/2011 & Aprobado & 8 & Examen & 
			17-106-55 & 1986 \\
			\hline
			(7542) TALLER DE PROGRAMACION I & 13/07/2011 & Aprobado & 10 & 
			Examen & 17-107-9 & 1986 \\
			\hline
			(7506) ORGANIZACION DE DATOS & 14/07/2011 & Aprobado & 5 & Examen & 
			17-107-17 & 1986 \\
			\hline
			(6620) ORGANIZACION DE COMPUTADORAS & 08/08/2011 & Aprobado & 8 & 
			Examen & 6-141-16 & 1986 \\
			\hline
			(7112) ESTRUCTURA DE LAS ORGANIZACIONES & 14/12/2011 & Aprobado & 4 
			& Examen & 11-153-84 & 1986 \\
			\hline
			(7114) MODELOS Y OPTIMIZACION I & 22/12/2011 & Aprobado & 6 & Examen
			& 11-153-145 & 1986 \\
			\hline
			(6211) MECANICA RACIONAL & 14/02/2012 & Aprobado & 8 & Examen & 
			2-109-191 & 1986 \\
			\hline
			(6606) ANALISIS DE CIRCUITOS & 15/02/2012 & Aprobado & 7 & Examen &
			6-141-176 & 1986 \\
			\hline
			(6674) SEÑALES Y SISTEMAS & 24/02/2012 & Aprobado & 8 & Examen & 
			6-141-206 & 1986 \\
			\hline
			(7508) SISTEMAS OPERATIVOS & 12/07/2012 & Aprobado & 7 & Examen & 
			17-109-103 & 1986 \\
			\hline
			(6609) LABORATORIO DE MICROCOMPUTADORAS & 13/07/2012 & Aprobado & 8
			& Examen & 6-142-46 & 1986 \\
			\hline
			(7509) ANALISIS DE LA INFORMACION & 13/08/2012 & Aprobado & 6 & 
			Examen & 17-110-54 & 1986 \\
			\hline
			(7552) TALLER DE PROGRAMACION II & 17/08/2012 & Aprobado & 10 & 
			Examen & 17-110-87 & 1986 \\
			\hline
			(7510) TECNICAS DE DISEÑO & 04/02/2013 & Aprobado & 6 & Examen & 
			17-111-36 & 1986 \\
			\hline
			(7515) BASE DE DATOS & 06/02/2013 & Aprobado & 8 & Examen & 
			17-111-44 & 1986 \\
			\hline
			(6608) CIRCUITOS ELECTRONICOS I & 27/02/2013 & Aprobado & 8 & Examen
			& 6-143-126 & 1986 \\
			\hline
			(7140) LEGISLACION Y EJERCICIO PROFESIONAL DE LA ING. EN INFORMÁTICA
			& 13/12/2013 & Aprobado & 8 & Examen & 71-0001453 & 1986 \\
			\hline
			(6618) TEORIA DE CONTROL I & 05/08/2013 & Aprobado & 6 & Examen & 
			86-0001220 & 1986 \\
			\hline
			(6675) PROCESOS ESTOCÁSTICOS & 09/08/2013 & Aprobado & 9 & Examen & 
			86-0001265 & 1986 \\
			\hline
			(7559) TECNICAS DE PROGRAMACION CONCURRENTE I & 13/08/2013 & 
			Aprobado & 8 & Examen & 95-0001370 & 1986 \\
			\hline
			(6669) CRIPTOGRAFIA Y SEGURIDAD INFORMATICA & 16/08/2013 & Aprobado
			& 8 & Examen & 86-0001295 & 1986 \\
			\hline
			(7567) SIST.AUTOM.DE DIAG.Y DETEC.DE FALLAS I & 04/08/2014 & 
			Aprobado & 7 & Examen & 95-0002261 & 1986 \\
			\hline
			(7565) MANUFACTURA INTEGRADA POR COMP.(CIM) I & 07/08/2014 & 
			Aprobado & 7 & Examen & 95-0002312 & 1986 \\
			\hline
			(7568) SIST.DE SOPORTE P/CELDAS DE PROD FLEXIB. & 10/12/2014 & 
			Aprobado & 10 & Examen & 95-0002440 & 1986 \\
			\hline
			(7566) MANUFACTURA INTEGRADA POR COMP.(CIM) II & 11/12/2014 & 
			Aprobado & 7 & Examen & 95-0002462 & 1986 \\
			\hline
			(6405) ESTATICA Y RESISTENCIA DE MATERIALES B & 15/12/2014 & 
			Aprobado & 9 & Examen & 64-0001589 & 1986 \\
			\hline
			\caption{Materias Aprobadas} \label{tab:matApr}
		\end{longtable}
	\end{center}
	\todo{materias faltantes: Introducción a los sistemas distribuidos, 
	Materiales industriales 1}
	\newpage
	\section{Plan de cursada}
	
	\begin{table}[!htb]
		\centering
		\begin{tabular}{|c|c|c|c|}
			\hline
			Código & Denominación & Créditos & Fecha \\
			\hline
			75.00 & TESIS & 24-OBL & 2 - 2015 \\
			\hline
		\end{tabular}
		\caption{Plan de cursada} \label{tabPlanCursada}
	\end{table}

	TOTAL CRÉDITOS: 24	
\end{document}
