% Appendix Template

\chapter{Definiciones Matemáticas} % Main appendix title

\label{AppendixB} % Change X to a consecutive letter; for referencing this appendix elsewhere, use \ref{AppendixX}

\lhead{Appendix B. \emph{Definiciones Matemáticas}} % Change X to a consecutive letter; this is for the header on each page - perhaps a shortened title


\section{Cuadrados mínimos} \label{sec:meanSquare}

Esta sección describe una estrategia de resolución para cuando un problema, del tipo $\mathbf{Ax} = \mathbf{b}$, no tienen solución. 
La solución encontrada devuelve una \textbf{x} que deje a \textbf{Ax} tan cercana a \textbf{b} como sea posible.  

Si \textbf{A} es de $m x n$ y \textbf{b} está en $\mathds{R}^m$, una solución por mínimos cuadrados de $\mathbf{Ax} = \mathbf{b}$
es una $\hat{x}$ en $\mathds{R}^n$ tal que

$$
\parallel \mathbf{b} - \mathbf{A\hat{x}}\parallel \le \parallel\mathbf{b}-\mathbf{Ax} \parallel
$$

para toda \textbf{x} en $\mathds{R}^n$.

El aspecto más importante del problema de mínimos cuadrados es que no importa cuál \textbf{x} se elija, el vector \textbf{Ax}
necesariamente estará en el espacio de columnas. Así que se busca un \textbf{x} adecuado para convertir a \textbf{Ax} en el 
punto de Col \textbf{A} más cercano a \textbf{b}. (Por supuesto, si sucede que \textbf{b} está en Col \textbf{A}, entonces 
\textbf{b} es \textbf{Ax} para algún \textbf{x}, y tal \textbf{x} es una “solución por mínimos cuadrados”.)

El conjunto de soluciones por mínimos cuadrados de $\mathbf{Ax} = \mathbf{b}$ coincide con el conjunto no vacío de soluciones
de las ecuaciones normales $\mathbf{A}^T\mathbf{Ax} = \mathbf{A}^T\mathbf{b}$ \cite{MatrixMin}.


\section{Matriz Hadamard}

Esta matriz fue descubierta por el matemático Jacques Hadamard, es una matriz cuadrada con valores $1$ o $-1$ y sus columnas 
son ortogonales. Sus propiedades son las siguientes \cite{Seberry2005}. 

Si se tiene una matriz $H$ de orden $n$, su traspuesta está cercamente relacionada con su inversa:

$$ H H^{\mathrm{T}} = n I_n $$

donde $I_n$ es la matriz de identidad de dimensión $n x n$ y $H^\mathrm{T}$ es la traspuesta de $H$. Esta propiedad es válida a causa 
que las columnas de $H$ son vectores ortogonales en el campo de los números reales y cada uno tiene una longitud de $\sqrt n$. A
su vez, indica que cuando tiene sus filas ortogonales dos a dos, la primera fila y columna son formadas sólamente por el valor +1
\cite{Armario}.

Dividiendo H por su longitud, se obtiene una matriz ortonormal que su traspuesta también es su inversa. El determinante es:

$$ \operatorname{det}(H) = \pm n^{\frac{n}{2}} $$

Donde $\operatorname{det}(H)$ es el determinante de $H$.

Si se supone que $M$ es una matriz compleja de orden n, con valores que cumplen la relación $|M_{ij}| \le 1$, por cada $i,j$ 
entre 1 y $n$. Entonces el determinante de la matriz hadamard resulta,
		    
$$ |\operatorname{det}(M)| \leq n^{n/2}. $$

La igualdad es válida solamente si $M$ es real y solo si $M$ también es una matriz Hadamard. La restricción sobre el órden
$n$ de dicha matriz es que la misma debe ser 1, 2 o múltiplo de 4.


\subsection{Construcción de Silvester}

Los primeros ejemplos de construcción de matrices Hadamard fueron realizados por James Joseph Sylvester en 1867. Si H es 
dicha matriz de orden $n$, su construcción es como sigue.

$$ \begin{bmatrix} H & H\\H & -H\end{bmatrix} $$

Resulta una matriz de Hadamard de orden $2n$. Este proceso puede ser repetido para obtener las matrices siguientes, 
conocidas también como matrices Walsh. 

$$ H_1 = \begin{bmatrix} 1 \end{bmatrix}, $$

$$  H_2 = \begin{bmatrix} 1 & 1 \\ 1 & -1 \end{bmatrix}, $$

and

$$ H_{2^k} = \begin{bmatrix} H_{2^{k-1}} & H_{2^{k-1}}\\ H_{2^{k-1}} & -H_{2^{k-1}}\end{bmatrix} $$

para $2 \le k \in N$ \cite{Singhal2012}.
