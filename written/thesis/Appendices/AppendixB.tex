% Appendix Template

\chapter{Definiciones Matemáticas} % Main appendix title

\label{AppendixB} % Change X to a consecutive letter; for referencing this appendix elsewhere, use \ref{AppendixX}

\lhead{Appendix B. \emph{Definiciones Matemáticas}} % Change X to a consecutive letter; this is for the header on each page - perhaps a shortened title

\section{Propiedades de las Matrices}
\todo{poner definicion de rango y cosas asi}

\section{Cuadrados mínimos}
\todo{dejar toda la definición encontrada de los cuadrados mínimos}

\section{Matriz Hadamard}
Esta matriz fue descubierta por el matemático Jacques Hadamard, es una matriz cuadrada con valores $1$ o $-1$ y sus columnas 
son ortogonales. Sus propiedades son las siguientes \cite{HadamardWiki}. 

Si se tiene una matriz $H$ de orden $n$, su traspuesta está cercamente relacionada con su inversa. Y su fórmula es:

$$ H H^{\mathrm{T}} = n I_n $$

Donde $I_n$ es la matriz de identidad de dimensión $n x n$ y $H^\mathrm{T}$ es la traspuesta de $H$. Esta propiedad es válida a causa 
que las columnas de $H$ son vectores ortogonales en el campo de los números reales y cada uno tiene una longitud de $\sqrt n$.
Dividiendo H por su longitud, se obtiene una matriz ortonormal que su traspuesta también es su inversa. El determinante es:

$$ \operatorname{det}(H) = \pm n^{\frac{n}{2}} $$

Donde $\operatorname{det}(H)$ es el determinante de $H$.

Si se supone que $M$ es una matriz compleja de orden n, con valores que cumplen la relación $|M_{ij}| \le 1$, por cada $i,j$ 
entre 1 y $n$. Entonces el determinante de la matriz hadamard resulta,
		    
$$ |\operatorname{det}(M)| \leq n^{n/2}. $$

La igualdad es válida solamente si $M$ es real y solo si $M$ también es una matriz Hadamard.

El orden de una matriz hadamard debe ser 1, 2 o un múltiplo de 4.

\subsection{Construcción de Silvester}

Los primeros ejemplos de construcción de matrices Hadamard fueron realizados por James Joseph Sylvester en 1867. Si H es 
dicha matriz de orden $n$, su construcción es como sigue.

$$ \begin{bmatrix} H & H\\H & -H\end{bmatrix} $$

Resulta una matriz de Hadamard de orden $2n$. Este proceso puede ser repetido para obtener las matrices siguientes, 
conocidas también como matrices Walsh. 

$$ H_1 = \begin{bmatrix} 1 \end{bmatrix}, $$

$$  H_2 = \begin{bmatrix} 1 & 1 \\ 1 & -1 \end{bmatrix}, $$

and

$$ H_{2^k} = \begin{bmatrix} H_{2^{k-1}} & H_{2^{k-1}}\\ H_{2^{k-1}} & -H_{2^{k-1}}\end{bmatrix} $$

para $2 \le k \in N$.
