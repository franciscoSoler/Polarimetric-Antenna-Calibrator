% Chapter 1

\chapter{Introducción} % Chapter title

\label{ch:introduccion} % For referencing the chapter elsewhere, use \autoref{ch:introduccion} 
\lhead{\emph{Introducción}} % Change X to a consecutive number; this is for the header on each page - perhaps a shortened title

%----------------------------------------------------------------------------------------
En este capítulo se dará una breve introducción a la problemática a resolver. Las siguientes secciones tratarán la motivación,
la definición del problema a trabajar y el plan de tesis.

\section{Definiciones previas}

A continuación se introducen algunas definiciones previas necesarias para el desarrollo de la tesis.


{\textbf{Antena:}} Es un objeto eléctrico el cual convierte potencia eléctrica en ondas de radio y viceversa. Usualmente es 
utilizada con un transmisor o receptor de radio. En transmisión, el transmisor provee una corriente eléctrica oscilando en 
RF en los terminales de la antena, y la antena radía dicha energía como ondas electromagnéticas (ondas de radio). En 
recepción, una antena intercepta parte de la onda electromagnética en orden de producir una mínima tensión en sus terminales,
que es aplicado al receptor a ser amplificado \cite{AntennaWiki}.

{\textbf{Polarización:}} La polarización de una onda de radio en general hace referencia a la orientación del campo 
eléctrico (plano-E) de la onda de radio con respecto a la superficie terrestre. La antena que la genera determina, de acuerdo
con su estructura física y orientación, qué tipo de polarización tendrá la onda emitida. Por dar un ejemplo, una antena que
sea simplemente un cable recto poseerá una polarización cuando se la monta verticalmente, y otra diferente cuando sea 
horizontalmente. Como una onda transversal, el campo magnético de una onda de radio es perpendicular al eléctrico en campo 
lejano, pero, por convención, la polarización hace referencia a la dirección del campo eléctrico \cite{AntennaWiki}.

\todo{agregar ROE y SWR como definición}

{\textbf{Adaptación de impedancias:}} La máxima transferencia de potencia requiere que la impedancia de un sistema de 
antena esté adaptada al conjugado complejo de la impedancia del transmisor o receptor. En el caso de un transmisor, 
la impedancia adaptada deseada no necesariamente corresponde a la impedancia de salida dinámica del transmisor analizado
como una fuente de impedancia sino al valor de diseño (típicamente de $50 \Omega$) requerido para una operación eficiente
y segura del circuito de transmisión. Normalmente, la impedancia deseada es resistiva pero un transmisor (y algunos receptores)
pueden tener ajustes adicionales para cancelar una cierta cantidad de reactancia en orden de obtener dicha adaptación. Cuando
una linea de transmisión es usada entre la antena y el transmisor (o receptor) uno generalmente desearía un sistema de antena
con una impedancia resistiva y cercana a la impedancia característica de dicha línea de transmisión, para minimizar el SWR \todo{buscar que es el swl}
e incrementar la transmisión \cite{AntennaWiki}. 


Maximum power transfer requires matching the impedance of an antenna system (as seen looking into the transmission line) to the complex conjugate of the impedance of the receiver or transmitter. In the case of a transmitter, however, the desired matching impedance might not correspond to the dynamic output impedance of the transmitter as analyzed as a source impedance but rather the design value (typically 50 ohms) required for efficient and safe operation of the transmitting circuitry. The intended impedance is normally resistive but a transmitter (and some receivers) may have additional adjustments to cancel a certain amount of reactance in order to "tweak" the match. When a transmission line is used in between the antenna and the transmitter (or receiver) one generally would like an antenna system whose impedance is resistive and near the characteristic impedance of that transmission line in order to minimize the standing wave ratio (SWR) and the increase in transmission line losses it entails, in addition to supplying a good match at the transmitter or receiver itself.



{\textbf{Antena de arreglo de fase:}} \todo{poner referencias en esta definicion} (en inglés: phased array) es un conjunto de antenas en que las fases relativas de las 
señales con que se alimenta cada antena se varían intencionadamente con objeto de alterar el diagrama de radiación del 
conjunto. Lo normal es reforzar la radiación en una dirección concreta y suprimirla en direcciones indeseadas. Si todos los 
elementos del arreglo de fase están contenidos en el mismo plano y la señal con que se alimentan es de la misma fase, 
entonces se estará reforzando la dirección perpendicular a ese plano. Si se altera la fase relativa de las señales se podrá
\enquote*{mover} el haz (en realidad lo que se está haciendo es cambiar la dirección en la cual las interferencias son 
constructivas). Se consigue de este modo hacer barridos sin necesidad de movimiento físico, con la ventaja añadida de que 
se pueden barrer ángulos del orden de miles de grados por segundo. Esto permite utilizar la antena para compaginar 
simultáneamente funciones de detección y de seguimiento muchos blancos individuales, asi como para obtener imágenes de 
apertura sintética. Su uso se va extendiendo debido a la confiabilidad derivada del hecho de que no tienen partes móviles. 
Casi todos los radares militares modernos se basan en phased arrays, relegando los sistemas basados en antenas rotatorias a 
aplicaciones donde el costo es un factor determinante (tráfico aéreo, meteorología, etc) Su uso está también extendido en
aeronaves militares debido a su capacidad de seguir múltiples objetivos. El primer avión en usar uno fue el B-1B Lancer, y 
el primer caza, el MiG-31 ruso.\\

{\textbf{Calibración:}} Operación que, bajo determinadas condiciones, en un primer paso, establece una relación entre 
valores cuantitativos con incertidumbres de medición provistas por estándares de medición e indicaciones correspondientes
con incertidumbres asociadas (del instrumento calibrado o estándar secundario) y, en un segundo paso, usa
esta información para establecer una relación para obtener un resultado de una medición desde una indicación \cite{CalDef}.

{\textbf{Calibración interna:}} Las mediciones obtenidas utilizando este sistema son útiles solamente al utilizarlas en 
conjuto a los resultados de los test realizados en tierra, los cuales definen la relación entre lo observado y la 
performance de los párametros del sistema. Ejemplos de satélites para los cuales se utilizó calibración interna son el 
E-ERS-1 y el SIR-C.

Los tests realizados en tierra para dichos sistemas complejos son sobre la electrónica de RF, la electrónica digital y la 
antena sobre temperatura; para los cuales es preferible realizarlos, cuando sea posible, en un ambiente al vacío. Los 
parámetros clave del sistema, como: potencia transmitida, pérdidas de transmisión o recepción, ganancia de recepción, 
ganancia y patrón de la antena, linealidad de la electrónica digital/RF, rango dinámico, y fase/amplitud vs estabilidad de la 
frecuencia son medidos en función de la temperatura para cada seteo de ganancia del radar y PRF (Pulse Repetition Frequency). 
Los elementos de calibración, como medidores de temperatura, de corriente y de potencia, podrán permitir la determinación de 
la performance del sistema en función de la variación de dichos parámetros.
    
Esta técnica asume que la variación en la performance del sistema puede ser modelada en función de los parámetros observables. 
También, se asume que los componentes de calibración están correctamente calibrados y son estables en el paso del tiempo. 
Además de estos tests, en la mayoría de los sistemas de radar se realizan mediciones de componentes de RF utilizando lazos de 
calibración \cite{Curlander1991}.

{\textbf{Caracterización:}} significa medir (el desempeño de) los parámetros de la antena con el objeto de describir en 
forma comprensible su comportamiento \cite{Mittermayer2007}.

{\textbf{Caracterización del sistema:}} adicionalmente incluye la determinación de las curvas de características basadas 
en los modelos y en las mediciones indirectas de los parámetros, en el caso en que los parámetros no pueden ser medidos 
en forma directa \cite{Mittermayer2007}.

{\textbf{Calibración:}} significa el ajuste de la representación de los parámetros físicos en los productos que arroje la 
antena de manera que los mismos puedan ser medidos dentro la especificación \cite{Mittermayer2007}.

{\textbf{Verificación:}} significa mostrar mediante mediciones apropiadas que un sistema o subsistema trabaja dentro de su 
especificación (o conjunto de especificaciones) o que un producto cumple con su especificación \cite{Mittermayer2007}.


{\textbf{Campaña de caracterización:}}

\section{Contexto}

Una antena es un transductor que transforma la energía eléctrica conducida en energía electromagnética radiada cuando 
trabaja como transmisora, o viceversa cuando actúa como receptora.

Hay una amplia variedad de antenas en la actualidad, se realiza una agrupación simplificada en la tabla \ref{tab:type_antennas}.

\begin{table}[H]
  \footnotesize
  \centering
  \begin{tabular}{|c|p{9cm}|}
	\hline
	\textbf{Tipo Antena} & \textbf{Características} \\\hline
	Isotrópica & Es una antena hipotética, la cual radía la misma potencia en todas las direcciones.\\\hline
	Monopolo & Es un único módulo radiante, generalmente un simple cilindro de metal conectado en una punta a un plano de 
	tierra y en la otra a la línea de alimentación. Usos: Radio AM/FM, walkie talkie, etc. \\\hline
	Dipolo & Es el tipo de antena más comunmente utilizado, consiste en dos RMs simétricos, los cuales son cilindros 
	metálicos o cables. Usos: antena de canales de tv VHF, antena de televisión analógica, etc. \\\hline
	Arreglo & Consiste en múltiples antenas trabajando como una única antena. Usos: Transmisión de canales de tv en VHF, 
	detección de misiles, comunicaciones satelitales, etc.\\\hline
	Loop & Consiste en una o múltiples vueltas de cable. Trabajan con campos magnéticos en vez de eléctricos. Usos: receptora 
	de canales UHF de tv, receptoras de radio de Amplitud Modulada, etc.\\\hline
	Apertura & Son el principal tipo de antenas direccionales utilizadas en frecuencias microondas. Consisten en una antena 
	del tipo dipolo o Loop junto a una estructura que guía las ondas en una dirección determinada. Usos: Comunicaciones 
	satelitales, comunicaciones marinas, etc.\\\hline
  \end{tabular}
  \caption{Cacacterísticas de cada grupo principal de antenas}
  \label{tab:type_antennas}
\end{table}

La presente tesis se enfoca en el tipo \enquote*{arreglo de antena de fase controlada de manera electrónica}, en inglés 
sería \enquote*{phase array electronically controlled antenna}. Este tipo de antenas son utilizadas en aplicaciones donde 
es necesario realizar apuntamientos en diferentes direcciones y con altas velocidades. Por ejemplo, comunicaciones móviles 
\cite{Chen2012}, aéreas \cite{MHong1989} y espaciales \cite{Shimada1995}\cite{Makhoul2012}. En el capítulo 
\ref{ch:phasedArray} se explica el principio de funcionamiento básico de este tipo de antena. 

Para generar diversos productos, por ejemplo imágenes satelitales en radares SAR \cite{Freeman1992}, es necesario que la 
antena se encuentre correctamente calibrada \cite{Luscombe1990}\cite{Seifert1996}\cite{Dall1994}. Esto implica, que las 
tolerancias de fases y amplitud se mantengan en el tiempo y/o sus valores sean bien conocidas para cada elemento del arreglo. 
En el capítulo \ref{ch:classicalCalibration} se introduce el concepto de calibración y en particular se hace foco en la 
calibración interna que es el objeto de estudio de esta tesis.

Se explica el método de calibración clásica (en el capítulo \ref{ch:classicalCalibration}), el cual presenta una serie de 
limitaciones. Por ello es que generalmente, este tipo de antenas son calibradas en tierra utilizando fuentes externas de campo 
lejano o cercano \cite{Agrawal2003}. Sin embargo, en aplicaciones aéreas o espaciales, la utilización de dichas fuentes es 
impráctica o difícil de implementar \cite{Aumann1989}. Resulta fundamental un esquema de calibración interna que permita 
tener bajo control estas variables y en el cual el método clásico presenta las limitaciones que se comentan. 

Como consideración a la problemática que ninguno de los métodos a ser presentados de calibración interna pueden determinar
la potencia absoluta de transmisión de la antena como elemento único, se introducen los dos principales métodos de 
calibración externa para complementar esta falencia \todo[inline]{(En el capítulo HHH)}. 

Calibración utilizando blancos puntuales: Los blancos puntuales son dispositivos construidos por el hombre, generalmente son
corner reflectors, transponders, generadores de tono o receptores. La característica de estos dispositivos es que no solo 
son muy chicos con respecto a la mínia área de observación de una antena y que el eco de la señal de dichos dispositivos 
posee mucha más intensidad que dicha área, sino que el reflejo es bien conocido. Por lo tanto, con su uso se puede deducir la 
potencia transmitida.

Calibración utilizando blancos distribuidos: Se refiere a la utilización de blancos naturales de largas áreas con 
propiedades de backscattering homogéneas. Se utiliza la suposición que estas propiedades son estables o que su variación es 
bien conocida.

Una vez introducidos los métodos de calibración externa, se procede a explicar los de calibración interna.

El primer método a mencionar es el llamado método REV, el cual calibra de a un módulo radiante a la vez utilizando una 
antena externa al panel como receptora. El proceso de calibración es el siguiente, se transmite por un módulo radiante
y se va variando la fase hasta obtener la máxima potencia recibida, luego se repite el proceso para todos los módulos. 

Otro método es el convencional, el cual implementa lazos internos de para calibrar los caminos de transmisión y 
recepción de la antena de forma independiente \cite{Makhoul2012}\cite{Luscombe1990}\cite{Seifert1996}\cite{Dall1994}
\cite{Freeman1995}\cite{Bibby2003}\cite{Bast2003}\cite{Stove2004}\cite{Srivastava1996}\cite{Wang2010}.
Para ello, la señal de calibración recibida se la compara con la potencia transmitida, obteniendo así, que dicha diferencia 
se corresponde con el desfasaje de potencia y fase de cada camino de transmisión/recepción. Luego, se opta por caracterizar 
todos aquellos componentes que no entran dentro de ningún lazo de calibración \cite{Freeman1995}. Este método presenta 
algunos inconvenientes. Por dar un par de ejmplos se puede mencionar que los recursos necesarios (tiempo, personal) durante 
la campaña de ensayos previa al lanzamiento impactan en todo el desarrollo de actividades, o que las consecuencias que puede 
traer el hecho de que la caracterización no sea válida porque un determinado componente envejece con el paso del tiempo.

En este contexto, se investiga y propone un método que permita reducir costos 
asociados a la calibración y por ende a los proyectos:

\begin{enumerate}
    \item Evitando la necesidad de realizar caracterizaciones previas de la antena de arreglo de fase.
    \item Permitiendo conocer en tiempo real, y para el estado real de la antena, los valores reales de fase y amplitud en 
		vuelo que transmite la antena, independientemente de su estado de envejecimiento.
\end{enumerate}

En este sentido se investiga y aprovecha el concepto acoplamiento mutuo inherente entre los módulos radiantes de la antena 
\cite{Aumann1989}, pero de manera complementaria al enfoque tradicional. Este método calibra tanto los caminos de transmisión
como recepción a la vez, para ello, se transmite en una polarización y se recibe en otra de a un módulo radiante a la vez. 
Una ventaja que tiene frente a los otros métodos realizados es que no solo calibra la antena en su totalidad, sino también 
sirve para determinar si hay módulos radiantes quemados.

\todo[inline]{pablo, la parte que está debajo del comentario la dejo??? la saco?? donde la pongo}
En esta tesis se realizarán las siguientes tareas:

\begin{enumerate}
    \item Se explicarán las ventajas y desventajas del método propuesto respecto del tradicional.
    \item Se analizarán y propondorán los requerimientos para poder implementarlo.
    \item Se desarrollará un modelo de antena de arreglo de fase polarimétrico 
			básico representativo en parámetros S \cite{Caspers}.
    \item Se añadirán al mismo parámetros de dispersión de comportamiento que 
			permitan analizar su comportamiento.
    \item Se realizará un modelo de calibración que será implementado algoritimicamente.
\end{enumerate}



\section{Motivación}

A la hora de adquirir imágenes satelitales es crucial que se conozca perfectamente la señal emitida y recibida por la antena. 
Ya sea por envejecimiento de los componentes \cite{Agrawal2003} o por variaciones de temperaturas se observan dispersiones de 
las mismas \cite{Keizer2011}. 

El método de calibración tradicional, por lazos de calibración internos, permite calibrar una antena de arreglo de fase, aunque 
adolece de algunos defectos entre los cuales se incluyen:

\begin{enumerate}
    \item Una degradación o directamente rotura de un elemento radiante, el cual está fuera del lazo de calibración interno, no 
			es detectado por la calibración interna.
    \item Existen elementos que quedan fuera del lazo de calibración interno. Como por ejemplo los circuladores. Esto lleva a 
			la necesidad de realizar caracterizaciones en tierra lo cual implica consumo de recursos de proyecto importantes, 
			además de tener que confiar que dicha caracterización será válida (o sea que no habrá envejecimiento de los mismo) 
			luego durante vuelo en toda la vida útil de la antena.
    \item Cada lazo de calibración no se interrelaciona con el resto haciendo que no se pueda disminuir el error de medición por 
			multiplicidad de caminos.
\end{enumerate}

Esto lleva a investigar opciones superadoras, que den lugar a un método de calibración que permita complementar al tradicional 
evitando la necesidad de realizar costosas caracterizaciones en tierra, previendo que los componentes en vuelo puedan envejecer 
y por ende dichas caracterizaciones no ser más validas, permitiendo detectar fallas en elementos que en la calibración interna 
tradicional quedan fuera del lazo de calibración, y disminuyendo la incertidumbre en la determinación de la fase y amplitud de 
salida.

El método que se propone en esta tesis, denominado de \enquote*{método de calibración interna complementada por acoplamientos 
mutos (MCINCAM)} toma la idea de mediciones por acoplamiento mutuo \cite{Agrawal2003}\cite{Shipley2000} \cite{Aumann1989}
\cite{Chen2012}, para integrarla de manera complementaria al método tradicional, sin necesidad de agregar hardware 
adicional, y además establece los requerimientos electrónicos para poder implementarla. En esta tesis además se realiza un modelo 
ad hoc, en parámetros S, de antena, para mostrar la eficacia del método. 


\section{Objetivo de la Tesis}

La presente tesis tiene varios objetivos:

\begin{enumerate}
    \item Presentar conceptualmente el método tradicional de calibración interna de una antena polarimétrica que abarque el 
			sistema completo de transmisión/recepción. con sus virtudes y defectos. Mencionar algunas misiones de ejemplo en
			las cuales el mismo se ha utilizado.
    \item Investigar, desarrollar y presentar conceptualmente un método alternativo de calibración interna que introduzca 
			mejoras al método tradicional sin necesidad de introducir hardware adicional y con la premisa fundamental de 
			reducir los costos asociados a las caracterizaciones que el método tradicional incluye. Introducir los 
			requerimientos necesarios para poder implementarlo.
    \item Investigar, desarrollar y presentar (generar) los algoritmos que permitan representar el modelo de antena 
			polarimétrico (modelo de capa física) que serán utilizados para poder comparar el método tradicional y el 
			alternativo. Dichos modelos deben representar el comportamiento en RF básico de las señales al propagarse por el 
			sistema.
    \item Generar los algoritmos que permitan correr el algoritmo de calibración alternativo de manera de poder comparar ambos 
			métodos.
    \item Sacar conclusiones respecto a los pros y contras del método propuesto, en particular en referencia al método 
			tradicional, por medio de algunos parámetros objetivos como ser tiempo de calibración, caracterizaciones necesarias, 
			incertidumbre de medición, entre otros.
\end{enumerate}    


\section{Especificaciones del problema}

Para poder aplicar e implementar el método de acoplamientos mútuos se deben cumplir las siguinetes hipótesis de base, que a su 
vez son consideradas las especificaciones del problema.

\todo{cambiar numeros por SP}

\begin{enumerate}
    \item Modularización de los componentes de la antena: La antena debe estar compuesta por defasadores, atenuadores, cables, 
		divisores de potencia y módulos radiantes.
    
	\item Modelización de los componentes en RF: Los componentes de la antena deben caracterizarse utilizando parámetros de alta 
		frecuencia.

    \item Sistemas LTI: La aplicación deberá reproducir el comportamiento del sistema, el cual será LTI (lineal e invariante 
		en el tiempo).
    
    \item Variabilidad para determinados componentes: Se deben poder utilizar distintos divisores/combinadores de potencia. 
		La diferencia entre ellos es la cantidad de puertos de salida.
    
    \item Dimensión de antena: Se debe poder configurar la dimensión de la antena.
    \item Distancia entre elementos radiantes: Se debe poder configurar la distancia entre elementos radiantes.
    \item Configuración individual de componentes: Se debe poder configurar los atributos que afecten la modelización de cada 
		componente de la antena. Largo, atenuación y defasaje por metro de cables. 
    
    \item Modelizaciones dispersiones: Se debe poder configurar la dispersión en el comportamiento de cada componente de la 
		antena de forma independiente.

    \item Introducción de elementos de control en el lazo de calibración: Se deben poder configurar los atenuadores y 
		defasadores a la hora de realizar la calibración. 

    \item Calibración clásica: La aplicación debe poder calibrar una antena polarimétrica con el método de calibración convencional.
    
    \item Calibración alternativa: La aplicación debe poder calibrar una antena polarimétrica con el método de calibración alternativo.
    
    \item Calibración ganancia: La aplicación debe poder calibrar la ganancia de transmisión y recepción.
    
    \item Calibración fase: La aplicación debe poder calibrar la fase de transmisión y recepción.
    
    \item Cal. pol. horizontal: La aplicación debe poder calibrar en la polarización horizontal. 
    \item Cal. pol. vertical: La aplicación debe poder calibrar en la polarización vertical. 
    \item Cal. Tx: La aplicación debe poder calibrar en transmisión. 
    \item Cal Rx: La aplicación debe poder calibrar en recepción. 
    
    \item Calibración independiente del estado inicial: Se debe poder alcanzar el estado de calibración deseado partiendo 
		cualquier estado inicial en los defasadores y atenuadores.
    
    \item Transmisión vs recepción por polarizaciones cruzadas: Para calibrar se debe transmitir y recibir en polarizaciones 
		diferentes.
    
    \item Planitud de antena: La antena tiene que ser perfectamente plana. No deben haber imperfecciones.
           
    \item Frecuencia de RF: Se debe poder configurar la frecuencia de trabajo.
	
	\item Receptor ideal: El receptor se lo considera ideal. La medición que se obtiene del mismo no posee incertidumbre.

    \item Dispersión señal de calibración entre pulsos: Se debe poder configurar los parámetros de dispersión (desvío estándar) para 
		la ganancia y fase de la chirp utilizada entre pulsos.

    \item Dispersión chirp réplica: Se debe poder configurar los parámetros de dispersión (desvío estándar) para la 
		ganancia y fase de la chirp réplica utilizada a la hora de realizar la calibración convencional.
    
    \item Dispersión walsh: Se debe poder configurar los parámetros de dispersión (desvío estándar) para la fase de la 
		codificación utilizada a la hora de realizar la calibración convencional.

    \item Simulación falla componente: Se debe poder simular, configurar la destrucción total de elementos de la antena SAR 
		como ser los TRMs.
\end{enumerate}


\section{Metodología de la tesis}
En la presente tesis se investigan los métodos de calibraciones actuales determinando las ventajas, desventajas, 
limitaciones y diferencias que hay entre cada una de ellos. De esta forma, se obtiene una visión global para 
determinar las posibles falencias puede tener este método alternativo.

Posteriormente, se investigan las limitaciones que poseen las antenas polarimétricas para determinar que recaudos 
se deben tener en cuenta a la hora de desarrollar el método.

Luego, tomando todo en cuenta, se determinan las hipótesis necesarias para que el algoritmo funcione correctamente. Para 
la validación del método se realiza un modelo de antena.

Finalmente, se prueban, analizan y documentan los resultados obtenidos de la comparación entre el algoritmo propuesto 
y el algoritmo de la calibración convencional. A su vez, se deja asentado que posibles mejoras se podrían aplicar al 
algoritmo para determinar otros aspectos que están fuera del alcance de esta tesis.

\section{Contribución}

Como contribución se desarrolla un modelo de antena que cumple con todos los requerimientos del problema previamente 
mencionados, logrando así, que sea representativo en RF. A su vez, este modelo fue programado de tal forma, que puede
ser reutilizado para poder modelar cualquier tipo de estrategia de calibración.

Además, este método de calibración aporta el hecho de poder calibrar no solo la potencia, sino que también la fase 
manteniendo el apuntamiento en que se va a utilizar dicha antena.

Como punto final, se presenta la comparación con el método de calibración clásico, mostrando ventajas, desventajas 
que posee y que tan compatibles son entre ambos.


\section{Estructura de la Tesis}

 (1) Seccion "Tipos de calibracion. En particular calibracion externa". (1 hoja)
 Explicar como es. Transponders, corner reflectors. Explicar el problema que trae depender de algo externo. Revisitas espaciadas. Mantenimiento en tierra de estos lugares de calibracion. etc.
 De alguna manera decir que es la mas sencilla pero no alcanza y los problemas que tiene.

 (2) Sección "Calibración Interna Clásica". (5 hojas).
 Explicar como es. Referenciar varios satelites con antenas que la han usado (en Curlander aparecen).
 Explicar para que sirve (solo para ver si un TRM se rompio o no. Enumerar los defectos que tiene si se la quiere usar para lo que no fue pensada (deja cosas fuera del lazo, hay que caracterizar un monton de cosas, si hay falta de planitud no sirve para nada, etc).
 La idea es tirarla abajo. Por recursos necesarios para implementarla (tiempo, gente). Porque depende de las caracterizaciones internas (que puedn tener errores). Porque las cosas no siempre envejecen como uno espera. Porque no incluye a los elementos radiantes (no sabemos si uno se rompio). Porqe no incluye a los circuladores (Que pueden tener variaciones). POrque depende de una planitud perfecfta. etc etc. Se busca un metodo que permita trascender esto. Ademas se busca simularlo, probarlo, ver las hipotesis necesarias a la hora del diseño para poder implementarlo.
 
 (3) Seccion "Propuestas alternativas superadoras al metodo clasico". (5 hojas)
 POrque se es conciente de estas limitaciones y cada vez se le pide mas en cuanto a requerimientos de fase y atenuacion a las antenas de arreglo de fase. Discriminar mejor estos valores permite poder tener mas aplicaciones mucho mas exigentes.
 Mencionar agregar antenas externas como un metodo. Problema de agregar hadware cuando no hay espacio. Perdida de robustez.
 Mencionar otros metodos. Mencionar el metodo de acoplamientos mutuos general.
 Mencionar cosas que no se encuentran en la bibliografia: no se habla de la fase.
 Mencionar que no estan listadas las hipotesis de base. No existe un algoritmo mostrando como aplicarlo a una antena electronica.  No se analiza en profundidad topologias para aumentar el numero de ecuaciones y reducir el numero de incognitas.
 
 (4) Aca arranca el aporte 1: Seccion "Listado de Hipotesis de base necesarias" p/aplicar el metodo de acoplamientos mutuos a una antena polarimetrica con apuntamiento electronico". (10 hojas)
 Para aplicar el metodo de acoplamientos mutuos antena arreglo de fase polarimetrica controlada de manera electronica, y que RESUELVA lo que el METODO CLASICO NO RESUELVE. Punto a punto mostrar de que manera lo resuelve. Esta es la innovacion. Es el 1er centro de la tesis. El listado de estas hipotesis de manera ordenada. Esto yo no lo vi en nigun paper ni en ningun lado.
 
 (5) Aca sigue el aporte 2: Seccion "Algoritmo de capa 1".  (30 hojas)
 Para poder probar el metodo consturuis vos (que antes no existia) un modelo en base a algoritmo.
 Propuesta de modelos en parametros S (breve introduccion a parametros S). Modularización. Descripcion de los módulos individualmente, entrada, salida, internamente. Interconexion entre modulos que permiten formar la red. Facilidad para cambiar las topologias y probar diferentes para ver ventajas y desventajas (se pueden poner dos de ejemplo y hacer una comprarcion) Pruebas para validarlo punta a punta.
 Propuesta de un par de topologias. Se escoge una de ellas porque genera mas ecuaciones que incognitas. Ventajas y desventajas de usar una RFDN mas o menos subdividida.
 
 (6) Aca sigue el aporte 3: Seccion "Algoritmo de capa 2". (30 hojas)
 Que es lo que se desea simular.
 Metodo clasico para calibrar. PCC.
 Metodo innovador utilizando matrices y aprovechando la SUPERPOSICION de caminos y simplificaciones.
 
 (7) Aca sigue el aporte 4: Seccion "Conclusiones del uso del algoritmo" (10 hojas).
 Este metodo supera al clasico, lo complementa. Solo se requiere tener las hipotesis de base (aporte1). Se require ademas un algoritmo para probarlo de capa fisicia (aporte 2). Luego el algoritmo funcional que corre sobre el fisico (aporte 3).
 
 (8) Seccion trabajo a futuro. (2 hojas con cosas que se podrian seguir investigando).
 
 (9) Apendices


Los siguientes capítulos se organizan de la siguiente manera: 
\begin{itemize}
	\item En el capítulo dos se presenta y detalla lo que es una antena, un patrón de antena y como se la modela en RF.
	\item En el capítulo tres se detalla la calibración clásica con sus ventajas y desventajas.
	\item En el capítulo cuatro se detalla la calibración utilizando los acoplamientos mútuos con sus desventajas y desventajas.
	\item En el capítulo cinco se detalla la modelización de la problemática a resolver. 
	\item En el capítulo seis se presentan los resultados obtenidos y los trabajos futuros 
	\item En el capítulo siete se presentan las conclusiones obtenidas.
\end{itemize}

\todo[inline]{Pablo, supongo que el capítulo que está debajo también lo saco no???}

\section{Presentación previa de este trabajo}

Hay dos grandes grupos de métodos de calibración a saber, externa e interna. Del primer grupo, están las de blancos 
puntuales y blancos distribuidos.

Calibración utilizando blancos puntuales: Los blancos puntuales son dispositivos construidos por el hombre, generalmente son
corner reflectors, transponders, generadores de tono o receptores. La característica de estos dispositivos es que no solo 
son muy chicos con respecto a la mínia área de observación de una antena y que el eco de la señal de dichos dispositivos 
posee mucha más intensidad que dicha área, sino que el reflejo es bien conocido. Por lo tanto, con su uso se puede deducir la potencia transmitida.

Calibración utilizando blancos distribuidos: Se refiere a la utilización de blancos naturales de largas áreas con 
propiedades de backscattering homogéneas. Se utiliza la supisición que estas propiedades son estables o que su variación es 
bien conocida.

Las distintas estrategias que resuelven la problemática de calibración interna se resumen a continuación. 

El primer método es el convencional, el cual calibra los lazos internos de transmisión y recepción de la antena de forma 
independiente. Para esto, se mide la señal recibida luego de ser transmitida por todos los caminos posibles de la RFDN
y se los compara con la potencia transmitida. Estas diferencias, tanto de fase como de potencia, determinan cuanto atenúa
y defasa cada camino de la antena. \todo{mejorar última oración}

Otro método utilizado es el llamado método REV, el cual calibra de a un módulo radiante a la vez utilizando una antena externa
al panel como receptora. El proceso de calibración es el siguiente, se transmite por un módulo radiante y se va variando la 
fase hasta obtener la máxima potencia recibida, luego se repite el proceso para todos los módulos. 

El método que se desarrolla en esta tesis es el llamado \emph{calibración utilizando los acoplamientos mútuos}. Este método calibra 
tanto los caminos de transmisión como recepción a la vez, para ello, se transmite en una polarización y se recibe en otra de 
a un módulo radiante a la vez. Una ventaja que tiene frente a los otros métodos realizados es que no solo calibra la antena 
en su totalidad, sino también que sirve para determinar si hay módulos radiantes quemados.

Previamente a las simulaciones se realizaron los cálculos matemáticos necesarios para demostrar que el método sirve para 
lo anteriormente realizado. A su vez se investigaron tanto las problematicas que poseen cada método de calibración como
las problemáticas de la modelización de los componentes que conforman una antena polarimétrica.

Para la simulación, se realizó un programa que modele una antena polarimétrica genérica. Con la posibilidad de configurar 
sus dimensiones, su geometría y que se puedan simular casos de fallas a los distintos componentes que la conforman.
conforman.

Para poder contrastar los resultados obtenidos con el método de calibración utilizando los acoplamientos mutuos, en el modelo 
de antena también se programó el método convencional y se corrieron los mismos set de pruebas.

\todo{poner resultados luego de simularlos}

