\chapter{Resultados y líneas futuras}

El presente capítulo sirve de sumario a los resultados obtenidos tras la finalización del proyecto. Se encuncian las 
conclusiones y se propone las diferentes lineas de ampliación del trabajo que se desprende del proyecto terminado.

\section{Resultados}

A continuación se listan los resultados obtenidos de los ensayos realizados.
\begin{itemize}
	\item El método de calibración por acoplamientos mútuos es mucho más robusto que el clásico ante fallas en el generador 
		de las señales a calibrar, tanto de ganancia como de fase.
	\item Comparando los desvíos de todos los set de datos, se aprecia una peor respuesta cuando se realizan apuntamientos 
		en dirección horizontal, esto es, en la dirección donde se encuentra la mayor cantidad de elementos radiantes.
	\item Por no abarcar todo el sistema de transmisión/recepción de la antena, el método de calibración clásico no solo no 
		puede calibrar correctamente, sino que también puede llevar el estado de la señal a uno que se aleja más al ideal.
	\item El método de calibración convencional tiene más puntos de falla que el de calibración por acoplamientos mútuos.
	\item La falta de ortogonalidad generados por desvíos en la configuración de los defasadores afecta principalmente a la 
		estimación de la ganancia de los elementos radiantes.
\end{itemize}

\section{Líneas futuras}

Este trabajo es la base para varias posibles líneas de trabajo futuras, a saber:
\begin{itemize}
	\item Para un ensayo más realista, se puede introducir la limitación de la potencia máxima y mínima que soporta el lazo 
		de recepción de una antena. El efecto de esta limitación es la de limitar la cantidad de lazos de calibración interna, 
		por ende, la cantidad de ecuaciones para calibrar la antena.
	\item Para el caso de destrucción de componentes, por ejemplo los TRMs, sobre el modelo de antena se pueden realizar ensayos 
		de las estrategias a tomar para corregir el diagrama de radiación configurando los atenuadores y defasadores cercanos al 
		elemento dañado.
	\item Se puede modificar el algoritmo del método de calibración por acoplamientos mútuos para la detección y medición de 
		la planitud de antena. Hay varias consideraciones a tener en cuenta, la primera es que el método no puede determinar 
		unívocamente el estado de la antena (por la naturaleza del mismo, hay dos posibles soluciones), se debe hace uso de algún 
		sistema externo para determinar cual de las dos posibles soluciones es la real. La segunda es que, para poder determinar
		la cantidad de acoplamientos que hay en el panel y para disminuir la incertidumbre en los mismos, se deben calibrar la 
		antena en todos sus modos a la vez.
\end{itemize}

\section{Conclusiones}

En la presente tesis se introdujeron los conceptos básicos de los tipos de antenas y en particular de las polarimétricas. De 
la dificultades presentes a la hora de calibrar las mismas y su necesidad de calibración.

A su vez, se presentaron los principales conjuntos de métodos de calibración, explicando cuales son los usos y 
ventajas/desventajas de cada uno. Para luego introducir como actúa el método de calibración interna clásico, mostrando sus 
virtudes y deficiencias. 

Se desarrolló un método alternativo que complementa muchas de las falencias del método previamente mencionado. Para ensayar 
y comparar el comportamiento de ambos métodos de calibración se implementó un modelo de antena que sea representativo en RF.
ensayó con distintos tipos de desvíos que puede sufrir una antena a lo largo de su vida útil.
