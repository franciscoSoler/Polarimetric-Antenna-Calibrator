\chapter{Resultados y líneas futuras}

El presente capítulo sirve de sumario a los resultados obtenidos tras la finalización del proyecto. Se encuncian las 
conclusiones y se propone las diferentes lineas de ampliación del trabajo que se desprende del proyecto terminado.

\section{Resultados}

A continuación se listan los resultados obtenidos de los ensayos realizados.
\begin{itemize}
	\item El método de calibración por acoplamientos mútuos es mucho más robusto que el clásico ante fallas en el generador 
		de las señales a calibrar, tanto de ganancia como de fase.
	\item Comparando los desvíos de todos los set de datos, se aprecia una peor respuesta cuando se realizan apuntamientos 
		en dirección horizontal, esto es, en la dirección donde se encuentra la mayor cantidad de elementos radiantes.
	\item Por no abarcar todo el sistema de transmisión/recepción de la antena, el método de calibración clásico no solo no 
		puede calibrar correctamente, sino que también puede llevar el estado de la señal a uno que se aleja más al ideal.
	\item El método de calibración convencional tiene más puntos de falla que el de calibración por acoplamientos mútuos.
	\item La falta de ortogonalidad generados por desvíos en la configuración de los defasadores afecta principalmente a la 
		estimación de la ganancia de los elementos radiantes.
\end{itemize}

\section{Líneas futuras}

Este trabajo es la base para varias posibles líneas de trabajo futuras, a saber:
\begin{itemize}
	\item Para un ensayo más realista, se puede introducir la limitación de la potencia máxima y mínima que soporta el lazo 
		de recepción de una antena. El efecto de esta limitación es la de limitar la cantidad de lazos de calibración interna, 
		por ende, la cantidad de ecuaciones para calibrar la antena.
	\item Para el caso de destrucción de componentes, por ejemplo los TRMs, sobre el modelo de antena se pueden realizar ensayos 
		de las estrategias a tomar para corregir el diagrama de radiación configurando los atenuadores y defasadores cercanos al 
		elemento dañado.
	\item Se puede modificar el algoritmo del método de calibración por acoplamientos mútuos para la detección y medición de 
		la planitud de antena. Hay varias consideraciones a tener en cuenta, la primera es que el método no puede determinar 
		unívocamente el estado de la antena (por la naturaleza del mismo, hay dos posibles soluciones), se debe hace uso de algún 
		sistema externo para determinar cual de las dos posibles soluciones es la real. La segunda es que, para poder determinar
		la cantidad de acoplamientos que hay en el panel y para disminuir la incertidumbre en los mismos, se deben calibrar la 
		antena en todos sus modos a la vez.
\end{itemize}

\section{Conclusiones}

En la presente tesis se introdujeron los conceptos básicos de un conjunto de antena de fase variable polarimétrica. La cual 
presenta la versatilidad de no solo poder modificar el apuntamiento electrónicamente, por efecto de construcción e interferencia 
electromagnétca, sino que transmite y recibe en dos tipos de polarizaciones a saber H y V.

Para lograr un mejor rendimiento del sistema completo, es extremadamente necesario poder conocer y mantener controlada la fase 
y atenuación de Tx/Rx. Esto se refleja en el diagrama de radiación, donde, los parámetros a tener controlados son la ganancia
del lóbulo principal, la ganancia relativa de los lóbulos secundarios al lóbulo principal y el ancho del lóbulo principal a 
-3dB. En caso de deformaciones en dicho diagrama, la antena puede llegar a estar apuntando en una dirección no deseada y 
generalmente disminuye drásticamente la calidad del producto final de la aplicación para la cual pudo haberse utilizado dicha 
antena.

Para poder mantener controlada la ganancia de Tx/Rx, se investigaron distintos métodos de calibración, externa e interna. La 
calibración externa presenta el problema de, al utilizar blancos en tierra (transponders o corner reflectors), el tiempo entre 
calibraciones es muy grande. Por este motivo se hace uso de la calibración interna, con el cual se logra mantener controlada 
los parámetros individuales de cada elemento del conjunto de antena. 

Como método de calibración interna, se presentó el modelo clásico, el cual determina si un módulo TR funciona correctamente, 
pero adolece de no poder calibrar el sistema en su totalidad por no abarcarlo completamente, que depende 
fuertemente del modelo térmico utilizado por la utilización de caracterizaciones de los componentes que no pertenecen a los 
lazos de calibración, que la complejidad de la electrónica aumenta por el agregado de hardware dedicado y que se requiere 
mantener la temperatura de los componentes dentro del rango de las temperaturas caracterizadas.

Se desarrolló un modelo de calibración convencional que abarca todo el sistema, evitando así el gasto de recursos en las 
campañas de caracterización, en tener que controlar tanto la temperatura de los componentes, en tener que agregar hardware 
dedicado de calibración. Pero requiere que se pueda transmitir y recibir de a pares de elementos a la vez. 

A la hora del modelado de los métodos de calibración y de la antena, se requiere que el comportamiento de cada componente 
individual de la antena sea modelado en RF, para esto, se requiere la utilización de parámetros S. El modelo de antena 
desarrollado es muy versatil para ensayar el comportamiento de cualquier conjunto de antena de distribución rectangular, 
pudiendose modificar la electrónica asociada a un mismo panel para determinar cual es la mejor estructura ante posibles fallas.
A su vez, se lo puede reutilizar para ensayar otros métodos de calibración.

Por último, al realizar comparaciones de ambos métodos de calibración con desvíos de distintas naturalezas (desvíos de 
comportamiento de distintos componentes, destrucción de componentes, inestabilidad en el generador, incertidumbre de medición 
del receptor, etc), se llegó a la conclusión que el método alternativo es más robusto que el clásico, por ende, se lo 
debería utilizar como complemento del clásico.

