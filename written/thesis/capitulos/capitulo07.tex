\chapter{Resultados y líneas futuras}

El presente capítulo sirve de resumen de los resultados obtenidos tras la finalización del proyecto. Se enuncian las 
conclusiones y se proponen diferentes lineas de ampliación del trabajo.

\section{Resultados}

A continuación se enuncian los resultados del trabajo realizado en la presente tesis. 
\begin{enumerate}
	\item Se investigó el método de calibración interna clásico para entender el modo de operación y detectar sus falencias.
	\item Se investigó un nuevo método de calibración, por acoplamientos mutuos, que complemente el previamente mencionado y se
		propusieron las hipótesis necesarias para implementarlo. Estas hipótesis impactan directamente sobre el diseño de la
		antena. Por dar un ejemplo, una de estas hipótesis es la capacidad de poder transmitir y recibir al mismo tiempo.
	\item Se investigaron los modelos matemáticos para representar los componentes de la antena en RF, cumpliendo las hipótesis
		pre-establecidas.
	\item Se desarrolló una aplicación que simula un modelo de antena, incluyendo la modelización de los acoplamientos
		mutuos. Dicha aplicación posee la cualidad de poder definir la estructura interna de forma tal que permita realizar 
		análisis de diferentes topologías de antena.
	\item Se desarrolló el modelo de calibración interna clásica. El mismo calibra el modelo de antena previamente mencionado.
	\item Se desarrolló un modelo de calibración avanzado, por acoplamientos mutuos. El mismo calibra el modelo de antena
		previamente mencionado. 
	\item Se propusieron e implementaron casos de prueba ad hoc para compararlos.
	\item Se explicó que el método de calibración por acoplamientos mutuos resulta particularmente de interés en sistemas de
		vuelo, donde las calibraciones externas no tienen la frecuencia necesaria, y en donde el agregado de elementos
	adicionales para realizar dicha calibración representa costos mayores.
\end{enumerate}

A continuación se listan los resultados obtenidos de los ensayos realizados para comparar los comportamientos de los dos métodos
de calibración interna estudiados, clásico y por acoplamientos mutuos. 
\begin{itemize}
	\item Para el caso de destrucción de TRMs, ambos calibradores poseen el mismo desempeño. Este comportamiento no afecta los
		resultados obtenidos de las calibraciones del resto de los componentes.
		
	\item El método de calibración por acoplamientos mutuos es mucho más robusto que el clásico ante fallas en el generador de
		señales, ubicado dentro de la unidad central de control UCC, tanto en ganancia como en fase. Para la topología de antena
		escogida, se obtiene una mejora de 8 veces en la determinación de la ganancia y de 16 veces en la determinación de la
		fase.
	\item Comparando los desvíos del conjunto de datos, se aprecia una peor respuesta cuando se realizan apuntamientos 
		en dirección horizontal, esto es, en la dirección donde se encuentra la mayor cantidad de elementos radiantes. Esto se
		debe a que hay menos elementos por panel, los cuales deben tener la misma fase entre ellos, haciendo que el promedio de
		dichas fases posea un mayor desvío que en la otra dirección. Este caso se puede mejorar utilizando, por ejemplo una
		topología con una mayor cantidad de ER.
	\item Por no abarcar la totalidad del sistema de transmisión/recepción de la antena, el método de calibración interna clásico
		no puede determinar correctamente los niveles de ganancia y fase transmitidos/recibidos. De esta forma, al realizar la
		corrección de dichas magnitudes, se puede llevar el estado de la señal a uno más alejado al deseado. En la sección
		\ref{ssc:classicalDispersion} se observa dicho comportamiento. 
	\item El método de calibración interna clásico tiene más puntos de falla que el de calibración por acoplamientos mutuos. Por
		ejemplo, si los circuladores o los ERs resultan dañados el método de calibración clásico no puede detectar dichas fallas.
	\item La falta de ortogonalidad generada por desvíos en la configuración de los desfasadores afecta principalmente a la 
		estimación de la ganancia de los elementos radiantes al correr el método de calibración interna clásica. Generando así
		que el diagrama de radiación resulte deformado al realizar la \enquote*{corrección} de dichos desvíos.

	\item En la topologia probada se observa que las incertidumbres individuales de la señal de transmisión son ampliamente
		disminuidas al ser transferidas a la ganancia de lóbulo principal del diagrama de radiación, esto ocurre por
		promediarse dichas señales al utilizar antenas con un gran número de elementos radiantes. En la sección
		\ref{sc:relationDispersionRadiationPattern} se ensayaron dos conjuntos de antena, la primera de 5 elementos radiantes y
		la segunda de 70 obteniendo como resultado que la segunda posee una incertidumbre cuatro veces menor.
	\item En el caso de dispersiones de fase de los códigos de walsh y de ganancia del chirp réplica, no se aprecian incrementos
		de dispersiones al realizar distintos apuntamientos con la antena.
	\item La incertidumbre asociada en las mediciones con el método de calibración interna por acoplamientos mutuos se ve
		disminuída por promediar los resultados individuales de las ganancias y fases transmitidas y recibidas por la utilización
		de lazos de calibración que comparten los caminos de transmisión y recepción de la antena.
	\item Comparando la cantidad de pulsos de calibración mínima para calibrar toda la antena para cada método de calibración
		interna, se llega a la conclusión que el método parcial o de planitud ideal por acoplamientos mutuos, si bien es del
		mismo orden que el de calibración interna clásica, $O(n)$, siendo $n$ la totalidad de elementos radiantes, requiere menos
		cantidad de pulsos. En cambio, el modo completo por acoplamientos mutuos es de mayor orden, $O(n^2)$, requiriendo así una
		mayor cantidad de pulsos para calibrar la antena completa, ver secciones \ref{ssc:classicalMethod} y
		\ref{ssc:operationalModes}. 
\end{itemize}


\section{Líneas futuras}

Este trabajo es la base para varias posibles líneas futuras de trabajo, a saber:
\begin{itemize}
	\item Para un ensayo más realista, se puede introducir la limitación de la potencia máxima y mínima que soporta el lazo 
		de recepción de una antena. El efecto de esta limitación es la de disminuir la cantidad de lazos de calibración interna, 
		por ende, la cantidad de ecuaciones para calibrar la antena.
	\item Para una modelización más realista, se puede introducir que el cambio de fase de los desfasadores es de a pasos discretos
		en vez de continuos.
	\item Para el caso de destrucción de componentes, por ejemplo los TRMs, sobre el modelo de antena se pueden realizar ensayos 
		de las estrategias a tomar para corregir el diagrama de radiación configurando los atenuadores y desfasadores cercanos al 
		elemento dañado.
	\item Se puede modificar el algoritmo del método de calibración por acoplamientos mutuos para la detección y medición de 
		la planitud de antena. Hay varias consideraciones a tener en cuenta, la primera es que el método no puede determinar 
		unívocamente el estado de planitud la antena (hay dos posibles soluciones por la naturaleza del mismo), se debe hacer uso de 
		algún sistema externo para determinar cual de las dos posibles soluciones es la real. La segunda es que, para poder determinar
		la cantidad de acoplamientos que hay en el panel y para disminuir la incertidumbre en los mismos, se deben calibrar la antena 
		en todos sus modos a la vez.
	\item Se puede realizar un ensayo modificando la distribución de la RFDN para determinar que configuración es la más robusta ante 
		las dispersiones del comportamiento de dichos componentes.
	\item Se puede ensayar el desempeño del algoritmo de calibración ante desadaptaciones de impedancias. 
	\item Se puede hacer un análisis que conste de transmitir con más de un ER a la vez para que la potencia transmitida sea mayor
		al armar un lazo de calibración, de forma tal de incrementar la cantidad de ecuaciones disminuyendo la cantidad de
		elementos que pertenecen a la zona de acoplamiento débil. 
\end{itemize}

\section{Conclusiones}

En la presente tesis se introdujeron los conceptos básicos de un conjunto de antena de fase variable polarimétrica. La cual 
presenta la versatilidad de poder transmitir y recibir en las polarizaciones H y V de forma independiente, a su vez permite
modificar el apuntamiento electrónicamente, por efecto de construcción e interferencia electromagnética.

Para lograr un mejor rendimiento del sistema completo, es necesario conocer y mantener controlada la fase y atenuación
transmitida y recibida por cada elemento radiante. Dichos parámetros se reflejan en la forma del diagrama de radiación,
particularmente en la ganancia relativa de los lóbulos secundarios con respecto al principal y en la ganancia y ancho (a -3 dB)
de dicho lóbulo. En caso de haber deformaciones en el diagrama, no solo puede haber un apuntamiento indeseado, sino que se
disminuye drásticamente la calidad del producto final de la aplicación para la cual pudo haberse utilizado dicha antena.

Los dos grupos principales de calibraciones que se utilizan para mantener controlados los desvíos de ganancias de transmisión y
recepción son el externo e interno. Se estudia y propone un método de calibración interna por sobre la externa debido a la
necesidad de mantener controlados los parámetros individuales de cada elemento del conjunto de antena. El segundo grupo presenta
la desventaja de no poder cumplir con dicho requerimiento dado que se determina la ganancia y fase del conjunto en su totalidad. 

Uno de los métodos de calibración interna presentados es el clásico, el cual determina si un TRM funciona correctamente. 
Sus principales desventajas son que no se puede calibrar el sistema en su totalidad por no abarcarlo completamente, que depende
fuertemente del modelo térmico utilizado por la utilización de caracterizaciones de los componentes que no pertenecen a
los lazos de calibración, que la complejidad de la electrónica se incrementa por el agregado de hardware dedicado y que se
requiere mantener la temperatura de los componentes dentro del rango de las temperaturas caracterizadas.

Otro método de calibración interna presentado y mejorado es el método por acoplamientos mutuos, el cual abarca toda la
antena, logrando así calibrar y de esta forma controlar la ganancia transmitida y recibida por cada elemento radiante. A su
vez es complementario al método de calibración interna clásica. Otras ventajas que posee son que, al abarcar todo el sistema de
antena, no es necesario el gasto de recursos en las campañas de caracterización, en tener que controlar tanto la temperatura de
los componentes, en tener que agregar hardware dedicado de calibración. Aunque, requiere que se pueda transmitir y recibir de a
pares de elementos radiantes a la vez en distintas polarizaciones. 

Hay que tener extremo cuidado en la determinación de las distintas zonas de acoplamientos para evitar configurar como receptores
los TRMs pertenecientes a la zonas de acoplamiento destructivo y débil. En el primer caso los mismos resultarían ser dañados por
recibir una potencia mayor a la permitida en su especificación. En el segundo caso el lazo de calibración no es significante dado
que la potencia recibida es menor a la del piso del ruido. 

A la hora del modelado de los métodos de calibración y de la antena, se requiere que el comportamiento de cada componente 
individual de la antena sea modelado en RF, para esto, se requiere la utilización de parámetros S. El modelo de antena 
desarrollado es muy versátil para ensayar el comportamiento de cualquier conjunto de antena de distribución rectangular, 
pudiéndose modificar la electrónica asociada a un mismo panel para determinar la mejor distribución ante posibles fallas.
Tambien, se lo puede reutilizar para ensayar otros métodos de calibración.

Se utilizó el modelo de antena para determinar si los desvíos individuales de ganancia y fase de cada señal transmitida que 
componen la antena afectan de igual magnitud a la ganancia del lóbulo principal del diagrama de radiación y se llegó a la 
conclusión que a medida que se utiliza una antena con mayor cantidad de elementos radiantes, es menor la incertidumbre en este 
parámetro. Por ejemplo, el desvío estándar de la ganancia del diagrama de radiación resulta 10 veces menor que el desvío de 
ganancia de cada ER para una antena de 70 elementos radiantes.

Por último, al realizar comparaciones de ambos métodos de calibración con desvíos de diferentes naturalezas (por ejemplo desvíos
en el comportamiento de distintos componentes, destrucción de componentes, inestabilidad en el generador, incertidumbre de medición 
del receptor), se llegó a la conclusión que el método de calibración por acoplamientos mutuos es más robusto que el método de
calibración interna clásico, por ende, se lo debería utilizar como complemento del clásico.

