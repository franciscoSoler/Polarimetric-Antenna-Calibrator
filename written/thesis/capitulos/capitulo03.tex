%Capítulo 3

\chapter{Calibración clásica}
\label{ch:classicalCalibration}
\lhead{\emph{Calibración clásica}}
%----------------------------------------------------------------------------------------

\section{Calibración clásica}

Como fue introducido previamente, una antena consta de un generador, una RFDN y el panel de módulos radiantes. Como puede
variar el comportamiento de cada componente, se utiliza la calibración. En particular, este método se lo utiliza para poder 
medir, detectar y corregir el mal funcionamiento de parte de la RFDN, en particular todas las desviaciones se las atribuyen 
los módulos de Transmisión/Recepción. A su vez, también se puede detectar si uno de dichos módulos queda inhabilitado a 
causa que cierta parte de la cadena de transmisión/recepción o dicho componente se destruyó.

Como limitación, esta calibración interna no puede medir las partes pasivas del sistema que están fuera del lazo de 
calibración interna (por ejemplo los módulos radiantes), ni la ganancia absoluta, debido a la ausencia de un target standard
de calibración \cite{Wang2010}. 

En la figura \ref{fig:classic_cal_scheme} se muestra el esquema de calibración, en el cual se observan tres modos de 
calibración. Cada uno posee distintos caminos, \textbf{P1} (líneas rojas) caracteriza el camino de transmisión, \textbf{P2}
(línea azul) caracteriza el camino de recepción y \textbf{P3} la electrónica central (CE) junto a los puertos auxiliares de 
transmisión/recepción. \textbf{P3} es utilizado para corregir posibles variaciones en los pulsos \textbf{P1} y \textbf{P2} 
\cite{Makhoul2012}.

\begin{figure}[H]
 \centering
 \includegraphics[width=10cm]{gfx/classic_cal_scheme.png}
 \caption{Esquema de calibración interna: camino de calibración de pulsos \textbf{P1} (Tx) en rojo, \textbf{P2} (Rx) en azul 
 y \textbf{P3} (electrónica central) en verde \cite{Makhoul2012}.}
 \label{fig:classic_cal_scheme}
\end{figure}

Para dar ejemplos de satélites que utilizaron el método se los puede nombrar al E-ERS-1, el SIR-C \cite{Curlander1991}, el 
terraSAR-X \cite{Schwerdt2005}, el ENVISAT ASAR \cite{Loop}, entre otros.

\subsection{Método} \label{ssc:classicalMethod}

A parte de la medición de la estabilidad del instrumento, es necesario obtener el funcionamiento de los TRMs de forma 
individual. Como el apuntamiento y la planitud de la señal dependen de la configuración de estos componentes, es necesario 
conocer su estado actual de funcionamiento. Comparando con datos de telemetría (ejemplo temperaturas y tensiones de los 
TRMs) junto a apropiada caracterización en tierra solo provee información limitada del comportamiento del radar \cite{Br2007}.

Una estrategia para mediciones individuales de cada TRM requiere que el resto esté apagado. El problema de esta estrategia 
es que, al tener parte de la antena apagada, no resulta un método representativo al modo de funcionamiento nominal (toda la 
antena operativa). Para solventar esta problemática y calibrar todos los TRMs a la vez, al menos en transmisión o recepción,
se hace uso de los defasadores para armar un código en que las señales sean ortogonales. En particular, por ejemplo, una de
las usos es defasar cada TRM en $\pm90^{\circ}$ siguiendo una determinada secuencia $c_{mn}(t)$ por cada uno.

\begin{figure}
 \centering
 \includegraphics[width=8cm]{gfx/superposition_signals_classic.png}
 \caption{Superposición de señales de todos los TRMs. Cada señal tiene su propia secuencia de código aplicada entre pulsos \cite{Br2007}.}
 \label{fig:sup_sign_classic}
\end{figure}

Por lo tanto, la fase de salida de cada TRM es la fase configurada $\varphi_{mn}$ sumada al defasaje del código de 
$90^{\circ}$. Consecuentemente, la superposición de todas las ganancias de los TRMs, $a_{mn}$, y fases, $\varphi_{mn}$,
es obtenida en el puerto de recepción la la RFDN, $s_c(t)$ como se muestra en la figura \ref{fig:sup_sign_classic}.

\begin{equation}
	s_c(t) = \sum_{m=0}^{M-1}\sum_{n=0}^{N-1}c_{mn}\cdot a_{mn}e^{j\varphi{mn}} + n_{mn}
\end{equation}

Donde $n_{mn}$ es el ruido inherente que hay en las mediciones de cada TRM. Para decodificar y obtener la ganancia 
$\tilde{a}_{mn}$ y fase estimada $\tilde\varphi_{mn}$ de algún TRM, la señal compuesta $s_c$ es correlacionada con 
la secuencia del módulo deseado. Con esta correlación la modulación de la secuencia se elimina dando como resultado
la ganancia estimada.

\begin{equation}
\begin{aligned}
	\tilde{x}_{mn} &= s_c \otimes c^*_{mn} \\
	\tilde{x}_{mn} = \int s_c(t) &\cdot c^*_{mn}(t) dt = \tilde{a}_{mn}e^{j\tilde{\varphi}_{mn}} \\
\end{aligned}
\label{eq:classic_correlation}
\end{equation}

En general el código de calibración utilizado es el código walsh. Dicho código deriva de las matrices de Hadamard (ver 
apéndice \ref{AppendixB}); dada sus propiedades de ortogonalidad, cada código, o fila, es unívocamente distinguible del 
resto. Para minimizar la cantidad de mediciones, el largo del código ($l$) debe ser lo más corto positble. El número de 
TRMs de la antena es el determinante de la cota inferior.

\begin{equation}
	l = 2^i \ge N \cdot M
\end{equation}

Siendo $N$ la cantidad de filas y $M$ la cantidad de paneles, o columnas, que tiene el array de antena. No es necesario que 
se calibren todos los TRMs de una. Hay tres estrategias que se utilizan con este método de calibración para obtener 
distintos niveles de granularidad de mediciones a saber.

\begin{itemize}
	\item \textbf{Nivel módulo:} Este nivel es el que utiliza los códigos más largos, dado que se calibran todos los módulos que 
		posee la antena en una polarización determinada ($l = 2^i \ge N \cdot M$).
	\item \textbf{Nivel panel:} En este nivel se utiliza el mismo código para todos los TRMs que son de un mismo panel, 
		logrando así, decrecer el largo del código ($l = 2^i \ge M$).
	\item \textbf{Nivel fila:} En este nivel se utiliza el mismo código para todos los TRMs que son de una misma fila, 
		logrando así, decrecer el largo del código ($l = 2^i \ge N$).
\end{itemize}

Este método a nivel panel y fila sirve para la caracterización del la configuración del apuntamiento de antena \cite{Br2007}.

\subsection{Problemas y limitaciones}

\begin{itemize}
	\item Caracterización prefvia de los elementos de antena: Para poder conocer la ganancia del lazo de transmisión o recepción
es necesario conocer la potencia de la señal de calibración, para sustraerla del resultado obtenido. El lazo de calibración 
donde se realiza esta medición es el que se muestra en la imagen \ref{fig:classic_cal_scheme}, llamado \textbf{P3}. En esta 
medición no se puede separar la potencia de generación de la atenuación del lazo, por lo tanto se opta por caracterizar en 
tierra dichos componentes para las frecuencias y temperaturas de trabajo. Esta estrategia no es completamente correcta dado
que, por envejecimiento de los materiales, el comportamiento es diferente.
\begin{itemize}
	\item bla
	\end{itemize}
\end{itemize}

Como se calibra solo una parte de la antena por vez, transmisión o recepción en una u otra polarización (H o V), es necesario que 
el lazo de calibración esté compuesto por la parte de la antena a calibrar junto a hardware dedicado (cables y switches) a esta 
tarea. Logrando así, no solo que la construcción de la antena sea más compleja y que se tengan que caracterizar más componentes, 
sino también que el defasaje y atenuación que este hardware dedicado posee se lo atribuye a los TRMs agregando así más 
error en la medición. 

Este método es sumamente susceptible a las variaciones de fase y potencia del generador entre pulsos de calibración. Por lo 
tanto, es de vital importancia armar un generador que sea sumamente estable.

Como hay componentes que están fuera del lazo de calibración, también se los caracteriza en tierra, por ejemplo los cables 
que comunican el panel de módulos radiantes con los circuladores (como ejemplo ver figura \ref{fig:classic_cal_scheme}), 
otra caracterización que se realiza es la del acoplamiento mutuo entre los RMs de la antena. Todas estas caracterizaciones
se las realizan en una fase llamada campaña de calibración, y tiene la gran desventaja del tiempo y costo que lleva realizar 
esta tarea.

A la hora de elegir la longitud del código Walsh, es importante que siempre haya un módulo radiante virtual en la antena para
evitar la primer columna de la matriz de códigos. En caso contrario el primer TRM siempre tendrá un error en la estimación 
de su ganancia, para mayor información ver \cite{Wang2010}.

\todo[inline]{este parrafo creo que hay uqe sacarlo dado que me parece que esto es para todas las calibracoines internas}
Como limitación, esta calibración interna no puede medir las partes pasivas del sistema que están fuera del loop de 
calibración interna, ni la ganancia absoluta, debido a la ausencia de un target standard de calibración \cite{Wang2010}.

Como última limitación, el defasador cambia la fase de a pasos discretos, si este componente no llega a defasar en $\pm$90º
por no ser múltiplo de dicho paso (defasaje utilizado para realizar el código para la calibración) o por variaciones en el 
comportamiento del componente.


