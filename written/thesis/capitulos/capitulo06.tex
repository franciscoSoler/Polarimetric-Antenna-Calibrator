\chapter{Simulaciones}
\lhead{\emph{Simulaciones}}

En este capítulo se presentan las mediciones y ensayos realizados para estudiar y comparar como se comportan ambos métodos de
calibración, clásica y utilizando acoplamientos mútuos.   

\section{Sin errores}
\section{Rotura de un TRM}
\section{Errores de comportamiento de elementos}

En esta simulación se agregan errores en el comportamiento de siguientes componentes: circuladores, TRMs, PSCs y RMs. Se adopta
un error con un desvío standard de 1 para cada uno de ellos.

\section{Errores de ganancia entre pulsos}
\section{Errores de fase entre pulsos}
\section{Error de ganancia de la chirp réplica}
\section{Error de fase de la chirp réplica}
\section{Error de fase del Walsh}

Como los códigos walsh son solamente utilizados en la calibración clásica, en esta simulación no se podrán comparar 
resultados, pero sirve para determinar que tan robusto es el método para esta clase de erorres.
