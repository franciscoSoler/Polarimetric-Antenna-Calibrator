\chapter{Simulaciones}
\lhead{\emph{Simulaciones}}

En este capítulo se presentan las mediciones y ensayos realizados para estudiar y comparar como se comportan ambos métodos de
calibración, clásica y utilizando acoplamientos mútuos.

La configuración de la antena a utilizar para todos los ensayos está especificada en la tabla \ref{tab:configurationUsed} y 
consta de 70 elementos radiantes. Las especificaciones de las propiedades físicas de cada componente están definidas en la 
tabla \ref{tab:configurationOfComponents}.
\begin{table}[H]
  \footnotesize
  \centering
  \begin{tabular}{|c|c|}
	\hline
	\textbf{Componente de Antena} & \textbf{Configuración} \tabularnewline \hline 
	freq &  1275000000 [Hz] \tabularnewline\hline 
	power & 0 [dB] \tabularnewline \hline 
	phase & 0 [deg] \tabularnewline \hline 
	desiredTxPower & 20 [dB] \tabularnewline \hline 
	desiredRxPower & 0 [dB] \tabularnewline \hline 
	quantityRows & 7 \tabularnewline \hline 
	quantityColumns & 10 \tabularnewline \hline 
	verticalSeparation & 0.2 [m] \tabularnewline \hline 
	hotizontalSeparation & 0.2 [m] \tabularnewline \hline 
	componentSequence & [cable1, psc17, cable1, psc15, cable2, psc12, trm, circulator, cable3, rm] \tabularnewline \hline 
  \end{tabular}
  \caption{Configuración de la antena común para todos los ensayos.}
  \label{tab:configurationUsed}
\end{table}
\begin{table}[H]
  \footnotesize
  \centering
  \begin{tabular}{|c|c|c|}
	\hline
	\textbf{Componente de Antena} & \textbf{Cacarcterísticas físicas} & \textbf{Configuración} \tabularnewline \hline 
	\multirow{2}{*}{cable1} &  attenuation [db] & 0.1\tabularnewline \cline{2-3}
	 & length [m] & 0.45\tabularnewline \hline 
	\multirow{2}{*}{cable2} &  attenuation [db] & 0.1\tabularnewline \cline{2-3}
	 & length [m] & 8\tabularnewline \hline 
	\multirow{2}{*}{cable3} &  attenuation [db] & 0.1\tabularnewline \cline{2-3}
	 & length [m] & 0.5\tabularnewline \hline 
	psc17 & outputPorts & 7\tabularnewline \hline
	psc15 & outputPorts & 5\tabularnewline \hline
	psc12 & outputPorts & 2\tabularnewline \hline
	\multirow{2}{*}{TRM} & gain [db] & 10\tabularnewline \cline{2-3}
	 & phaseShift [deg] & 10\tabularnewline \hline 
	circulator & & \tabularnewline \hline 
	RM & & \tabularnewline \hline 
  \end{tabular}
  \caption{Configuración de las propiedades físicas de cada componente de la antena utilizada en todos los ensayos.}
  \label{tab:configurationOfComponents}
\end{table}
La numeración de los elementos tiene un orden de izquierda a derecha y de forma descendiente, de tal forma que el elemnto cero 
es el superior izquierdo de la antena.

\section{Sin errores}

Los ensayos de esta sección tienen las siguientes caracterísiticas,
\begin{itemize}
	\item Ensayar ambos métodos de calibración con diferentes apuntamientos de la antena.
	\item Los distintos apuntamientos utilizados son: Uniforme, 10 grados en la dirección horizontal y 10 grados en la dirección 
		vertical.
	\item Los componentes de antena no tienen ningún prolema de funcionamiento ni hay desadaptaciones.
	\item No hay errores de calibración.
	\item No hay errores en la señal.
\end{itemize}

\subsection{Utilizando la calibración clásica}
\todo{:D}
\subsection{Utilizando la calibración con acoplamientos mútuos}

En los siguientes gráficos se puede observar que el método funciona correctamente, en la figura A, la fase se mantiene i
constante e igual a cero para todos los elementos radiantes. En la figura b se observa que 

\begin{figure}[H]
	\centering
 	\subfloat[]{
		\includegraphics[width=10cm]{gfx/nonErrMutual0deg.png}}
 	
	\subfloat[]{
		\includegraphics[width=8cm]{gfx/nonErrMutual10degCol.png}}
	\subfloat[]{
		\includegraphics[width=8cm]{gfx/nonErrMutual10degRow.png}}
		\caption{Antena calibrada con distintos apuntamientos utilizando acoplamientos mútuos. (a) 0 grados. (b) 10 grados en 
		dirección horizontal. (c) 10 grados en dirección vertical.}
	\label{fig:nonErrorMutual}
\end{figure}


\begin{figure}
 \centering
 \includegraphics[width=10cm]{gfx/panelBase.png}
 \caption{Flujo de ejecuci\'on del modelo de antena.}
\end{figure}

\begin{figure}
 \centering
 \includegraphics[width=10cm]{gfx/panelBase.png}
 \caption{Flujo de ejecuci\'on del modelo de antena.}
\end{figure}



\section{Rotura de un TRM}
\section{Errores de comportamiento de elementos}

En esta simulación se agregan errores en el comportamiento de siguientes componentes: circuladores, TRMs, PSCs y RMs. Se adopta
un error con un desvío standard de 1 para cada uno de ellos.

\section{Errores de ganancia entre pulsos}
\section{Errores de fase entre pulsos}
\section{Error de ganancia de la chirp réplica}
\section{Error de fase de la chirp réplica}
\section{Error de fase del Walsh}

Como los códigos walsh son solamente utilizados en la calibración clásica, en esta simulación no se podrán comparar 
resultados, pero sirve para determinar que tan robusto es el método para esta clase de erorres.
