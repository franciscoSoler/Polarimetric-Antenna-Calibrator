\chapter{Modelización}
\label{ch:modelizacion}
\lhead{\emph{Modelización}}

\todo[inline]{ver donde poner esto}
\section{Modelización de los acoplamientos mútuos}

El acoplamiento mútuo entre módulos radiantes es modelizado con el mismo comportamiento al de un cable, pero con las 
propiedades de atenuación y defasaje de una onda en el vacío. Las cuales son:

\todo{Poner que significa el alfa}
\begin{equation}
	c = \dfrac{e^{-2\alpha r}}{4\pi r^2}
\end{equation}


\section{Modelo de la antena}

El modelo de antena está programado de tal forma que se lo pueda utilizar para simular una antena con parámetros de RF,
la misma está conformada por los siguientes componentes: cables, psc, trms, defasadores y elementos radiantes. Como es 
requerido calibrar con dos métodos de calibración una misma antena en un estado determinado, para poder comparar los resultados
obtenidos, se opta por guardar en archivos tanto la configuración como el comportamiento de la misma. 

La modelizaci\'on de los componentes en RF es utilizando los par\'ametros S. En el apéndice \ref{AppendixC} se muestra un 
ejemplo de estos archivos para una antena de dos módulos radiantes.

Como es necesario realizar ensayos de montecarlo como una comparación de resultados obtenidos utilizando disitntos tipos de 
calibraciones ante una misma antena con una misma configuración y comportamiento. Se optó por bajar a disco dicha información
utilizando archivos json. En los cuales se guarda la configuración y propiedades físicas tanto del panel como de la RFDN. Las
propiedades del primer grupo son simplemente la distancia de todos a todos los elementos y las del segundo grupo, corresponden
a la matriz de parámetros S y que componente tienen conectado en los conectores de salida. En el anexo \ref{AppendixC} se puede
observar un ejemplo de estos archivos para una antena con dos elementos radiantes.

Dado lo mencionado previamente, el simulador termina siendo de dos etapas, la primera parte es la encargada de la modelización
de la antena y sus componentes; y la segunda, de calibrar dicho modelo. La figura \ref{fig:prog_inic} muestra de forma 
simplificada el flujo de ejecuci\'on del programa.

\begin{figure}
 \centering
 \includegraphics[width=10cm]{gfx/FlujoEjecucion.png}
 \caption{Flujo de ejecuci\'on del modelo de antena.}
 \label{fig:prog_inic}
\end{figure}


\subsection{Generador de Antena}

En principio, una antena necesita que todos los caminos entre el punto donde se inyecta o recibe la se\~nal y los m\'odulos 
radiantes sean iguales. Esta caracter\'istica es similar a la de un \'arbol balanceado. Por lo tanto, para definir la 
estructura interna de la antena, se definen simplmenete los elementos que conforman dicho \'arbol y, para determinar el orden 
de armado de la antena, se utiliza una lista. El orden es descendiente, si se para en un elemento de la lista, los de la 
izquierda son ascendientes y los de la derecha son descendientes. En la figura \ref{fig:2RMAntenna} se muestra la antena que 
se construye teniendo como configuración: [\enquote*{cable1}, \enquote*{psc12}, \enquote*{cable2}, \enquote*{trm}, 
\enquote*{circulator}, \enquote*{cable3}, \enquote*{rm}].

\begin{figure}
 \centering
 \includegraphics[width=10cm]{gfx/RFDN.png}
 \caption{Estructura interna de una de las polarizaciones de una antena con dos elementos radiantes.}
 \label{fig:2RMAntenna}
\end{figure}

Es necesario definir cuales son las características físicas de cada uno de los componentes mencionados en la lista anterior. 
La tabla \ref{tab:propertiesOfComponents} determina las propiedades de cada elemento contenido dentro de una antena. 

\begin{table}[H]
  \footnotesize
  \centering
  \begin{tabular}{|c|c|}
	\hline
	\textbf{Componente de Antena} & \textbf{Cacarcterísticas físicas} \tabularnewline \hline 
	\multirow{2}{*}{cable} &  attenuation [db] \tabularnewline \cline{2-2}
	 & length [m] \tabularnewline \hline 
	\multirow{2}{*}{TRM} & gain [db]\tabularnewline \cline{2-2}
	 & phaseShift [deg] \tabularnewline \hline 
	PSC1$j$ & outputPorts = $j$ \tabularnewline \hline 
	circulator & \tabularnewline \hline 
	RM & \tabularnewline \hline 
  \end{tabular}
  \caption{Propiedades físicas de cada componente de una antena}
  \label{tab:propertiesOfComponents}
\end{table}

Como es necesario que la representación de las propiedades mencionadas previamente sean representativas en radio frecuencia,
se utilizan los parámetros S (como se menciona en el capítulo \todo[inline]{buscar capitulo}). En la figura 
\ref{fig:creationPackage} se muestra el diagrama de clases donde se puede observar que el generador hace uso de los distintos 
generadores de los distintos componentes de scattering parameters.

\begin{figure}
 \centering
 \includegraphics[width=15cm]{gfx/creationPackage.png}
 \caption{Diagrama de clases del generador de antena.}
 \label{fig:creationPackage}
\end{figure}

Con estos componentes, si bien está definida la cantidad de elementos radiantes, falta definir las dimensiones de la antena, 
en otras palabras, la cantidad de paneles, la cantidad de elementos por panel y la separación, tanto horizontal como vertical,
entre elementos. Los primeros se traducen a cantidad de columnas y filas de RMs. 

Como dicha información es medianamente redundante y posiblemente incompatible con la definición de la estructura interna de 
la antena, es necesario un chequeo de compatibilidad. La figura \ref{fig:frontAntenna} muestra el frente de una antena que se 
construye teniendo como configuración: "quantityRows": 1, "quantityColumns": 2, "verticalSeparation": 0.2, 
"horizontalSeparation": 0.2.

\begin{figure}
 \centering
 \includegraphics[width=5cm]{gfx/FrontAntenna2.png}
 \caption{Frente de antena de dos elementos radiantes.}
 \label{fig:frontAntenna}
\end{figure}

Una vez obtenida la estructura de la antena, es necesario determinar que componentes van a tener desvíos en el comportamiento 
deseado. Para ello, se agrega una lista indicando que componentes se desea que tengan errores. Los errores tienen una 
distribución gaussiana con media 0. El desvío estandar es configurable.
% \todo[inline]{pablo, saco o dejo esta parte??}

Para el caso en que se desee realizar la calibración en el modo Completo, se debería agregar errores de planitud de la antena.
Para ello, los errores se verían reflejados en el archivo de la configuración del panel de la antena. De todas formas, a modo 
como el modelo se realizó para validar el método, solamente se realizó el modelo de la planitud ideal.

\subsection{Calibrador}

En la figura \ref{fig:modelPackage} se muestra el diagrama de clases del calibrador del modelo de antena. En el mismo se pueden 
observar que está la clase Antena que pertenece al modelo y tiene todo el comportamiento de todos sus componentes. Los 
calibradores pueden ser el clásico o el de acoplamientos mútuos; el primero utiliza un creador de señales chirp y de códigos
walsh, en cambio, el segundo utiliza clases, del tipo MatrixCalibratorBuilder, que tienen distintas estrategias para generar 
ecuaciones en base a los lazos de calibración generados.

\begin{figure}
 \centering
 \includegraphics[width=15cm]{gfx/modelPackage.png}
 \caption{Diagrama de clases del calibrador de antena.}
 \label{fig:modelPackage}
\end{figure}


A la hora de calibrar, los primeros parámetros a definir son la potencia (en dB), fase (en grados) y frecuencia (en Hz) de la 
señal utilizada con que se alimenta la antena para ser transmitida. A su vez, se debe definir el apuntamiento deseado. Para 
ello, se tienen que configurar la fase de los defasadores de cada elemento radiante de la polarización a transmitir tomando 
en cuenta lo explicado en la sección \ref{ssec:beamSteering}. Estos valores están definidos en grados bajo el nombre de 
\enquote*{Row Steering} y \enquote*{Column Steering}. Los mismos se traducen como el apuntamiento vertical y horizontal 
respectivamente con respecto a la antena.

Una vez definidos los parámetros anteriormente mencionados, se puede obtener la ganancia de la antena en las partes de 
transmisión y recepción para crear los lazos de calibración. Dichas ganancias están definidas como matrices de parámetros
S y se las obtiene realizando los siguientes pasos.

\begin{enumerate}
	\item Se recorre recursivamente la estructura de la antena, se van leyendo los parámetros S de cada componente y se va 
		armando una lista de caminos distintos.
	\item Si el componente tiene más de un puerto de salida, se agranda la lista de elementos encontrados. La lista final tiene 
		tantos elementos como RMs tiene la antena.
	\item A la matriz leída se la transforma en parámetros T utilizando la transformación definida en la sección 
		\ref{ssec:conversion} y se la multiplica con los datos guardados en la lista de elementos encontrados. Se realiza esto 
		por la propiedad de multiplicación entre matrices de transferencia definida en la sección \ref{ssec:transMatrix}.
	\item Una vez recorrida toda la antena, la lista de parámetros T se la convierte nuevamente a parámetros S.
	\item Si los parámetros S son para recepción de la antena, se deben intercambiar los elementos de la matriz. $S_{11}$ con 
		$S_{22}$ y $S_{21}$ con $S_{12}$.
\end{enumerate}

Para armar los lazos de calibración, se tiene que tener en cuenta distintos tipos de errores que pueden afectar a los mismos, 
los cuales son: error de ganancia entre pulsos, error de fase entre pulsos, error de ganancia de la chirp replica, error de 
fase de la chirp replica y error de fase del walsh. Todos estos errores son tratados como gaussianos de media igual a cero.

\todo{donde poner la definicion de estos errores previamente mencionados??????}
Error de ganancia o fase entre pulsos: Este error se puede dar por inestabilidades en el oscilador local que genera la señal a 
transmitir/calibrar la antena.

Error de ganancia o fase de la chirp replica: Este error se da por el cambio de comportamiento del lazo de calibración donde se
mide la chirp réplica, como dicho lazo está caracterizado, cuando cambia su valor de ganancia, se le atribuye dicho cambio a la
chirp en vez de al lazo.

Error de fase del walsh: Este error aparece a causa de las diferencias de comportamiento real y configurado en los defasadores. 

Los lazos de calibración dependen del método utilizado, los cuales se explicarán a continuación.

\subsubsection{Calibración clásica}
Para armar los datos a ser calibrados, todos los lazos se los multiplica con tantas chirps como el método indica, con los errores
que deba tener (según configuración) y con la codificación del código walsh siguiendo lo especificado en la sección 
\ref{ssc:classicalMethod}. El resultado es una matriz adquirida, la cual, para decodificar y obtener la ganancia/fase de un TRM,
se lo debe multiplicar por un array de chirp réplicas defasadas según la columna del código Walsh al que perteneciente el TRM. 

\subsubsection{Calibración con acoplamientos mútuos}

Para obtener los distintos lazos de calibración se desarrollaron varias estrategias a saber: 

\begin{itemize}
	\item \textbf{LinearBuilder:} Esta estrategia construye ecuaciones utilizando la resta de dos lazos que tengan un elemento en 
		común, llamado central.Estos lazos tienen distancias equidistantes al elemento en común siguendo la regla quchas distancias
		constituyen una misma recta de unión. La figura \ref{fig:linealBuilder} muestra varios ejemplos de esta estrategia.
			
		\begin{figure}[H]
		 \centering
		 \includegraphics[width=2cm]{gfx/FrontAntenna2.png}
		 \caption{Estrategia LinearBuilder.}
		 \label{fig:linealBuilder}
		\end{figure}

	\item \textbf{CrossBuilder:} Esta estrategia construye ecuaciones utilizando la resta de dos lazos que tengan un elemento en 
		común, llamado central. Estos lazos tienen distancias equidistantes al elemento en común siguiendo la regla que dichas 
		distancias son perpendiculares entre sí. La figura \ref{fig:crossBuilder} muestra varios ejemplos de esta estrategia.
			
		\begin{figure}[H]
		 \centering
		 \includegraphics[width=2cm]{gfx/FrontAntenna2.png}
		 \caption{Estrategia CrossBuilder.}
		 \label{fig:crossBuilder}
		\end{figure}

	\item \textbf{TinyBuilder:} Esta estrategia construye ecuaciones solamente si la antena tiene dos elementos como filas o 
		columnas y utiliza los cuatro lazos de calibración de los elementos que consituyen dicha fila/columna sumando los 
		transmitidos de un elemento al otro y restando los del otro al primero. De esta forma se obtiene la doble diferencia de los 
		elementos el Tx o Rx. La figura \ref{fig:tinyBuilder} muestra un ejemplo de esta estrategia.
		
		\begin{figure}[H]
		 \centering
		 \includegraphics[width=2cm]{gfx/FrontAntenna2.png}
		 \caption{Estrategia TinyBuilder.}
		 \label{fig:tinyBuilder}
		\end{figure}

	\item \textbf{DoubleBuilder:} Esta estrategia construye ecuaciones solamente si la antena no tiene dos elementos en ninguna de 
		sus dimensiones y no solo utiliza los cuatro lazos de calibración que hay entre todos los pares de elementos que constituyen
		la antena (en línea vertical y horizontal), sino que también utiliza todos los pares de elementos incluídos en la línea que 
		une los primeros dos que sean equidistantes al primer par. La figura \ref{fig:doubleBuilder} muestra un ejemplo de esta estrategia.
		
		\begin{figure}[H]
		 \centering
		 \includegraphics[width=2cm]{gfx/FrontAntenna2.png}
		 \caption{Estrategia DoubleBuilder.}
		 \label{fig:doubleBuilder}
		\end{figure}

	\item \textbf{DefaultBuilder:} Esta estrategia se la utiliza para agregar las ecuaciones donde se mide la potencia 
		transmitida/recibida para que el resultado del método sea único. Por ejemplo, puede ser el resultado de la medición de un 
		TRM obtenido de la calibración clásica.
\end{itemize}


Una vez obtenidas todas las ecuaciones de las calibraciones, se procede a utilizar el método de cuadrados mínimos definido en 
la sección \ref{sec:meanSquare}. Para la fase, hay que tener extremo cuidado con que es de naturaleza modular, por lo tanto 
previamente a la realización del cálculo de los cuadrados mínimos, hay que acomodar el arreglo de los resultados de las 
ecuaciones previamente mencionadas, sino el resultado obtenido no va a ser el correcto.

Para realizar la corrección de fase, se utiliza el valor obtenido con la caracterización de lo que se defasa la señal al ser 
transmitida/recibida con la antena. Este valor, junto al defasaje configurado de cada defasador, son utilizados para obtener el
resultado que debería asemejarse cada resultado de las ecuaciones a las que se le va a aplicar cuadrados mínimos. 

\todo{agregar el grafico de la composicion de una antena}


\subsection{Configuraciones del sistema}

A modo de resumen, se lista en la tabla \ref{tab:conf_modelo_antena} todas las posibles configuraciones en el modelo para 
realizar los distintos ensayos.

\begin{center}
  \footnotesize
  \centering
  \begin{longtable}{|c|p{9cm}|}
    \hline 
	\multicolumn{2}{|c|}{\textbf{Parámetros de entrada}} \\ 
	\hline
    Frequencia		& Es la frecuenia central de trabajo en Hz \tabularnewline \hline 
    Potencia		& Potencia con la que se alimenta la antena en db \tabularnewline \hline 
    Fase			& Fase con la que se alimenta la antena en grados \tabularnewline \hline 
    Row Steering	& Apuntamiento horizontal que se le quiere dar al beam de salida, en grados  \tabularnewline \hline 
    Column Steering	& Apuntamiento vertical que se le quiere dar al beam de salida, en grados  \tabularnewline \hline 
	\multicolumn{2}{|c|}{\textbf{Parámetros de calibración}} \\ 
	\hline
	potencia Tx deseada	& Potencia de transmisión deseada para calbirar  \tabularnewline \hline 
	potencia Rx deseada	& Potencia de recepción deseada para calbirar  \tabularnewline \hline 
	Errores	& Son los errores referentes a la hora de calibrar el modelo de antena. Pueden ser: interPulseGainChirpError, interPulsePhaseChirpError, gainChirpRepError, phaseChirpRepError, WalPhaseErrors  \tabularnewline \hline
	\multicolumn{2}{|c|}{\textbf{Parámetros de Antena}} \\ 
	\hline
	Cantidad Filas	& Da la cantidad de módulos radiantes en dirección vertical \tabularnewline \hline 
	Cantidad Columnas	& Da la cantidad de módulos radiantes en dirección horizontal \tabularnewline \hline 
	Separación vertical & Es la separación vertical entre RMs \tabularnewline \hline 
	Separación horizontal & Es la separación horizontal entre RMs \tabularnewline \hline 
	Secuencia de componentes & Es la secuencia de componentes que conforma la RFDN, los mismos pueden ser: cables, psc, trm, circulador, rm \tabularnewline \hline 
	Errores  & Son los componentes de la antena que pueden tener errores. Los mismos pueden ser: RMError, TRMError, CirculatorError, PSCError \tabularnewline \hline 
	TRMs muertos & Es una lista que indica que trms están muertos en el panel de la antena. \tabularnewline \hline 
	\multicolumn{2}{|c|}{\textbf{Componentes de Antena}} \\
	\hline
	$cable_i$ & Se pueden definir tantos cables como se deseen, los parámetros a definir son: attenuation [db], length [m], type = cable \tabularnewline \hline 
	$trm_i$ & Se pueden definir tantos TRMs como se deseen, los parámetros a definir son: gain [db], phaseShift [deg], type = TRM \tabularnewline \hline 
	$psc_{1j}$ & Se pueden definir tantos PSC como se deseen, los parámetros a definir son: outputPorts = $j$, type = PSC1$j$ \tabularnewline \hline 
	circulator & aca se puede definir un circulador, el parámetro a definir es: type = circulator \tabularnewline \hline 
	$rm$ & Se puede definir un RM, el parámetro a definir es: type = RM \tabularnewline \hline 
	\multicolumn{2}{|c|}{\textbf{Desvío estandard del error}} \\
	\hline
	Error del cable & Desvío estandard de los cables. \tabularnewline \hline 
	Error del circulador & Desvío estandard de los circuladores. \tabularnewline \hline 
	Error del TRM & Desvío estandard de los TRMs. \tabularnewline \hline 
	Error del PSC & Desvío estandard de los PSC. \tabularnewline \hline 
	Error del RM & Desvío estandard de los RM. \tabularnewline \hline 
	Error de ganancia entre pulsos & Desvío estandard de ganancia entre pulsos de calibración. \tabularnewline \hline 
	Error de fase entre pulsos & Desvío estandard de fase entre pulsos de calibración. \tabularnewline \hline 
	Error de ganancia de la chirp replica & Desvío estandard de ganancia de la chirp replica. \tabularnewline \hline 
	Error de fase de la chirp replica & Desvío estandard de fase de la chirp replica. \tabularnewline \hline 
	Error de fase del walsh & Desvío estandard de fase del seteo de los defasadores en calibración. \tabularnewline \hline 
	\caption{Configuraciones del modelo de antena}
  \end{longtable}
  \label{tab:conf_modelo_antena}
\end{center}

A su vez, también se puede configurar que tipo de calibración se desea correr para poder obtener los resultados y se pueden 
graficar tanto los patrones de antena obtenidos en algún corte corte de azimuth/rango o el arreglo de ganancias y fases con 
respecto al valor ideal. La figura \ref{fig:visualPackage} muestra el diagrama de clases del visualizador.

\begin{figure}
 \centering
 \includegraphics[width=11cm]{gfx/visualPackage.png}
 \caption{Diagrama de clases de los visualizadores.}
 \label{fig:visualPackage}
\end{figure}

