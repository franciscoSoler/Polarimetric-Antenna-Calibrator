\chapter{Modelización}
\label{ch:modelizacion}
\lhead{\emph{Modelización}}


En este capítulo se utiliza toda la información descripta en los capítulos anteriores:
\begin{itemize}
	\item El modelo de la antena y de todos sus componentes en parámetros S descriptos en el capítulo \ref{ch:phasedArray}.
	\item El modelo matemático y algorítmico de la calibración clásica \ref{ch:classicalCalibration}.
	\item El modelo matemático y algorítmico de la calibración por acoplamientos mutuos \ref{ch:mutualCalibration}.
\end{itemize}
para definir las especificaciones del problema completo y el diseño de la arquitectura de software que cubra dichas especificaciones. 


\section{Especificaciones del problema} \label{sc:specifications}

Para poder aplicar e implementar el método de acoplamientos mutuos se deben cumplir las siguientes hipótesis de base, que a su
vez son consideradas las especificaciones del problema.

\begin{enumerate}
    \item Modularización de los componentes de la antena: la antena debe estar compuesta por desfasadores, atenuadores, cables,
		divisores de potencia y elementos radiantes, que son los componentes que conforman, interconectados entre sí, la antena. Estos
		elementos deben ser modelizados de manera modular de forma de poder intercambiarlos entre sí para poder simular diferentes
		topologías.

	\item Modelización de los componentes en RF: se desea que el comportamiento del modelo de antena que se construye sea 
		representativo de su comportamiento en RF desde el punto de vista de los efectos sobre la señal de interés.

    \item Sistemas LTI: la aplicación deberá reproducir el comportamiento del sistema, el cual será lineal e invariante
		en el tiempo.

	\item Adaptación de impedancias: todos los componentes se encuentran perfectamente adaptados.

    \item Variabilidad para determinados componentes: se deben poder utilizar distintos divisores/combinadores de potencia.
		La diferencia entre ellos es la cantidad de puertos de salida/entrada.

    \item Dimensión de antena: se debe poder configurar la cantidad de elementos radiantes por columna y fila de la antena.
    \item Distancia entre elementos radiantes: Se debe poder configurar la distancia entre elementos radiantes.

    \item Configuración individual de componentes: Se debe poder configurar los atributos que afecten la modelización de cada
		componente de la antena; largo, atenuación y desfasaje por metro de cables.

    \item Modelizacion de dispersiones: se debe poder configurar las dispersiones en el comportamiento de cada componente de la
		antena de forma independiente.

    \item Introducción de elementos de control en el lazo de calibración: se deben poder configurar los atenuadores y
		defasadores a la hora de realizar la calibración.

    \item Calibración clásica: sa aplicación debe poder calibrar una antena polarimétrica con el método de calibración clásico.

    \item Calibración por acoplamientos mutuos: sa aplicación debe poder calibrar una antena polarimétrica con el método de
		calibración propuesto.

    \item Calibración de ganancia: la aplicación debe poder calibrar la ganancia de transmisión y recepción.

    \item Calibración de fase: la aplicación debe poder calibrar la fase de transmisión y recepción.

    \item Calibración en polarización horizontal: la aplicación debe poder calibrar en la polarización horizontal.
    \item Calibración en polarización vertical: la aplicación debe poder calibrar en la polarización vertical.
    \item Calibración en Tx: la aplicación debe poder calibrar en transmisión.
    \item Calibración en Rx: la aplicación debe poder calibrar en recepción.

    \item Calibración independiente del estado inicial: se debe poder alcanzar el estado de calibración deseado partiendo
		cualquier estado inicial en los desfasadores y atenuadores.

    \item Transmisión vs recepción por polarizaciones cruzadas: para calibrar se debe transmitir y recibir en polarizaciones
		diferentes.

    \item Planitud de antena: la antena tiene que ser perfectamente plana. No deben haber imperfecciones.

    \item Frecuencia de RF: se debe poder configurar la frecuencia de trabajo.

    \item Dispersión señal de calibración entre pulsos: se debe poder configurar los parámetros de dispersión (desvío estándar) para
		la ganancia y fase de la chirp utilizada entre pulsos.

    \item Dispersión chirp réplica: se debe poder configurar los parámetros de dispersión (desvío estándar) para la
		ganancia y fase de la chirp réplica utilizada a la hora de realizar la calibración convencional.

    \item Dispersión walsh: se debe poder configurar los parámetros de dispersión (desvío estándar) para los valores de fase
		utilizados en los códigos walsh a la hora de realizar la calibración convencional.

    \item Simulación falla componente: se debe poder simular, configurar la destrucción total de los TRMs o RMs de la antena.
\end{enumerate}


\section{Modelo de la antena}

El modelo de antena está programado de tal forma que se lo pueda utilizar para simular una antena con parámetros de RF. La
misma está conformada por los siguientes componentes: cables, PSCs, TRMs, desfasadores y elementos radiantes. Cada uno de estos
está modelado con los conectores asociados que permiten interconectarlos entre sí. Además se modeliza las interacciones entre
ellos por medio de los acoplamientos entre elementos radiantes. 

Como se van a modelar desvíos en el comportamiento de los distintos componentes de la antena debido a diversos factores, en
algunos casos se utilizará la metodología montecarlo para estudiar el comportamiento de los calibradores.

Como es necesario realizar ensayos de montecarlo utilizando distintos tipos de calibraciones ante una misma configuración y 
comportamiento de antena, dicha informacion debe persistir entre corrida y corrida, por lo tanto se optó por guardar en disco
dicha información como archivos con formato json. En los cuales se guarda la configuración y propiedades físicas tanto del
panel como de la RFDN. Las propiedades del primer grupo son las distancias entre todos los elementos. Las del segundo grupo
corresponden a la matriz de parámetros S, desarrollado en el capítulo \ref{ch:phasedArray}, y la interconexión entre
componentes. En el apéndice \ref{AppendixC} se muestra se muestra un ejemplo sencillo, para comprender como el modelo
matemático se ha modelizado, en este caso para una antena muy básica de dos elementos radiantes. Tener en cuenta que la antena
a modelizar tiene más de un orden de magnitud mayor de tamaño.

La aplicación se divide en tres etapas principales. La primera parte es la encargada de la modelización de la antena y sus
componentes. La segunda contiene el copmortamiento de ambos modelos de calibración interna, los cuales se corresponden con las
distintas estrategias de calibración de la UCC y la tercera consta de los visualizadores de resultados. La figura
\ref{fig:prog_inic} muestra el flujo de ejecuci\'on del programa.

\begin{figure}
 \centering
 \includegraphics[width=11cm]{gfx/FlujoEjecucion.png}
 \caption{Flujo de ejecuci\'on del modelo de antena.}
 \label{fig:prog_inic}
\end{figure}

A continuación se describirán cada uno de los bloques del flujo de ejecución de la figura \ref{fig:prog_inic}

\subsection{Generador de Antena (AntennaGenerator)}

La antena bajo estudio, con todos todos los controladores de fase y atenuacion (TRMs), en idéntica posicion, deben generar una
onda plana saliente, transmitida, por los elementos radiantes. De idéntica forma, una onda plana recibida, con todos los
receptores de los TRMs en igual posicion de fase y atenuacion, debe dar lugar a una señal coherente como resultado de la
sumatoria. Por lo tanto, es necesario que todos los caminos entre el punto donde se inyecta o recibe la señal y los módulos 
radiantes sean iguales. Esta característica es análoga a la de un árbol balanceado. 

La estructura interna de la antena queda definida listando los elementos que la componen y el armado de la antena respeta el
orden de aparición de elementos en la lista. Por ejemplo, si la lista es [\enquote*{cable1}, \enquote*{PSC12},
\enquote*{cable2}], la RFDN tiene conectado al transmisor/receptor el componente \enquote*{cable1}, a su vez, el puerto común
del \enquote*{PSC12} está conectado al mismo elemento y cada uno de sus otros puertos están conectados a un \enquote*{cable2}
distinto. En la figura \ref{fig:2RMAntenna} se muestra la antena que se construye teniendo como configuración:
[\enquote*{cable1}, \enquote*{psc12}, \enquote*{cable2}, \enquote*{trm}, \enquote*{circulator}, \enquote*{cable3},
\enquote*{rm}].

\begin{figure}
 \centering
 \includegraphics[width=10cm]{gfx/RFDN.png}
 \caption{Estructura interna de una de las polarizaciones de una antena con dos elementos radiantes.}
 \label{fig:2RMAntenna}
\end{figure}

Una vez obtenida los componentes que determinan la estructura de la antena, es necesario definir las características físicas de
cada uno de ellos. La tabla \ref{tab:propertiesOfComponents} lista dichas las propiedades.

\begin{table}[H]
  \footnotesize
  \centering
  \begin{tabular}{|c|c|}
	\hline
	\textbf{Componente de Antena} & \textbf{Carcterísticas físicas} \tabularnewline \hline 
	\multirow{2}{*}{cable} &  attenuation [db] \tabularnewline \cline{2-2}
	 & length [m] \tabularnewline \hline 
	\multirow{3}{*}{TRM} & isDead \tabularnewline \cline{2-2}
	 & gain [db]\tabularnewline \cline{2-2}
	 & phaseShift [deg] \tabularnewline \hline 
	PSC1$j$ & outputPorts = $j$ \tabularnewline \hline 
	circulator & \tabularnewline \hline 
	RM & \tabularnewline \hline 
  \end{tabular}
  \caption{Parámetros de cada componente de una antena.}
  \label{tab:propertiesOfComponents}
\end{table}

Como es necesario que la representación de las propiedades mencionadas previamente sean representativas en radio frecuencia,
se utilizan los parámetros S (como se explicó y detalló en el capítulo \ref{ch:phasedArray}). En la figura \ref{fig:creationPackage} 
se define un diagrama de clases, realizado ad hoc para esta aplicación donde se puede observar que la clase \enquote*{AntennaCreator}
hace uso de los generadores de parámetros S de cada componente.

\begin{figure}
 \centering
 \includegraphics[width=15cm]{gfx/creationPackage.png}
 \caption{Diagrama de clases del generador de antena.}
 \label{fig:creationPackage}
\end{figure}

Al determinar la RFDN, si bien está definida la cantidad de elementos radiantes, falta definir las dimensiones de la antena, 
en otras palabras, la cantidad de columnas de ER, o de paneles, la cantidad de filas de ER, o la cantidad de elementos por
panel, y la separación, tanto horizontal como vertical, entre elementos. 

Como dicha información es redundante y posiblemente incompatible con la definición de la estructura interna de la antena, es 
necesario un chequeo de compatibilidad. La figura \ref{fig:frontAntenna} muestra el frente de una antena que se construye 
teniendo como configuración: \enquote*{quantityRows}: 1, \enquote*{quantityColumns}: 2, \enquote*{verticalSeparation}: 0.2 y
\enquote*{horizontalSeparation}: 0.2.

\begin{figure}
 \centering
 \includegraphics[width=3cm]{gfx/FrontAntenna2.png}
 \caption{Frente de antena de dos elementos radiantes.}
 \label{fig:frontAntenna}
\end{figure}

Una vez creada la estructura completa de la antena, es necesario determinar que componentes van a tener desvíos en el
comportamiento deseado. Para ello, se agrega una lista indicando los componentes con errores. Para las incertidumbres se define
una distribucion gaussiana con media nula por las caracteristicas de los componentes representados. El desvío estándar es 
configurable.

Si se deseara realizar la calibración en modo completo, se deberían agregar errores de planitud de la antena. Con lo cual, 
dichos desvíos se verían reflejados en el archivo de la configuración del panel de la antena. En la presente tesis solamente 
se desarrolló el modelo con planitud ideal.


\subsection{Calibrador}

En la figura \ref{fig:modelPackage} se muestra el diagrama de clases del calibrador del modelo de antena. Se puede observar que
la clase \enquote*{Antena}, la cual pertenece al modelo, modeliza el comportamiento de todos sus componentes. Los calibradores
pueden ser el clásico o el de acoplamientos mutuos; el primero utiliza un creador de señales chirp y de códigos walsh, en
cambio, el segundo utiliza clases del tipo MatrixCalibratorBuilder, las cuales tienen distintas estrategias para generar
ecuaciones en base a los lazos de calibración generados.

\begin{figure}
 \centering
 \includegraphics[width=15cm]{gfx/modelPackage.png}
 \caption{Diagrama de clases del calibrador de antena.}
 \label{fig:modelPackage}
\end{figure}


A la hora de calibrar, los primeros parámetros a definir son la potencia (en dB), fase (en grados) y frecuencia (en Hz) de la 
señal utilizada con que se alimenta la antena para ser transmitida. A su vez, se debe definir el apuntamiento deseado. Para 
ello, se tienen que configurar la fase de los desfasadores en transmisión de la polarización a transmitir, tomando en cuenta lo 
explicado en la sección \ref{ssec:beamSteering}. Estos valores están definidos en el archivo de configuración bajo el nombre 
de \enquote*{Row Steering} y \enquote*{Column Steering}. Los mismos se traducen como el apuntamiento vertical y horizontal 
respectivamente de la antena.

Una vez definidos los parámetros previamente mencionados, se puede obtener la ganancia de la antena en las partes de 
transmisión y recepción para crear los lazos de calibración. Dichas ganancias están definidas como matrices de parámetros
S y se las obtiene realizando los siguientes pasos.

\begin{enumerate}
	\item Se recorre recursivamente la estructura de la antena, se van leyendo los parámetros S de cada componente y se va 
		armando una lista de caminos distintos.
	\item Si el componente tiene más de un puerto de salida, se agranda la lista de elementos encontrados. La lista final tiene 
		tantos elementos como RMs tiene la antena.
	\item A la matriz leída se la transforma en parámetros T utilizando la transformación definida en la sección 
		\ref{ssec:conversion} y se la multiplica con los datos guardados en la lista de elementos encontrados. Se realiza esto 
		por la propiedad de multiplicación entre matrices de transferencia definida en la sección \ref{ssec:transMatrix}.
	\item Una vez recorrida toda la antena, la lista de parámetros T se la convierte nuevamente a parámetros S.
	\item Si los parámetros S son para recepción de la antena, se deben intercambiar los elementos de la matriz. $S_{11}$ con 
		$S_{22}$ y $S_{21}$ con $S_{12}$.
\end{enumerate}


Para armar los lazos de calibración, se tiene que tener en cuenta tanto los distintos tipos de errores que pueden afectar a los
mismos, como el método de calibración utilizado. 

\subsubsection{Casos de incertidumbres}

Los posibles errores que afectan directamente a los calibradores son: error de ganancia entre pulsos, error de ganancia de la
chirp replica y error de fase del walsh. Todos estos errores son tratados como gaussianos con media cero. A continuación se
describen cada uno de ellos:
\begin{itemize}
	\item Desvíos de ganancia o fase entre pulsos: Esta incertidumbre se puede dar por inestabilidades en el oscilador local que
		genera la señal a transmitir/calibrar la antena o incertidumbres de medición del receptor.
	\item Desvíos de ganancia o fase de la chirp replica: Esta incertidumbre se da por el cambio de comportamiento del lazo de
		calibración donde se mide la chirp réplica, como dicho lazo está caracterizado, cuando cambia su valor de ganancia, esta
		variación es atribuida a la chirp en vez de al lazo.
	\item Desvíos de fase del walsh: Esta incertidumbre aparece a causa de las diferencias de comportamiento real y configurado en
		los desfasadores. 
\end{itemize}


\subsubsection{Calibración interna clásica}

Para armar los datos a ser calibrados, todos los lazos (que abarcan hasta los TRMs de la RFDN) se los multiplica con 
tantas chirps como el método indica, con los errores que deba tener (según configuración) y con la codificación del código 
walsh siguiendo lo especificado en la sección \ref{ssc:classicalMethod}. El resultado es una matriz adquirida, la cual, para
decodificar y obtener la ganancia/fase de un TRM, se lo debe multiplicar por un conjunto de chirp réplicas desfasadas según la 
columna del código Walsh al que perteneciente el TRM. 


\subsubsection{Calibración interna por acoplamientos mutuos}

Para obtener los distintos lazos de calibración se desarrollaron varias estrategias a saber: 

\begin{itemize}
	\item \textbf{LinearBuilder:} Esta estrategia construye ecuaciones utilizando la resta de dos lazos que tengan un elemento en 
		común, llamado central. Dicho elemento puede ser utilizado para transmitir o recibir, dependiendo si se desea obtener una 
        ecuación de Rx o Tx respectivamente. El método requiere que las distancias del elemento en común al resto sean iguales
		para poder eliminar los acoplamientos mutuos y que los tres componentes estén alineados entre sí.
         
		Para ejemplificar, en la figura \ref{fig:linealBuilder} se pueden observar dos pares de lazos de calibración. Si se
		restan los grises oscuros, aprovechando que $C_{14} = C_{74}$, se obtiene una ecuación de la rama de transmisión; en
		cambio, si se restan los grises claros, aprovechando que $C_{53} = C_{57}$, se obtiene una ecuación de la rama de
		recepción de la antena.
			
		\begin{figure}[H]
		 \centering
		 \includegraphics[width=5cm]{gfx/linearBuilder.png}
		 \caption{Estrategia LinearBuilder. Obtiene la relación de lazos de calibración equidistantes con un elemento en común
		 perteneciente a la línea de unión del resto de los ER.}
		 \label{fig:linealBuilder}
		\end{figure}

	\item \textbf{CrossBuilder:} Esta estrategia construye ecuaciones utilizando la resta de dos lazos que tengan un elemento en 
		común, llamado central. Dicho elemento puede ser utilizado para transmitir o recibir como en el LinearBuilder. La única 
        diferencia es que en vez de requerir que los tres elementos estén alineados, tienen que formar una L. En otras palabras,
        las rectas de unión deben ser perpendiculares entre sí.
		
        Para ejemplificar, en la figura \ref{fig:crossBuilder} se pueden observar dos pares de lazos de calibración. Si se
		restan los grises claros, aprovechando que $C_{12} = C_{14}$, se obtiene una ecuación de la rama de recepción. En cambio, 
        si se restan los grises oscuros, aprovechando que $C_{53} = C_{59}$, se obtiene una ecuación de la rama de transmisión de
		la antena.

		\begin{figure}[H]
		 \centering
		 \includegraphics[width=5cm]{gfx/crossBuilder.png}
		 \caption{Estrategia CrossBuilder.}
		 \label{fig:crossBuilder}
		\end{figure}

	\item \textbf{TinyBuilder:} Esta estrategia construye ecuaciones solamente si la cantidad de elementos en alguna dirección, 
        horizontal o vertical, es igual a dos. Utilizando los cuatro lazos de calibración posibles por cada par de elementos 
		que hay en la dirección previamente mencionada, resta los lazos transmitidos por un elemento con los del otro. El resultado 
		es la resta de las dos ramas de transmisión. Para obtener ecuaciones de la diferencia en recepción el razonamiento es 
		análogo, se deben restar los lazos recibidos de uno con los del otro elemento.

        Para ejemplificar, la figura \ref{fig:tinyBuilder} muestra una antena de dos elementos en configuración vertical. En 
        esta estrategia se restan los caminos grises oscuros con los claros aprovechando que $C_{11} = C_{22}$ y que
		$C_{12} = C_{21}$.
		
		\begin{figure}[H]
		 \centering
		 \includegraphics[width=10cm]{gfx/tinyBuilder.png}
		 \caption{Estrategia TinyBuilder.}
		 \label{fig:tinyBuilder}
		\end{figure}

        Nota: En la figura \ref{fig:tinyBuilder} se puede llegar a la interpretación que hay 4 elementos radiantes en vez 
        de 2. Cada elemento fue duplicado para enfatizar que la polarización de transmisión es diferente a la de recepción.

	\item \textbf{DoubleBuilder:} Esta estrategia construye ecuaciones solamente si la cantidad de elementos radiantes de la 
        antena, en cualquiera de sus direcciones, es distinta de dos. A diferencia de la estrategia TinyBuilder, también se 
        utilizan las ramas de recepción de elementos que tengan las mismas distancias a los elementos transmisores.
        
        Para ejemplificar, se tiene una antena de 5 elementos en configuración vertical, como se muestra en la figura 
		\ref{fig:doubleBuilder}. Esta estrategia resta los caminos grises oscuro con los claros aprovechando que
		$C_{12} = C_{54}$ y que $C_{14} = C_{52}$ (por simetría de la antena), dando como resultado dos veces la resta de las
		ramas de transmisión.
		
		\begin{figure}[H]
		 \centering
		 \includegraphics[width=10cm]{gfx/doubleBuilder.png}
		 \caption{Estrategia DoubleBuilder. Al restar los lazos gris oscuro con los claros queda solamente la resta de la cadena
			 en transmisión.}
		 \label{fig:doubleBuilder}
		\end{figure}

        Nota: En la figura \ref{fig:doubleBuilder} se puede llegar a la interpretación que hay 10 módulos radiantes en vez 
        de 5. Cada elemento fue duplicado para enfatizar que la polarización de transmisión es diferente a la de recepción.

	\item \textbf{DefaultBuilder:} Esta estrategia se la utiliza para agregar las ecuaciones donde se mide la potencia 
		transmitida/recibida para que el resultado del método sea único. Por ejemplo, puede ser el resultado de la medición de un 
		TRM obtenido de la calibración clásica.
\end{itemize}

Una vez obtenidas todas las ecuaciones de las calibraciones, se procede a utilizar el método de cuadrados mínimos definido en 
la sección \ref{sec:meanSquare}. Para la fase, hay que tener extremo cuidado por su naturaleza modular, por lo tanto, 
previamente a la realización del cálculo de cuadrados mínimos, se acomoda el conjunto de resultados de acuerdo a la sección 
\ref{ssc:mutualPhase}.


\subsection{Configuraciones del sistema}

A modo de resumen, se lista en la tabla \ref{tab:conf_modelo_antena} todas las posibles configuraciones en el modelo para 
realizar los distintos ensayos.

\begin{center}
  \footnotesize
  \centering
  \begin{longtable}{|c|p{9cm}|}
    \hline 
	\multicolumn{2}{|c|}{\textbf{Parámetros de entrada}} \\ 
	\hline
    Frequencia		& Es la frecuencia central de trabajo en Hz \tabularnewline \hline 
    Potencia		& Potencia con la que se alimenta la antena en db \tabularnewline \hline 
    Fase			& Fase con la que se alimenta la antena en grados \tabularnewline \hline 
    Row Steering	& Apuntamiento horizontal que se le quiere dar a la señal transmitida, en grados  \tabularnewline \hline 
    Column Steering	& Apuntamiento vertical que se le quiere dar a la señal transmitida, en grados  \tabularnewline \hline 
	\multicolumn{2}{|c|}{\textbf{Parámetros de calibración}} \\ 
	\hline
	potencia Tx deseada	& Potencia de transmisión deseada para calibrar  \tabularnewline \hline 
	potencia Rx deseada	& Potencia de recepción deseada para calibrar  \tabularnewline \hline 
	Errores	& Son los errores referentes a la hora de calibrar el modelo de antena. Pueden ser: interPulseGainChirpError, 
	interPulsePhaseChirpError, gainChirpRepError, phaseChirpRepError, WalPhaseErrors  \tabularnewline \hline
	\multicolumn{2}{|c|}{\textbf{Parámetros de Antena}} \\ 
	\hline
	Cantidad Filas	& Da la cantidad de módulos radiantes en dirección vertical \tabularnewline \hline 
	Cantidad Columnas	& Da la cantidad de módulos radiantes en dirección horizontal \tabularnewline \hline 
	Separación vertical & Es la separación vertical entre RMs \tabularnewline \hline 
	Separación horizontal & Es la separación horizontal entre RMs \tabularnewline \hline 
	Secuencia de componentes & Es la secuencia de componentes que conforma la RFDN, los mismos pueden ser: cables, psc, trm, circulador, rm \tabularnewline \hline 
	Errores  & Son los componentes de la antena que pueden tener errores. Los mismos pueden ser: CableError, RMError, TRMError, CirculatorError, PSCError 
	\tabularnewline \hline 
	TRMs muertos & Es una lista que indica que trms no responden a ningún estímulo externo o que poseen una ganancia casi nula. \tabularnewline \hline 
	\multicolumn{2}{|c|}{\textbf{Componentes de Antena}} \\
	\hline
	$cable_i$ & Se pueden definir tantos cables como se deseen, los parámetros a definir son: attenuation [db], length [m], type = cable \tabularnewline \hline 
	$trm_i$ & Se pueden definir tantos TRMs como se deseen, los parámetros a definir son: gain [db], phaseShift [deg], type = TRM \tabularnewline \hline 
	$psc_{1j}$ & Se pueden definir tantos PSC como se deseen, los parámetros a definir son: outputPorts = $j$, type = PSC1$j$ \tabularnewline \hline 
	circulator & aca se puede definir un circulador, el parámetro a definir es: type = circulator \tabularnewline \hline 
	$rm$ & Se puede definir un RM, el parámetro a definir es: type = RM \tabularnewline \hline 
	\multicolumn{2}{|c|}{\textbf{Desvío estándar del error}} \\
	\hline
	Error del cable & Desvío estándar de los cables. \tabularnewline \hline 
	Error del circulador & Desvío estándar de los circuladores. \tabularnewline \hline 
	Error del TRM & Desvío estándar de los TRMs. \tabularnewline \hline 
	Error del PSC & Desvío estándar de los PSC. \tabularnewline \hline 
	Error del RM & Desvío estándar de los RM. \tabularnewline \hline 
	Error de ganancia entre pulsos & Desvío estándar de ganancia entre pulsos de calibración. \tabularnewline \hline 
	Error de fase entre pulsos & Desvío estándar de fase entre pulsos de calibración. \tabularnewline \hline 
	Error de ganancia de la chirp replica & Desvío estándar de ganancia de la chirp replica. \tabularnewline \hline 
	Error de fase de la chirp replica & Desvío estándar de fase de la chirp replica. \tabularnewline \hline 
	Error de fase del walsh & Desvío estándar de fase de la configuración de los desfasadores en calibración. \tabularnewline \hline 
	\caption{Configuraciones del modelo de antena}
  \end{longtable}
  \label{tab:conf_modelo_antena}
\end{center}

A su vez, también se puede configurar que tipo de calibración se desea correr para poder obtener los resultados.


\subsection{Visualizador}

La figura \ref{fig:visualPackage} muestra el diagrama de clases del visualizador. Con el mismo se pueden llegar a graficar tanto
los diagramas de radiación obtenidos en un corte configurado, azimuth o rango, como el conjunto de ganancias y fases de cada ER
con respecto al valor ideal. En el capítulo \ref{ch:simulations} se podrán apreciar los gráficos obtenidos de diferentes
simulaciones.

\begin{figure}[H]
 \centering
 \includegraphics[width=11cm]{gfx/visualPackage.png}
 \caption{Diagrama de clases del visualizador.}
 \label{fig:visualPackage}
\end{figure}


\section{Conclusiones}

En este capítulo se introdujo la estructura de la aplicación que modela tanto el comportamiento de los componentes de la antena
como las posibles estrategias a tomar por la UCC para calibrar la misma. Se definieron las posibles incertidumbres y errores que
se pueden simular en las distintos componentes de la antena y las distintas estrategias para armar lazos de calibración para el 
método de calibración interna por acoplamientos mutuos. El modelo desarrollado de la antena es completamente flexible para
definir cualquier tipo de estructura de antena deseada y así de esta forma, estudiar y comparar las distintas topologías. 

Cabe destacar que cada clase del modelo de antena posee tests unitarios que validan sus métodos individualmente. A su vez, se
escribieron tests de integración para asegurar el correcto funcionamiento del simulador.
de integración aplicación donde se simulan funcionales para 

